\documentclass[a4paper,12pt]{article} 
\usepackage[left=2cm,right=2cm,top=2cm,bottom=2cm]{geometry}
%%% Работа с~русским языком
\usepackage{cmap}					% поиск в~PDF
\usepackage[LGR, T2A]{fontenc}			% кодировка
\usepackage[greek, russian]{babel}	% локализация и~переносы
\usepackage{alphabeta}
\usepackage{hyperref}  % для гиперссылок

\hypersetup{
    colorlinks=true,
    linkcolor=black,
    filecolor=magenta,      
    urlcolor=cyan,
}

\title{\textsc{Основы Веры}\\Урок №3.6 Любовь Бога}
\author{Библейская Церковь Санкт-Петербурга}
\date{}
\setcounter{secnumdepth}{0}  % Unnumbered sections in toc

\begin{document}

\maketitle

\thispagestyle{empty}

\tableofcontents

% \newcommand{\myline}{\noindent\makebox[\linewidth]{\rule{\linewidth}{0.1pt}}}


\section*{Введение}

Любовь~--- это слово, которое часто используется без осознания его истинного значения. О~любви часто думают как о~чувстве или влечении, но Писание дает гораздо более глубокое понимание того, что такое любовь. Более того, именно Божья любовь является образцом для всех других форм любви. В~этой части урока мы посмотрим на глубину Божьей любви, которая бесконечна и~самоотверженна.

Рекомендуемое время~--- 20 минут.
        
\subsubsection*{Основные тексты}

\noindent 1 Иоанна 4:7–12

\begin{quote}
\textsubscript{7}~Возлюбленные! будем любить друг друга, потому что любовь от Бога, и~всякий любящий рожден от Бога и~знает Бога. 
\textsubscript{8}~Кто не любит, тот не познал Бога, потому что Бог есть любовь. 
\textsubscript{9}~Любовь Божия к~нам открылась в~том, что Бог послал в~мир Единородного Сына Своего, чтобы мы получили жизнь через Него. 
\textsubscript{10}~В том любовь, что не мы возлюбили Бога, но Он возлюбил нас и~послал Сына Своего в~умилостивление за грехи наши. 
\textsubscript{11}~Возлюбленные! если так возлюбил нас Бог, то и~мы должны любить друг друга. 
\textsubscript{12}~Бога никто никогда не видел. Если мы любим друг друга, то Бог в~нас пребывает, и~любовь Его совершенна есть в~нас. 
\end{quote}

\noindent Римлянам 5:6-9

\begin{quote}
\textsubscript{6}~Ибо Христос, когда еще мы были немощны, в~определенное время умер за нечестивых. 
\textsubscript{7}~Ибо едва ли кто умрет за праведника; разве за благодетеля, может быть, кто и~решится умереть. 
\textsubscript{8}~Но Бог Свою любовь к~нам доказывает тем, что Христос умер за нас, когда мы были еще грешниками. 
\textsubscript{9}~Посему тем более ныне, будучи оправданы Кровию Его, спасемся Им от гнева. 
\end{quote}

\section*{Цели урока}
\begin{enumerate}
    \item Обсудить проявления Божественной вечной любви
\item Показать неизмеримую и~бесконечную Божью любовь 
\item Побудить христиан искать утешение в~Отцовской любви
\end{enumerate}

\section*{Конспект}

\subsection{1. Различные типы Божьей любви}

\begin{enumerate}

    \item Любовь Бога Отца к~Богу Сыну и~Сына к~Отцу.
    
    Особенно полно эта тема раскрывается в~Евангелии от Иоанна. Там дважды говорится о~том, что Отец любит Сына. Например, евангелист подчеркивает, что мир должен узнать о~том, как Иисус любит Отца:
    \begin{quote}
        <<Но это сбудется, чтобы мир знал, что Я~люблю Отца и, как заповедал Мне Отец, так и~творю. Встаньте, пойдем отсюда>>, Иоанна 14:31.
    \end{quote}
        Эта любовь внутри Троицы выделяет христианство из всех других монотеистических религий
    \begin{quote}
        <<Отец любит Сына и~все дал в~руку Его>>, Иоанна 3:35.
    \end{quote}

    \item Любовь Бога ко всему творению (то, что мы называем <<общей благодатью>>)
    
    Бог создал все существующее и~до того, как в~мире появляется грех, объявляет все сотворенное «хорошим»:
    \begin{quote} 
         <<И увидел Бог все, что Он создал, и~вот, хорошо весьма. И~был вечер, и~было утро – день шестой>> Книга Бытие 1:31
    \end{quote}

    Таково творение любящего Бога. Господь Иисус говорит о~том, что в~этом мире Бог одевает полевую траву славой луговых цветов, не всегда видимых человеку, но видимых Богу. Лев ревет и~выискивает добычу, но его кормит Бог. Птицы небесные находят себе пищу, но это провидение любящего Бога. Ни один воробышек не упадет с~неба без воли Всевышнего: 

    \begin{quote} 
    <<Взгляните на птиц небесных: они не сеют, и~не жнут, и~не собирают в~житницы; и~Отец ваш Небесный питает их. Вы не гораздо ли лучше их? Да и~кто из вас, заботясь, может прибавить себе росту хотя на один локоть? И~об одежде что заботитесь? Посмотрите на полевые лилии, как они растут: не трудятся, не прядут; но говорю вам, что и~Соломон во всей славе своей не одевался так, как всякая из них. Если же траву полевую, которая сегодня есть, а~завтра будет брошена в~печь, Бог так одевает, тем более вас, маловеры! Итак, не заботьтесь и~не говорите: „Что нам есть?<<, или „Что пить?<<, или „Во что одеться?<<, потому что всего этого ищут язычники и~потому что Отец ваш Небесный знает, что вы имеете нужду во всем этом>>, Евангелие от Матфея 6:26--32.
    \end{quote}      
        
    \item Желание Бога спасти падший мир
        
    И~хотя Бог будет судить этот мир, Он также являет себя как Бог, который заповедует и~повелевает всем людям покаяться. Он поручает Своему народу нести Евангелие в~самые отдаленные уголки земли, объявляя его всем людям повсюду. Восставшим против Него Вседержитель говорит: 
    
    \begin{quote}
    «Скажи им: живу Я, говорит Господь Бог: не хочу смерти грешника, но чтобы грешник обратился от пути своего и~жив был. Обратитесь, обратитесь от злых путей ваших; для чего умирать вам, дом Израилев?» Иезекиль 3:11.
    \end{quote} 

    \item Божья особая любовь к~избранным (мы называем ее особой благодатью)

    Избранным может быть весь израильский народ, или церковь как единое целое, или верующие по отдельности. В~каждом случае Бог проявляет особое отношение к~избранным, которое отличается от Его отношения ко всем остальным. Очевидно, что этот аспект Божьей любви в~Библии отличается от тех, которые мы рассматривали ранее. Божья любовь особым образом направлена на избранных (Напр.: Мал. 1:2-3; Еф. 5:25).

\end{enumerate}

\subsection{2. Жертвенная Божья любовь}

\begin{enumerate}
    \item Настоящая любовь~--- это любовь, которая отдает и~не требует и~даже не ожидает ничего взамен.
    
    \emph{Иллюстрация:} Я~люблю свою супругу много за что (красота, дружба, забота и~прочее). Но если это все пропадет, то останется выкристаллизованная любовь, которая есть решение и~которая не зависит от качеств моей супруги. То есть несмотря ни на что, я~буду любить. Мы называем эту любовь~--- \emph{любовь решение}.
    
    \item Жертвенная любовь Бога наиболее ярко проявлена в~том, что Отец отдал Своего единственного Сына, чтобы искупить грешников:
    \begin{quote}
    <<Но Бог Свою любовь к~нам доказывает тем, что Христос умер за нас, когда мы были еще грешниками>> (Римлянам 5:8). 
    \end{quote}
    
    То есть Иисус не увидел что-то красивое или полезное в~нас, что решил взять и~спасти. Он решил, потому что захотел, а~не потому что мы показались ему особенно ценными.
    
    \item Как добрый пастырь, который отдает Свою жизнь за Своих овец, Христос положил Свою жизнь за погибающих <<Я есмь пастырь добрый: пастырь добрый полагает жизнь свою за овец>> (Иоанна 10:11).
\end{enumerate}

\subsection{3. Любовь Бога вопреки условиям}

\begin{enumerate}
    \item Любовь~--- это Его желание. Он избрал Свой народ не по причине наличия в~них чего-то достойного Его любви:
    \begin{quote}
    <<Не потому, чтобы вы были многочисленнее всех народов, принял вас Господь и~избрал вас,~--- ибо вы малочисленнее всех народов>>, Второзаконие 7:7.
    \end{quote}

    \item  Бог~--- инициатор отношений с~людьми:
    \begin{quote}
    <<Будем любить Его, потому что Он прежде возлюбил нас>>, 1~Иоанна 4:19. 
    \end{quote}

\end{enumerate}

\noindent \emph{Важно:} Его любовь к~нам не зависит (не изменяется) от нашего поведения. Она не строится на основании того, что мы сделали или не сделали, сказали или не сказали, подумали или не подумали. Она исключительно основана на Божьем решении и~жертве Иисуса Христа.

\subsection{4. Cердце Бога полно любви к~Своему народу}

\begin{enumerate}
    \item Нужно обязательно пояснить разницу между возмездием и~наказанием. 
    
    \emph{Определение:} Возмездие это кончательная расплата, а~наказание~--- заботливое исправление.

    \emph{Иллюстрация:} Когда мы воспитываем детей, то наказание используется для того, чтобы исправить что-то, но не выместить гнев и~уничтожить.

    \item У~Бога подготовлено заботливое исправление для Своих детей. 
    \begin{itemize}
        \item 
        \begin{quote}
        <<с избытком даст тебе Господь Бог твой успех во всяком деле рук твоих, в~плоде чрева твоего, в~плоде скота твоего, в~плоде земли твоей; ибо снова радоваться будет Господь о~тебе, благодетельствуя тебе, как Он радовался об отцах твоих>> (Втор. 30:9) 
        \end{quote}
        \item 
        \begin{quote}
        <<Ибо от века не слыхали, не внимали ухом, и~никакой глаз не видал другого бога, кроме Тебя, который столько сделал бы для надеющихся на него>> (Исаия 64:4)
        \end{quote}
    \end{itemize}
\end{enumerate}

\noindent \emph{Итого:} Когда мы говорим о~любви Бога, то нужно всегда аккуратно определять про какую именно любовь мы говорим, по отношению к~кому эта любовь проявлена. Например, 
\begin{itemize}
    \item Слова <<Господь любит всех>> не могут быть использованы в~контексте спасения. 
    \item <<Господь любит тебя>> также могут ввести человека в~заблуждение, потому что не предупреждают о~грядущем Божьем гневе на непокаявшегося грешника. Как минимум, так нельзя говорить при благовестии.
\end{itemize}

\vfill
\tiny{Актуальную (с последними правками) версию документа всегда можно найти на сайте \href{https://github.com/kdorichev/faith_basics/blob/main/God/lesson_3.6.pdf}{github.com/kdorichev/faith\_basics}}
\end{document}
