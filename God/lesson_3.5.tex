\documentclass[a4paper,12pt]{article} 
\usepackage[left=2cm,right=2cm,top=2cm,bottom=2cm]{geometry}
%%% Работа с~русским языком
\usepackage{cmap}					% поиск в~PDF
\usepackage[LGR, T2A]{fontenc}			% кодировка
\usepackage[greek, russian]{babel}	% локализация и~переносы
\usepackage{alphabeta}
\usepackage{hyperref}  % для гиперссылок

\hypersetup{
    colorlinks=true,
    linkcolor=black,
    filecolor=magenta,      
    urlcolor=cyan,
}

\title{\textsc{Основы Веры}\\Урок №3.5 Суверенность Бога}
\author{Библейская Церковь Санкт-Петербурга}
\date{}
\setcounter{secnumdepth}{0}  % Unnumbered sections in toc

\begin{document}

\maketitle

\thispagestyle{empty}

% \tableofcontents

% \newcommand{\myline}{\noindent\makebox[\linewidth]{\rule{\linewidth}{0.1pt}}}


\section*{Введение}

Через все Писание громогласно провозглашено, что Бог есть Царь царей и~Господь господствующих. Концепция, наиболее тесно связанная с~Его царством,~--- это Его суверенитет. Сказать, что Бог суверенен, не означает просто сказать, что Он сильнее всех остальных, хотя это также верно.

Называть Его суверенным, значит показать Его власть и~правление, которые превосходят пространство и~время, не оставляя ничего вне сферы Его правления.

В этой части урока мы исследуем природу и~глубину суверенного правления Бога. Стремимся показать, как библейское понимание этой темы может изменить наше понимание истории, происходящих сегодня событий и~даже обстоятельств нашей собственной жизни.

Рекомендуемое время~--- 20 минут.
        
\subsubsection*{Основные тексты}

Псалом 134:6

\begin{quote}
Господь творит все, что хочет, на небесах и~на земле, на морях и~во всех безднах;
\end{quote}

\noindent
Псалом 92:1–5

\begin{quote}
\textsubscript{1}~Господь царствует; Он облечен величием, облечен Господь могуществом и~препоясан: потому вселенная тверда, не подвигнется.
\textsubscript{2}~Престол Твой утвержден искони: Ты~--- от века.
\textsubscript{3}~Возвышают реки, Господи, возвышают реки голос свой, возвышают реки волны свои.
\textsubscript{4}~Но паче шума вод многих, сильных волн морских, силен в~вышних Господь.
\textsubscript{5}~Откровения Твои несомненно верны. Дому Твоему, Господи, принадлежит святость на долгие дни. 
\end{quote}

\noindent
Ефесянам 1:4–5

\begin{quote}
\textsubscript{4}~так как Он избрал нас в~Нем прежде создания мира, чтобы мы были святы и~непорочны пред Ним в~любви,
\textsubscript{5}~предопределив усыновить нас Себе чрез Иисуса Христа, по благоволению воли Своей,
\end{quote}

\section*{Цели урока}
\begin{enumerate}
    \item Объяснить основополагающую важность Божьего суверенитета.
    \item Проиллюстрировать величину суверенитета Бога, который простирается через пространство и~время.
    \item Помочь расти в~страхе Божьем.
    \item Показать, как суверенитет Бога способен обеспечить утешение тем, кто сталкивается с~трудностями и~скорбями. 
\end{enumerate}

\section*{Конспект}

Вопрос: Что такое суверенность? Пусть студенты сформулируют своими словами.

Суверенный происходит от франц. souverain~--- «верховный».

\begin{enumerate}

    \item Суверенность~--- обладание верховной властью. Пример: суверенный правитель. Суверенность Бога выливается в~Его право царствовать, управлять и~руководить абсолютно всем.
    \begin{itemize}
        \item Бог имеет право царствовать по праву Творца и~по причине Своего Превосходства во всем (знание, сила и~т.д.) над всем творением.
        \item Священное Писание неоднократно показывает это. Прочитайте Псалом 134:6 и~Псалом 92:1--5.
    \end{itemize}

    \item Затем покажите \emph{всеохватность} суверенитета Бога во Вселенной через разные сферы Его правления.

    \begin{itemize}
    \item Бог управляет Своим Творением
    
    \begin{enumerate}
        \item Все, что существует, имеет в~качестве причины своего существования Бога.
        \item Раз Бог решил, что что-то будет существовать, то это значит, что Ему оно было нужно, а~это значит, что все принадлежит Богу и~существует для Его славы. 
        \item Растения, существа и~материалы, законы природы находятся под полным Божьим контролем
        \begin{quote}
        <<Ибо Им создано всё, что на небесах и~что на земле, видимое и~невидимое: престолы ли, господства ли, начальства ли, власти ли,~--- все Им и~для Него создано; И~Он есть прежде всего, и~все Им сто\'{и}т>>, Колоссянам 1:16--17.
        \end{quote} 
    \end{enumerate}
    \item Бог управляет историей.
    \begin{enumerate}
        \item Бог направляет человеческие дела 
        \begin{quote}
        <<Бог, сотворивший мир и~всё, что в~нем, Он, будучи Господом неба и~земли, не в~рукотворенных храмах живет и~не требует служения рук человеческих, [как бы] имеющий в~чем-либо нужду, Сам давая всему жизнь и~дыхание и~всё. От одной крови Он произвел весь род человеческий для обитания по всему лицу земли, назначив предопределенные времена и~пределы их обитанию,Дабы они искали Бога, не ощутят ли Его и~не найдут ли — хотя Он и~недалеко от каждого из нас, Ибо мы Им живем и~движемся и~существуем, как и~некоторые из ваших стихотворцев говорили: «мы Его и~род»>>, Книга Деяний 17:24--28.
        \end{quote}

        \item Во всех обстоятельствах Бог работает ради славы Своего имени, и~одной из целей имеет благо для Своего народа. То есть Бог недалеко от нас и~желает для нас исключительного блага.
        \begin{quote}
        <<Притом знаем, что любящим Бога, призванным по Его изволению, все содействует ко благу>>, Римлянам 8:28.
        \end{quote}  
        \item Бог контролирует абсолютно все, что происходит в~наших жизнях. Каждая молекула около нас движется именно так, как хочет Бог.
        \begin{enumerate}
            \item Наше будущее не находится в~руках людей, даже тех, кто обладает мирской властью над нами.
            \item Наша судьба не зависит от воли случая или слепой случайности. Потому что случайность/случай это ничто. Он не материален, и~у него нет и~не может быть воли. Обычно случаем мы описываем процессы, которые не можем до конца понять, но это не значит, что они не имеют под собой строгих причин. 
            
            \textbf{Иллюстрация:} если мы с~вами встретимся случайно в~Новой Зеландии, то это не значит, что у~меня и~у вас не было причин там оказаться, просто не зная все эти причины, мы говорим, что случайно встретились. А~у~Бога нет случайностей, потому что Он совершенно все знает, более того, все определяет.
        \end{enumerate}

        \item Бог властвует и~над Своими врагами
        \begin{enumerate}
            \item Многие в~нашей культуре живут так, как будто дьявол является равноценным противником Бога. Поэтому они считают, что он правит в~своём сатанинском царстве и~боятся последствий его власти и~злобы в~своей жизни. 
            \item Хотя сила сатаны выше нашей, она ничем не сравнима с~могуществом нашего суверенного Господа. Более того, Бог правит и~сатаной, и~его слугами. 
            \item Прочитайте: из 1й главы книги Иова: 
            \begin{quote}
            <<Был человек в~земле Уц, имя его Иов; и~был человек этот непорочен, справедлив и~богобоязнен и~удалялся от зла. И~родились у~него семеро сыновей и~три дочери. Имения у~него было: семь тысяч мелкого скота, три тысячи верблюдов, пятьсот пар волов и~пятьсот ослиц и~весьма много прислуги; и~был человек этот знаменитее всех сынов Востока. Сыновья его сходились, делая пиры каждый в~своем доме в~свой день, и~посылали и~приглашали трех сестер своих есть и~пить с~ними. Когда круг пиршественных дней завершался, Иов посылал за ними и~освящал их, и, вставая рано утром, возносил всесожжения по числу всех их. Ибо говорил Иов: «Может быть, сыновья мои согрешили и~похулили Бога в~сердце своем». Так делал Иов во все такие дни. И~был день, когда пришли сыны Божии предстать пред Господом; между ними пришел и~сатана. И~сказал Господь сатане: «Откуда ты пришел?» И~отвечал сатана Господу и~сказал: «Я ходил по земле и~обошел ее». И~сказал Господь сатане: «Обратил ли ты внимание твое на раба Моего Иова? Ибо нет такого, как он, на земле: человек непорочный, справедливый, богобоязненный и~удаляющийся от зла».И отвечал сатана Господу и~сказал: «Разве даром богобоязнен Иов? Не Ты ли кругом оградил его, и~дом его, и~все, что у~него? Дело рук его Ты благословил, и~стада его распространяются по земле; но простри руку Твою и~коснись всего, что у~него, – благословит ли он Тебя?» И~сказал Господь сатане: «Вот, все, что у~него, в~руке твоей; только на него не простирай руки твоей». И~отошел сатана от лица Господнего>>, Иов 1:1--12.
            \end{quote}
        \end{enumerate}
        \item Итого: Божье правление не ограничивается Церковью, но распространяется на всех людей и~весь мир! Абсолютно все находится под Божьим контролем.
        \item \emph{Для группы переходящих только:} Бог руководит спасением.
        \begin{enumerate}
            \item Бог предопределил Свой народ для спасения на основании Его вечных целей, не просто предвидя человеческие действия и~решения 
            \begin{quote}
            <<Благословен Бог и~Отец Господа нашего Иисуса Христа, благословивший нас во Христе всяким духовным благословением в~небесах, так как Он избрал нас в~Нем прежде создания мира, чтобы мы были святы и~непорочны пред Ним в~любви, предопределив усыновить нас Себе чрез Иисуса Христа, по благоволению воли Своей, в~похвалу славы благодати Своей, которою Он облагодатствовал нас в~Возлюбленном, в~Котором мы имеем искупление Кровию Его, прощение грехов, по богатству благодати Его>> (Еф.1:3--7).
            \end{quote}
            \item Выбор Бога Его избранных был сделан только Им и~к Его славе.
            \begin{itemize}
                \item Поскольку темы предопределения мы будем рассматривать далее, то здесь нужно обозначить суть проблемы~--- либо Бог суверенный, либо Бог не суверенный. 
                \item Проще всего это будет обозначить следующим образом: Задайте студентам вопрос <<Если Бог в~своем суверенном праве и~добром любящем расположении к~грешнику решил спасти этого конкретного человека, создал все условия~--- послал проповедника, убедился, что Евангелие верно донесено, открыл сердце слушающего (то есть Бог сделал все необходимое для спасения), то может ли человек не спастись?>> 
                \item Обязательно возникнет напряжение о~<<свободе воли>>, но не нужно сейчас отвечать на этот вопрос. Ваша задача~--- обозначить, что напряжение есть и~оно будет снято в~дальнейших уроках. Студенты уйдут домой с~хорошей <<пищей для размышления>>.
            \end{itemize}
        \end{enumerate}
    \end{enumerate}
    \end{itemize}
\end{enumerate}

\vfill
\tiny{Актуальную (с последними правками) версию документа всегда можно найти на сайте \href{https://github.com/kdorichev/faith_basics/blob/main/God/lesson_3.5.pdf}{github.com/kdorichev/faith\_basics}}
\end{document}я
