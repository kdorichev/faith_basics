\documentclass[a4paper,12pt]{article} 
\usepackage[left=2cm,right=2cm,top=2cm,bottom=2cm]{geometry}
%%% Работа с~русским языком
\usepackage{cmap}					% поиск в~PDF
\usepackage[LGR, T2A]{fontenc}			% кодировка
\usepackage[greek, russian]{babel}	% локализация и~переносы
\usepackage{alphabeta}
\usepackage{hyperref}  % для гиперссылок

\hypersetup{
    colorlinks=true,
    linkcolor=black,
    filecolor=magenta,      
    urlcolor=cyan,
}

\title{\textsc{Основы Веры}\\Урок №3.3 Святость Бога}
\author{Библейская Церковь Санкт-Петербурга}
\date{}
\setcounter{secnumdepth}{0}  % Unnumbered sections in toc

\begin{document}

\maketitle

\thispagestyle{empty}

\tableofcontents

% \newcommand{\myline}{\noindent\makebox[\linewidth]{\rule{\linewidth}{0.1pt}}}


\section{Введение}

Для многих людей сегодня святость~--- это чуждое для них понятие. Тем не менее для авторов Писания святость является одним из самых выдающихся качеств Бога. Она указывает как на Его абсолютную нравственную чистоту, так и~ на отделенность от какого-либо греха, да и~ от всего творения. В этой части урока мы посмотрим, что значит святость Бога, и~ как понимание этого влияет на наше приближение к~ живому Богу и~ жизнь перед Ним в~ смиренном почтении.

Рекомендуемое время~--- 20 минут.
        
\subsubsection*{Основные тексты}

Откровение Иоанна Богослова 1:12--18

\begin{quote}
\textsubscript{12}~Я обратился, чтобы увидеть, чей голос, говоривший со мною; и~ обратившись, увидел семь золотых светильников 
\textsubscript{13}~и, посреди семи светильников, подобного Сыну Человеческому, облеченного в~ подир и~ по персям опоясанного золотым поясом: 
\textsubscript{14}~глава Его и~ волосы белы, как белая волна, как снег; и~ очи Его, как пламень огненный; 
\textsubscript{15}~и ноги Его подобны халколивану, как раскаленные в~ печи, и~ голос Его, как шум вод многих. 
\textsubscript{16}~Он держал в~ деснице Своей семь звезд, и~ из уст Его выходил острый с~ обеих сторон меч; и~ лице Его, как солнце, сияющее в~ силе своей. 
\textsubscript{17}~И когда я~ увидел Его, то пал к~ ногам Его, как мертвый. И Он положил на меня десницу Свою и~ сказал мне: не бойся; Я есмь Первый и~ Последний, 
\textsubscript{18}~и живый; и~ был мертв, и~ се, жив во веки веков, аминь; и~ имею ключи ада и~ смерти. 

\end{quote}

Исайя 6:1--5

\begin{quote}
\textsubscript{1}~В год смерти царя Озии видел я~ Господа, сидящего на престоле высоком и~ превознесенном, и~ края риз Его наполняли весь храм. 
\textsubscript{2}~Вокруг Него стояли Серафимы; у~ каждого из них по шести крыл: двумя закрывал каждый лице свое, и~ двумя закрывал ноги свои, и~ двумя летал. 
\textsubscript{3}~И взывали они друг ко другу и~ говорили: Свят, Свят, Свят Господь Саваоф! вся земля полна славы Его! 
\textsubscript{4}~И поколебались верхи врат от гласа восклицающих, и~ дом наполнился курениями. 
\textsubscript{5}~И сказал я: горе мне! погиб я! ибо я~ человек с~ нечистыми устами, и~ живу среди народа также с~ нечистыми устами,~--- и~ глаза мои видели Царя, Господа Саваофа. 
\end{quote}

\section{Цели урока}
\begin{enumerate}
    \item Объяснить, что Писание говорит о~ святости Бога
    \item  Помочь возрасти в~ страхе Господнем!
    \item  Побудить к~ богоугодному поклонению, которое Бог ожидает и~ принимает.
\end{enumerate}

\section{Конспект}

\subsection{1. Исключительность Божьей святости}

Задайте вопрос: Что такое святость? Пусть студенты попробуют своими словами сформулировать. Обращайте внимание, когда они пытаются Человеческую и~ Божью святость смешать.

\begin{enumerate}
    \item Святость Бога особым образом показана в~ Писании.
    \begin{itemize}
        \item Прочитайте: первые 5 стихов из 6 главы Книги пророка Исайи.
        \item Библия неоднократно описывает людей, места и~ вещи, связанные с~ Богом, как «святые».
        \item Небесные существа, которые окружают Божий престол, постоянно восклицают: «Свят, свят, свят!». Описывая Бога таким образом, ангелы заявляют, что Бог абсолютно свят. 
        \item Из многих качеств, используемых для описания Бога, святость~--- одна из самых заметных. Хотя Писание часто говорит о~ Божьей любви, истине и~ суверенитете, все остальные качества не представлены так ярко и~ так часто как святость.
    \end{itemize}
\end{enumerate}
        
\subsection{2. Высота Божьего статуса}        
            
\begin{enumerate}    

    \item Сказать, что Бог свят, значит подтвердить, что Он отделен от Своего творения.
    \begin{itemize}
        \item Еврейское слово, переведенное как «свят» (кадош), в~ первую очередь означает разделение или акт по отделению.
        \item Таким образом, Бог совершенно и~ бесконечно превосходит Своё творение во всех смыслах.
    \end{itemize}
    \item Сказать, что Бог свят, значит приписать Ему царственное величие
    \begin{itemize}
        \item <<Кто, как Ты, Господи, между богами? Кто, как Ты, величествен святостью, досточтим хвалами, Творец чудес?>> (Исход 15:11)
        \item <<Кто взойдет на гору Господню, или кто станет на святом месте Его?>> (Псалтырь 23:3).
        \item Божья слава (откровение Божьего величия) слишком ошеломляет людей, чтобы ее равнодушно созерцать или постигать. Его царственное величие многократно превосходит великолепие самых возвышенных человеческих правителей и~героев.
    \end{itemize}
    \item Божественная святость точно описана в~ Исаии 6:1-7.
    \begin{itemize}
        \item края царской одежды Бога настолько широки, что наполняют весь храм. <<и~края риз Его наполняли весь храм>> Исайя 6:1
        \item даже ослепительный серафим, пылающий своей силой для Божьей славы, должен покрывать себя в~ присутствии Божьего великолепия. <<Вокруг Него стояли Серафимы; у~ каждого из них по шести крыл: двумя закрывал каждый лице свое, и~ двумя закрывал ноги свои, и~ двумя летал>> Исайя 6:2
        \item ошеломленный этим коротким взглядом на Божью святость, Исайя глубоко осознал свою нечистоту. <<И сказал я: горе мне! погиб я! ибо я~ человек с~ нечистыми устами, и~ живу среди народа также с~ нечистыми устами,~--- и~ глаза мои видели Царя, Господа Саваофа>> Исайя 6:5
    \end{itemize}
    \item Человек, понимающий святость Творца и~ живущий в~ Его присутствии будет иметь крайне скромное мнение о~ себе и~ о~ людях в~ целом. Более того, Церковь не может быть духовно сильной и~ духовно-здоровой, если не сознает и~ не почитает Святость Бога. Прочитайте: естественную реакцию апостола Иоанна на Божье присутствие в~ 1 главе Книги Откровения.
\end{enumerate}

\subsection{3. Моральное совершенство Бога}

\begin{enumerate}
    \item Бог не имеет греха и~ порока, да и~ не может иметь по определению. Именно таким Он открывается нам на страницах Писания.
    \item Все решения, пути и~ суды Бога морально совершенны.
    \item Будучи бесконечно святым, Бог не является нейтральным к~ добру или злу.
    \begin{itemize}
        \item Бог радуется всему, что истинно, достойно и~ честно.
        \item Поскольку сама природа Бога является совершенно и~ морально чистой, Он не может быть терпимым к~ чему-то нечистому, несовершенному, или даже направленному против Него. 
        
        \emph{Важно:} Грех является прямым выступлением против святости Бога. И у~ Него всегда на грех конкретная реакция~--- Он гневается! Всякий грех противен Его сущности!
    \end{itemize}
\end{enumerate}

\end{document}
