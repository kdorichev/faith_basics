\documentclass[a4paper,12pt]{article} 
\usepackage[left=2cm,right=2cm,top=2cm,bottom=2cm]{geometry}
%%% Работа с~русским языком
\usepackage{cmap}					% поиск в~PDF
\usepackage[LGR, T2A]{fontenc}			% кодировка
\usepackage[greek, russian]{babel}	% локализация и~переносы
\usepackage{alphabeta}
\usepackage{hyperref}  % для гиперссылок

\hypersetup{
    colorlinks=true,
    linkcolor=black,
    filecolor=magenta,      
    urlcolor=cyan,
}

\title{\textsc{Основы Веры}\\Урок №3.2 Вечное и~независимое бытие Бога (трансцендентность)\footnote{от лат. transcendens «переступающий, превосходящий, выходящий за пределы»~--- то, что принципиально недоступно опытному познанию, выходит за пределы чувственного опыта.} }
\author{Библейская Церковь Санкт-Петербурга}
\date{}
\setcounter{secnumdepth}{0}  % Unnumbered sections in toc

\begin{document}

\maketitle

\thispagestyle{empty}

\tableofcontents

% \newcommand{\myline}{\noindent\makebox[\linewidth]{\rule{\linewidth}{0.1pt}}}


\section{Введение}

Из множества различных качеств Бога одним из самых трудных для понимания является Его независимое существование.

В отличие от нас, у~Бога нет ни начала, ни конца, и~Он не зависит ни от кого и~ни от чего, но все зависит от Него.

В этой части мы рассматриваем безграничную Божию достаточность и~нашу собственную глубокую зависимость от Творца, которая поможет найти нам полное удовлетворение и~восхищение в~Том, кто является причиной всех вещей.

Рекомендуемое время~--- 20 минут
        
\subsubsection*{Основные тексты}

Бытие 1:1

\begin{quote}
    <<В начале сотворил Бог небо и~землю>>.
\end{quote}

Книга Деяний 17:17--25

\begin{quote}
17 Итак он рассуждал в~синагоге с~Иудеями и~с чтущими Бога, и~ежедневно на площади со встречающимися. 
18 Некоторые из эпикурейских и~стоических философов стали спорить с~ним; и~одни говорили: <<что хочет сказать этот суеслов?>>, а~другие: <<кажется, он проповедует о~чужих божествах>>, потому что он благовествовал им Иисуса и~воскресение. 
19 И, взяв его, привели в~ареопаг и~говорили: можем ли мы знать, что это за новое учение, проповедуемое тобою? 
20 Ибо что-то странное ты влагаешь в~уши наши. Посему хотим знать, что это такое? 
21 Афиняне же все и~живущие у~них иностранцы ни в~чем охотнее не проводили время, как в~том, чтобы говорить или слушать что~--- нибудь новое. 
22 И, став Павел среди ареопага, сказал: Афиняне! по всему вижу я, что вы как бы особенно набожны. 
23 Ибо, проходя и~осматривая ваши святыни, я~нашел и~жертвенник, на котором написано <<неведомому Богу>>. Сего-то, Которого вы, не зная, чтите, я~проповедую вам. 
24 Бог, сотворивший мир и~все, что в~нем, Он, будучи Господом неба и~земли, не в~рукотворенных храмах живет 
25 и~не требует служения рук человеческих, как бы имеющий в~чем-либо нужду, Сам дая всему жизнь и~дыхание и~все.
\end{quote}

Книга пророка Исайи 55:8--9
\begin{quote}
Мои мысли~--- не ваши мысли, ни ваши пути~--- пути Мои, говорит Господь.
Но как небо выше земли, так пути Мои выше путей ваших, и~мысли Мои выше мыслей ваших.
\end{quote}

\section{Цели урока}
\begin{enumerate}
    \item Обсудить смысл вечной и~независимой (в том числе и~от нас) природы бытия Бога
    \item Признать неспособность человека постичь беспредельный характер Бога
    \item Призвать христиан к~поиску единственного упования на Бога по причине Его вседостаточности
\end{enumerate}

\section{Конспект}

\subsection{1. Бесконечное величие Бога}

\begin{enumerate}
    \item Вечное и~независимое бытие (трансцендентность) – самое отличительное качество Бога. Бог обладает другой природой, не физической. Мы не имеем никакого инструмента, чтобы в~этой духовной области что-то измерить. То есть Бог это Бытие наивысшего порядка, Которое ограниченные существа вроде нас с~вами постичь полностью не могут. 
    
    \begin{itemize}
        \item Вечное и~независимое Божественное бытие указывает на вечное самосуществование Бога. Если по простому, то Бог имеет источник необходимой для существования энергии в~себе самом.
        \item У Бога нет ни момента, ни места рождения. Нам трудно мыслить такими категориями, потому что у~физического мира есть точка начала. Нам трудно представить, что есть мир за пределами физического, где времени нет. Также у~Бога нет и~точки останова.
        \item Бог~--- единый источник бытия. Мы называем Бога <<первопричиной>> всего. Греческие философы называли эту первопричину <<Логос>>. Пример: Мы видим в~мире движение тел~--- Земля вертится вокруг Солнца, Луна вертится вокруг Земли. Мы знаем, что по первому закону Ньютона тело сохраняет состояние покоя или прямолинейного равноускоренного движения, если на него не действуют другие тела. Но раз в~мире есть движение, то Кто-то должен был запустить <<маховик>> Вселенной. Этого Кого-то мы называем Богом и~первопричиной.
        \item Бог существует, а~точнее сказать <<есть>> всегда. Сложновато, но факт.
        \begin{enumerate}
            \item Бог уже был до сотворения мира <<В начале сотворил Бог небо и~землю>> (Бытие 1:1)
            \item Божье правление над творением вечно <<Царство Твое~--- царство всех веков, и~владычество Твое во все роды>> (Псалом 144:13)
        \end{enumerate}
    \end{itemize}
\end{enumerate}
        
\subsection{2. Источник всего}        
            
\begin{enumerate}    

    \item Конечная вселенная обязана своим существованием бесконечному Богу. То есть Бог является источником необходимой для поддержания вселенной энергии. Если убрать Бога из этого уравнения, то Вселенная прекратит свое существование.
    \item До создания времени и~пространства триединый Бог существовал вечно без творения. Существовав без творения Он не был одинок. Три личности Троицы пребывали в~совершенном общении.
    \item Божий акт творения говорит нам о~Его бесконечном могуществе и~величии
    \begin{itemize}
        \item Бог не испытывал недостатка в~отсутствие творения или в~чем-либо ином
        \item Бог не создал вселенную по необходимости, находясь под каким-либо давлением или обязательством принести что-то новое в~бытие.
        \item Он просто захотел и~сделал. Причина этого желания нам, увы, неизвестна.
    \end{itemize}

    \item Бог является источником всей жизни.
    
    \begin{itemize}
        \item Поскольку Он всецело существует Сам в~Себе, Он не порожден и~не зависим от чего-либо (кого-либо) вне Себя <<Ибо, как Отец имеет жизнь в~Самом Себе, так и~Сыну дал иметь жизнь в~Самом Себе>>. (Ин. 5:26)
        \item Будучи Его творениями, наша физическая, духовная и~даже вечная жизнь с~Ним имеют Бога своим источником ибо мы Им живем и~движемся и~существуем, как и~некоторые из ваших стихотворцев говорили: <<мы Его и~род>>. (Деян. 17:28).
    \end{itemize}

    \item Библия говорит, что все «из Него, через Него (Им) и~к Нему» (Рим. 11:36).
    
    \begin{itemize}
        \item то, что все «из» Него, отражает, что Он~--- источник всех вещей.
        \item то, что все «через» Него отражает, что Он является видимым через все сотворенные вещи.
        \item то, что все «к» Нему, отражает, что Он является целью всех вещей. То есть все в~этом мире существует ради Его славы!
    \end{itemize}
\end{enumerate}

\subsection{3. Дети заботливого совершенного Отца (имманентность)}

Имманентность\footnote{лат. immanens, «пребывающий внутри»}~--- учение о проявлении божественного в материальном мире.

\begin{enumerate}
    \item Неверно думать, что Бог далек от нас, что Он настолько другой, что ничего общего у~него с~человеком нет. Бог~--- наш источник жизни. Мы созданы для того, чтобы отразить (прославить) Его святой характер.
    \begin{itemize}
        \item <<Не знает покоя сердце наше, пока не успокоится в~тебе>>~--- Блаженный Августин 
        \item Чем больше наше понимание того, кто есть Бог, тем больше наши сердца наполняются покоем, который мы находим только в~Нем.
    \end{itemize}
    \item Хотя мы ничтожны по сравнению с~Богом, Он призывает нас принести все наши заботы к~Нему и~доверять Ему во всем.
    \begin{itemize}
        \item Обсудите со студентами: мы часто задаемся вопросом, заботится ли Он о~мелочах в~нашей жизни, или же Он только озабочен глобальными вопросами.
        \item Однако, по сравнению с~Его величием и~самодостаточностью, все в~наших жизни мало; тем не менее, Он все равно призывает нас прийти к~Нему.
    \end{itemize}
    \item Поскольку Бог является вседостаточным, мы можем знать, что в~нашей жизни нет нужды, которую Он не мог бы разрешить.
\end{enumerate}

\subsection{4. Вездесущность и~всеведение Бога}

Помимо того, что Бог имеет вечное, независимое бытие и~является источником всего, наш непревзойденный Господь обладает такими качествами как вездесущность и~всеведение. 

\begin{enumerate}
    \item Вездесущность 
    
Определение: Бог присутствует всем Своим существом в~каждой точке пространства. Бог не имеет размера или пространственных измерений и~присутствует в~каждой точке пространства всем Своим существом, и~все же Бог действует по-разному в~разных местах.

    \item Всеведение 

Определение: Бог бесконечен в~знании. Он знает Себя и~всё остальное в~совершенстве всю вечность, будь то действительность или лишь возможность, будь то прошлое, настоящее или будущее.

    \item Текст: Псалом 138 хорошо показывает оба этих качества Бога. 
\end{enumerate}

\end{document}
