\documentclass[a4paper,12pt]{article} 
\usepackage[left=2cm,right=2cm,top=2cm,bottom=2cm]{geometry}
%%% Работа с~русским языком
\usepackage{cmap}					% поиск в~PDF
\usepackage[LGR, T2A]{fontenc}			% кодировка
\usepackage[greek, russian]{babel}	% локализация и~переносы
\usepackage{alphabeta}
\usepackage{hyperref}  % для гиперссылок

\hypersetup{
    colorlinks=true,
    linkcolor=black,
    filecolor=magenta,      
    urlcolor=cyan,
}

\title{\textsc{Основы Веры}\\Урок №3.4 Божий гнев}
\author{Библейская Церковь Санкт-Петербурга}
\date{}
\setcounter{secnumdepth}{0}  % Unnumbered sections in toc

\begin{document}

\maketitle

\thispagestyle{empty}

\tableofcontents

% \newcommand{\myline}{\noindent\makebox[\linewidth]{\rule{\linewidth}{0.1pt}}}


\section*{Введение}

Многие люди не знают и~даже не хотят знать о~Божьем гневе. Библия же об этом говорят прямо и~определенно. В Писании Божий гнев служит предупреждением для неверующих людей и~одной из причин святой жизни Его народа. В этой части урока мы посмотрим на природу и~проявление Божьего гнева

Рекомендуемое время~--- 20 минут.
        
\subsubsection*{Основные тексты}

Рим 1:21--25, 28--32

\begin{quote}
\textsubscript{21}~Но как они, познав Бога, не прославили Его, как Бога, и~не возблагодарили, но осуетились в~умствованиях своих, и~омрачилось несмысленное их сердце; 
\textsubscript{22}~называя себя мудрыми, обезумели, 
\textsubscript{23}~и славу нетленного Бога изменили в~образ, подобный тленному человеку, и~птицам, и~четвероногим, и~пресмыкающимся,~--- 
\textsubscript{24}~то и~предал их Бог в~похотях сердец их нечистоте, так что они сквернили сами свои тела. 
\textsubscript{25}~Они заменили истину Божию ложью, и~поклонялись, и~служили твари вместо Творца, Который благословен во веки, аминь. 

\ldots

\textsubscript{28}~И как они не заботились иметь Бога в~разуме, то предал их Бог превратному уму~--- делать непотребства, 
\textsubscript{29}~так что они исполнены всякой неправды, блуда, лукавства, корыстолюбия, злобы, исполнены зависти, убийства, распрей, обмана, злонравия, 
\textsubscript{30}~злоречивы, клеветники, богоненавистники, обидчики, самохвалы, горды, изобретательны на зло, непослушны родителям, 
\textsubscript{31}~безрассудны, вероломны, нелюбовны, непримиримы, немилостивы. 
\textsubscript{32}~Они знают праведный суд Божий, что делающие такие дела достойны смерти; однако не только их делают, но и~делающих одобряют. 
\end{quote}

\noindent Исайя 63:1--6

\begin{quote}
\textsubscript{1}~Кто это идет от Едома, в~червленых ризах от Восора, столь величественный в~Своей одежде, выступающий в~полноте силы Своей? <<Я~--- изрекающий правду, сильный, чтобы спасать>>. 
\textsubscript{2}~Отчего же одеяние Твое красно, и~ризы у~Тебя, как у~топтавшего в~точиле? 
\textsubscript{3}~<<Я топтал точило один, и~из народов никого не было со Мною; и~Я топтал их во гневе Моем и~попирал их в~ярости Моей; кровь их брызгала на ризы Мои, и~Я запятнал все одеяние Свое; 
\textsubscript{4}~ибо день мщения~--- в~сердце Моем, и~год Моих искупленных настал. 
\textsubscript{5}~Я смотрел, и~не было помощника; дивился, что не было поддерживающего; но помогла Мне мышца Моя, и~ярость Моя~--- она поддержала Меня: 
\textsubscript{6}~и попрал Я~народы во гневе Моем, и~сокрушил их в~ярости Моей, и~вылил на землю кровь их>>. 
\end{quote}

\section*{Цели урока}
\begin{enumerate}
    \item Объяснить неизбежность и~различные проявления Божьего гнева.
    \item Показать, что Божий гнев~--- это реакция Его святости на грех.
    \item Возбудить святой страх к~Богу, Который будет судить мир.
    \item Возблагодарить Бога, показав, от чего спасены верующие, и~что претерпел Христос.
\end{enumerate}

\section*{Конспект}

\subsection{1. Неизбежность Божьего гнева}

Вопрос: Что вы слышали до этого о~Божьем гневе? Что такое гнев? Попросите студентов сформулировать своими словами.

\begin{enumerate}
    \item Определение: «Гнев»~--- это древнее слово, определяемое в~толковых словарях как «глубокое, интенсивное чувство возмущения и~негодования». «Негодование», в~свою очередь,~--- это «проявление недовольства, раздражения и~чувства вражды из-за нанесенной обиды или оскорбления». «Возмущение» определяется как «справедливое негодование, появляющееся в~результате несправедливости и~подлости». Вот что такое гнев. И, как сообщает нам Библия, гнев~--- это одна из присущих Богу черт.
    \item Гнев Божий~--- это проявление Его святости. Бог не может быть нейтральным по отношению ко греху, потому что Святой и~не терпит никакого греха во всей Сотворенной Им Вселенной
    \item Чтобы увидеть какова в~действительности реакция Бога на грех, прочитайте со студентами первые шесть стихов 63й главы книги пророка Исайи. И обсудите обязательно, что думают студенты об этом.
\end{enumerate}
        
\subsection{2. Природа Божьего гнева}        
            
\begin{enumerate}    

    \item Божий гнев в~действии
    \begin{itemize}
        \item Божественный гнев уже пребывает над грешниками  <<Ибо открывается гнев Божий с~неба на всякое нечестие и~неправду человеков, подавляющих истину неправдою>> (Римлянам 1:18). Божий гнев выражается в~том, что грешник, неоднократно отвергающий знание о~Боге и~Евангелие Иисуса Христа, оставляется и~предается Богом для его обольщения собственными грехами  <<то и~предал их Бог в~похотях сердец их нечистоте, так что они осквернили сами свои тела. Они заменили истину Божию ложью, и~поклонялись, и~служили твари вместо Творца, Который благословен во веки, аминь. Потому предал их Бог постыдным страстям: женщины их заменили естественное употребление противоестественным>> Римлянам 1:24-26 Если человек не раскается сейчас, то он испытает всю полноту Божьего сокрушительного справедливого гнева. 
        \item В настоящее время Его милость и~долготерпение удерживают Бога от того, чтобы излить весь Свой гнев на нас, потому призовите студентов к~покаянию.
    \end{itemize}
    \item Божий гнев катастрофический
	\begin{itemize}
        \item Для того, чтобы развеять сказку о~<<добром дедушке на небесах>>, вспомните, что в~человеческой истории были времена, когда Бог вершил страшное правосудие над родом человеческим:
        \begin{enumerate}
                \item Потоп
                \item Огонь и~сера с~небес на Содом и~Гоморру. 
                \item Казни Египетские
        \end{enumerate}
        \item Это и~есть выражения святого Божьего гнева против грешного человечества.
    \end{itemize}
    \item Неотвратимость Божьего воздаяния 
    \begin{itemize}
        \item Божий гнев раскрывается в~принципе посева и~жатвы:
        \begin{quote}
            <<Не обманывайтесь: Бог поругаем не бывает. Что посеет человек, то и~пожнет>>, Галатам 6:7;
        \end{quote}
        \item Любой сделанный человеком грех неизбежно столкнется с~Божьим карающим гневом.
    \end{itemize}
    \item Грядущий Божий гнев 
    \begin{itemize}
        \item Однажды Божий гнев будет явлен в~завершении человеческой истории, во время Второго Пришествия Христа:
        \begin{quote}
        <<Из уст же Его исходит острый меч, чтобы им поражать народы. Он пасет их жезлом железным; Он топчет точило вина ярости и~гнева Бога Вседержителя>>, Откр. 19:15. 
        \end{quote}  
    \end{itemize}
    \item Божий гнев вечный 
    \begin{itemize}
        \item Божий гнев пребывает в~аду:
        \begin{quote}
        <<И смерть и~ад повержены в~озеро огненное. Это смерть вторая>>,\\ Откр. 20:14.
        \end{quote}
    \end{itemize}
\end{enumerate}

\noindent Итого: души либо будут ввержены в~вечные адские муки, либо помилованы Тем, кто может помиловать. Других вариантов нет.

\vfill
\tiny{Актуальную (с последними правками) версию документа всегда можно найти на сайте \href{https://github.com/kdorichev/faith_basics/blob/main/God/lesson_3.4.pdf}{github.com/kdorichev/faith\_basics}}

\end{document}
