\documentclass[a4paper,12pt]{article} 
\usepackage[left=2cm,right=2cm,top=2cm,bottom=2cm]{geometry}
%%% Работа с~русским языком
\usepackage{cmap}					% поиск в~PDF
\usepackage[LGR, T2A]{fontenc}			% кодировка
\usepackage[greek, russian]{babel}	% локализация и~переносы
\usepackage{alphabeta}
\usepackage{hyperref}  % для гиперссылок

\hypersetup{
    colorlinks=true,
    linkcolor=black,
    filecolor=magenta,      
    urlcolor=cyan,
}

\title{\textsc{Основы Веры}\\Урок №3.1 Введение в~Троицу}
\author{Библейская Церковь Санкт-Петербурга}
\date{}
\setcounter{secnumdepth}{0}  % Unnumbered sections in toc

\begin{document}

\maketitle

\thispagestyle{empty}

% \begin{abstract}
%     Данное пособие подготовлено в~помощь желающим познакомиться с~основами христианской веры благодаря самостоятельному изучению Евангелия от Иоанна. Тексты Евангелия взяты по Синодальному переводу. Добавлена буква~ё.
    
%     Структура и~вопросы заимствованы из пособия Русской Библейской Церкви (г. Москва). Вёрстка выполнена в~Библейской Церкви г.\ Санкт-Петербурга. 
% \end{abstract}

\tableofcontents

\newcommand{\myline}{\noindent\makebox[\linewidth]{\rule{\linewidth}{0.1pt}}}


\section{Введение}
Прежде чем говорить о~Боге, нужно понять зачем это делать и~как это делать. Попробуем немного разобраться с~такими терминами как Бог, Троица, качества Бога, чтобы наша дальнейшая беседа была осмысленной.

Рекомендуемое время~--- 20 минут
        
Основные тексты

Книга Второзаконие 6:4 
<<Слушай, Израиль: Господь, Бог наш, Господь един есть\ldots>>
    

\section{Цели урока}
\begin{enumerate}
    \item Обозначить Троицу и~Ее участие в~замысле искупления.
    \item Дать определения и~примеры Божественных атрибутов.
    \item Обсудить, как триединство Бога описывается Божественными атрибутами.
    \item Показать, как знание о~Боге воздействует на жизнь и~поклонение.
\end{enumerate}

\section{Конспект}

\subsection{1. Некоторые неверные сложившиеся взгляды на Бога}
\begin{enumerate}
    \item Агностики\footnote{от др.-греч. ἄγνωστος «непознаваемый»} (не может знать, есть ли Бог). 
    
    Вопрос: Спросите у~студентов, что они слышали о~них.
    \begin{itemize}
        \item Стоят на мнении, что невозможно познать истину в~вопросах существования Бога или вечной жизни, с~которыми связано христианство и~прочие религии. Или, если это и~не невозможно вообще, то, по крайней мере, не представляется возможным в~настоящее время;
        \item Воздерживаются от суждения, говоря, что нет достаточных оснований ни для подтверждения, ни для отрицания Бога. в~то же время, некоторые агностики могут считать, что существование Бога, хотя и~не невозможно, но вряд ли вероятно. Тем самым они устремляются в~сторону атеизма.
    \end{itemize}

    \item Атеисты\footnote{др.-греч. ἄθεος — «отрицание бога, безбожие»; от ἀ — «без» + θεός — «Бог»} (не верит в~Бога и~знает, что Его нет).
    
    Вопрос: Спросите у~студентов, что они слышали о~них или есть ли среди в~группе такие.

    \begin{itemize}
        \item Отрицают различные религиозные представления, культы и~утверждения самоценности бытия мира и~человека;
        \item Считают любую религию иллюзией сознания и~творением самого человека;
        \item Не просто отрицает существование Бога, но и~представляет собой мировоззрение, включающее в~себя научные, моральные и~социальные основания для отрицания Его существования и~философию жизни без Бога.
        \item Признают естественный, окружающий человека мир единственным и~самодостаточным;
        \item Основываются на естественнонаучном постижении мира, противопоставляя полученное таким путём знание вере
        \item Основываются на принципах светского гуманизма (признание ценности человека как личности, его права на свободное развитие и~проявление своих способностей, утверждение блага человека как критерий оценки общественных отношений)
        \item Утверждают первостепенное значение человека, человеческой личности и~человеческого существа по отношению к~любой социальной или религиозной структуре.
        \item в~целом учение атеизма непоследовательное и~легко разбиваемое логическими аргументами.
    \end{itemize}

    \item Деисты (Бог сотворил мир, и~отошел в~сторону. Поэтому не управляет творением, хотя и~дал ему Свои законы). 
    
    Вопрос: Спросите у~студентов, что они слышали о~них.
    \begin{itemize}
        \item Признают существование Бога и~сотворение Им мира
        \item Отрицают большинство сверхъестественных и~мистических явлений, божественное откровение и~религиозный догматизм
        \item Полагают, что Бог после сотворения мира не вмешивается в~течение событий (как некий великий часовщик, который сделал часы и~больше не вмешивается в~их ход) 
        \item Предполагают, что разум, логика и~наблюдение за природой~--- единственные средства для познания Бога и~Его воли. 
        \item Высоко ценят человеческий разум и~свободу. 
        \item Стремятся привести к~гармонии науку и~идею о~существовании Бога, а~не противопоставлять науку и~религию.
        \item Деизм был особенно популярен два-три века назад
    \end{itemize}

\end{enumerate}

        
\subsection{2. Понимание Троицы}        
            
\begin{enumerate}    

    \item Нужно дать определение термину Бог. Обычно это вызывает серьезные сложности, потому попробуйте со студентами вывести некое определение, которое в~конечном итоге должно свестись к~формулировке Ансельма Кентеберийского <<Бог~--- это то, больше чего ничего нельзя помыслить>>.
    То есть абсолют во всех отношениях: если любит~--- то любит абсолютно, если справедлив, то судит абсолютно справедливо, если суверенен, то суверенен абсолютно и~т.д.
    \item Поговорите со студентами о~том, как определить Троицу~--- пусть они дадут свои варианты. В~финале обсуждения нужно пояснить, что исторически церковь системно занималась этим вопросом. И вывела формулу: <<Един по сущности, Трое в~личностях>>.
    \item Бог един в~Своей сущности, но имеет три Личности
    \begin{itemize}
        \item Различие в~Личностях Бога не разделяет единство Божьей сущности
        \item Новый Завет подтверждает неразрывность и~единство Бога <<Я~и Отец одно>> Ин~10:30 и~<<Верьте Мне, что Я~в~Отце и~Отец во Мне есть>> Ин~14:11.
    \end{itemize}
    \item Писание исповедует монотеизм
        \begin{itemize}
        \item Книга Второзаконие 6:4 <<Слушай, Израиль: Господь, Бог наш, Господь един есть…>>
        \item Библия говорит об одном Боге, а~не проповедует политеизм, присущий множеству окружающих культур того и~настоящего времени <<да не будет у~тебя других богов пред лицем Моим>> Исх 20:3 и~<<Ибо хотя и~есть так называемые боги, или на небе, или на земле, так как есть много богов и~господ много,~--- но у~нас один Бог Отец, из Которого все, и~мы для Него, и~один Господь Иисус Христос, Которым все, и~мы Им>> 1~Кор. 8:1--6.
        \end{itemize}
    \item Писание использует различные формы множественности лиц при описании Бога
        \begin{itemize}
        \item имя Бога <<Элохим>>~--- сущ. мн. ч, используемое с~глаголом ед.ч. <<И сказал Бог: сотворим человека по образу Нашему по подобию Нашему>> в~Бытие 1:26
        \item Диалог между Личностями внутри Бога <<Сказал Господь Господу моему: седи одесную Меня, доколе положу врагов Твоих в~подножие ног Твоих>>. (Пс 109:1)
        \end{itemize}
    \item Новый Завет последовательно описывает Бога как Единого, но все же приписывает божественность каждому из трех лиц Троицы
        \begin{itemize}
        \item Иисус несколько раз упоминает о~Себе в~Евангелии от Иоанна, используя «Я~есть», идентифицируя Себя с~Богом. <<Иисус сказал им: истинно, истинно говорю вам: прежде нежели был Авраам, Я~есмь>>. Иоанна 8:58
        \item Иисус постоянно приглашает людей узнать Его как Сына Божьего
        \item Иисус принимает поклонение, соглашаясь с~исповеданием других, что Он есть Бог.
        \end{itemize}
    \item Тайна Троицы в~Иоанна 1:1: <<Слово>> описывается как отличное от Бога, но вместе с~тем оно с~(рядом, вместе) Богом и~даже есть Сам Бог. <<В начале было Слово, и~Слово было у~(вместе, с) Бога, и~Слово было Бог>> 
    \item Триединый Бог определил путь спасения/искупления человека
        \begin{itemize}
        \item Евангелие~--- единственно возможный путь к~Богу. <<но проповедуем премудрость Божию, тайную, сокровенную, которую предназначил Бог прежде веков к~славе нашей," (1 Коринфянам 2:7)
        \item Иисус Христос, Сын Бога, Личность Бога являлся инструментом Бога Отца для нашего спасения "Сего, по определенному совету и~предведению Божию преданного, вы взяли и, пригвоздив руками беззаконных, убили;" (Деян 2:23)
        \item Дух Святой применил искупление к~человечеству <<Но Бог, наш Спаситель, проявил к~нам доброту и~любовь. Он спас нас не за наши праведные дела, которые мы совершили, а~по Своей милости, через возрождающее омовение и~обновление Святым Духом>> (Тит.3:5). Мы пока не разобрали все эти термины <<омовение>> и~<<обновление>>, но очевидно говорится в~контексте спасения души.
        \end{itemize}
\end{enumerate}

\subsection{3. Важность правильного понимания Бога}
\begin{enumerate}
    \item Если мы хорошо понимаем, как выглядят вещи и~поступки в~глазах Бога (то есть истинная система вещей), то мы точно сможем понять, как Ему можно угодить. Ибо не угодить всемогущему Богу как минимум страшно. 
    \item В~этом уроке на основании Писания будут рассмотрены некоторые качества личности Бога, и~это поможет лучше узнать Его.
    \item Качество (или атрибут)~--- это характеристика, неотъемлемое свойство предмета или явления. 
\end{enumerate}

\subsection{4.  Обзор Божественных качеств}
\begin{enumerate}
    \item Вечное и~независимое бытие~--- жизнь в~самом себе, которая не зависит ни от чего.
    \item Святость~--- чист и~отделен от всего творения и~греха.
    \item Божий гнев~--- Он любит чистоту и~осуждает нечистоту.
    \item Суверенность~--- управляет всем ходом вещей с~полным контролем.
    \item Любовь~--- демонстрирует бескорыстную любовь. 
    \item Милость~--- не воздает так, как должно по справедливости.
    \item Истинность~--- всегда говорит аккуратно и~авторитетно.
    \item Духовная сущность~--- нет материального тела, трансцендентность.
    \item Вездесущность~--- присутствует везде.
    \item Всеведение~--- все знает.
    \item Всемогущество~--- его сила не ограничена.
    \item Неизменность~--- никогда не меняется.
    \item Мудрость~--- достигает своих целей самыми лучшими средствами.
    \item Благость~--- обильно заботиться о~своем творении.
    \item Благодать~--- бесплатно дает спасение не заслужившим того грешникам.
\end{enumerate}

\subsection{5. Восприятие Божественных качеств}
\begin{enumerate}
    \item Принято разделять качества Бога на «передаваемые» и~«непередаваемые» другим объектам, например, человеку.
    \begin{itemize}
        \item непередаваемые качества Бога, такие как вечная неизменность и~абсолютное знание, принадлежат только Богу;
        \item передаваемые качества Бога, такие как любовь и~мудрость, находят свое полное и~совершенное выражение в~Нем, но также могут отображаться и~в~меньших масштабах носителями Его образа – людьми. 
        \item Определение: Отражение качеств Бога в~творении (в красоте, сложности и~непостижимости природы, или моральные качества в~человеке) мы называем <<Славой Бога>>. То есть прославить~--- отразить качества Бога перед наблюдателями.
    \item Взаимосвязанность качеств в~Боге. Это означает, что все качества Бога есть всегда. То есть между ними стоит союз И, а~не ИЛИ. Пример: И благой, и~гневающийся.


    \begin{enumerate}
        \item Присутствуют во всем Божестве и~присущи всем трем личностям Бога. Например, Бог всегда абсолютно милостив, абсолютно справедлив и~абсолютно свят.
        \item Вечная природа качества Бога, то есть Бог не приобретал этикачества во времени, но всегда был таков, каков Есть.
        \item Бог не получает и~не утрачивает Свои качества. Он уже совершенен, потому никакая доработка или исправление не нужно.
        \item Следует отметить неизменность Бога Ветхого и~Нового заветов. Принято считать, что Бог Ветхого Завета~--- злой тиран, а~Нового~--- любящий и~милостивый. Это значит, что кто-то не читал Библию, но ухватился за чье-то мнение не разобравшись. Пример: Послание к~Римлянам 1 глава и~3 глава Евангелия от Иоанна возвещают о~гневе Творца.
        \item Качества раскрывают и~дополняют друг друга.
        \begin{itemize}
            \item В~богословии принято различать что-то, а~не разделять. Пример: если я~разделю у~студентов разум и~тело, то, скорее всего, студент умрет. Потому я~различаю, что у~студента есть и~разум, и~тело.
            \item Невозможно разложить Бога на отдельные составляющие, как нет смысла картину разбивать на пиксели и~рассматривать.
        \end{itemize}
    \end{enumerate}
    \end{itemize}
\end{enumerate}

\subsection{Вывод}

Если мы верно воспринимаем Бога, то наша жизнь будет точно угодной Ему, а~раз угодной Ему, то можно избежать одной насущной проблемы. Его гнева.

\end{document}
