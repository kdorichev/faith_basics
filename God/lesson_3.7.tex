\documentclass[a4paper,12pt]{article} 
\usepackage[left=2cm,right=2cm,top=2cm,bottom=2cm]{geometry}
%%% Работа с~русским языком
\usepackage{cmap}					% поиск в~PDF
\usepackage[LGR, T2A]{fontenc}			% кодировка
\usepackage[greek, russian]{babel}	% локализация и~переносы
\usepackage{alphabeta}
\usepackage{hyperref}  % для гиперссылок

\hypersetup{
    colorlinks=true,
    linkcolor=black,
    filecolor=magenta,      
    urlcolor=cyan,
}

\title{\textsc{Основы Веры}\\Урок №3.7 Божья милость}
\author{Библейская Церковь Санкт-Петербурга}
\date{}
\setcounter{secnumdepth}{0}  % Unnumbered sections in toc

\begin{document}

\maketitle

\thispagestyle{empty}

\tableofcontents

\section*{Введение}

К величайшему сожалению мы склонны трактовать и~определять термины так, что их смысл оказывается вдали от реальности. Термин милость является одним из таковых. Попробуем немного посмотреть на милость глазами Бога.

Рекомендуемое время~--- 20 минут.
        
\subsubsection*{Основные тексты}

\noindent Исайя 55:7

\begin{quote}
Да оставит нечестивый путь свой и~беззаконник~--- помыслы свои, и~да обратится к~Господу, и~Он помилует его, и~к Богу нашему, ибо Он многомилостив. 
\end{quote}

\noindent Псалом 102:8

\begin{quote}
Щедр и~милостив Господь, долготерпелив и~многомилостив\ldots 
\end{quote}

\noindent Числа 14:18
\begin{quote}
Господь долготерпелив и~многомилостив, прощающий беззакония и~преступления, и~не оставляющий без наказания, но наказывающий беззаконие отцов в~детях до третьего и~четвертого рода.
\end{quote}

\section*{Цели урока}

\begin{enumerate}
    \item Объяснить соотношение милости Божьей с~другими качествами Бога.
    \item Рассмотреть проявление милости в~спасении.
    \item Возблагодарить Бога за Его милость к~грешному человеку, заслуживающего лишь осуждение и~вечное наказание.
\end{enumerate}

\section*{Конспект}

\subsection{1. Соотношение милости и~других качеств Бога}

\begin{enumerate}

    \item Милость, милосердие, долготерпение в~Писании 
    \begin{itemize}
        \item <<Но Ты, Господи, Боже щедрый и~благосердный, долготерпеливый и~многомилостивый и~истинный>> (Пс 85:15) 
        \item <<Щедр и~милостив Господь, долготерпелив и~многомилостив>> (Пс 102:8)
        \item <<Раздирайте сердца ваши, а~не одежды ваши, и~обратитесь к~Господу Богу вашему; ибо Он благ и~милосерд, долготерпелив и~многомилостив и~сожалеет о~бедствии>> (Иол 2:13)
        \item <<И прошел Господь пред лицем его и~возгласил: Господь, Господь, Бог человеколюбивый и~милосердный, долготерпеливый и~многомилостивый и~истинный>> (Исх. 34:6)
        \begin{enumerate}
            \item милость Бога означает благорасположение Бога к~тем, кто заслуживает только осуждения и~наказания
            \item милосердие Бога означает Его благость к~пребывающим в~страдании и~нужде
        \end{enumerate}
        \item только Бог может является самым истинным утешителем, знающим как и~чем можно утешить страдающего и~нуждающегося <<Благословен Бог и~Отец Господа нашего Иисуса Христа, Отец милосердия и~Бог всякого утешения>> (2 Кор 1:3)
        \begin{enumerate}
            \item пример милосердного отношения Бога к~людям является для нас примером нашего отношения к~людям
            \item долготерпение означает Его благость к~тем, кто не принимает Его как Бога, противится Ему и~упорствует в~грехе
        \end{enumerate}

         
    \end{itemize}

    \item Разница милости и~благодати
    \begin{itemize}
        \item Милость~--- не получить, что заслужил~--- Божий гнев.
        \item Благодать~--- получить, что не заслужил~--- Божью любовь.
    \end{itemize}
\end{enumerate}

\subsection{2. Ключевые особенности милости Божьей}

\begin{enumerate}
    \item Добровольность
    
    \begin{itemize}
        \item милость (в значении <<помилование>>) никогда не обретается в~обмен на что-либо со стороны человека — она даруется исключительно по свободному расположению Бога:
        \begin{quote}
        <<Итак помилование зависит не от желающего и~не от подвизающегося, но от Бога милующего>>, Рим. 9:16;
        \end{quote}
        \item Бог властен не проявлять Свою милость:
        \begin{quote}
        <<Я Господь, Я~говорю: это придет и~Я сделаю; не отменю и~не пощажу, и~не помилую. По путям твоим и~по делам твоим будут судить тебя, говорит Господь Бог>>, Иез. 24:14
        \end{quote}
        \item Суверенный характер Божьей милости (Рим. 9:15--16~--- цитата из Исх. 33:19)
        \begin{itemize}
        \item <<Итак помилование зависит не от желающего и~не от подвизающегося, но от Бога милующего>> (Рим. 9:16)
        \item <<Ибо Он говорит Моисею: кого миловать, помилую; кого жалеть, пожалею.>> (Рим 9:15)
        \item <<И сказал Господь: Я~проведу пред тобою всю славу Мою и~провозглашу имя Иеговы пред тобою, и~кого помиловать~--- помилую, кого пожалеть~--- пожалею.>> (Исх 33:19)
        \end{itemize}
    \end{itemize}
    
    \item Конкретная направленность на объект помилования
    
    \begin{itemize}
        \item определенный человек
        \begin{quote}
        <<Призри на меня и~помилуй меня, как поступаешь с~любящими имя Твоё>>,  Псалом 118:132;
        \end{quote}

        \item народ Израильский
        \begin{enumerate}
            \item <<да призрит на тебя Господь светлым лицем Своим и~помилует тебя!>> (Числ 6:25) 
            \item <<Но Господь умилосердился над ними, и~помиловал их, и~обратился к~ним ради завета Своего с~Авраамом, Исааком и~Иаковом, и~не хотел истребить их, и~не отверг их от лица Своего доныне>> (4 Цар 13:23) 
            \item <<И сказал Господь: Я~увидел страдание народа Моего в~Египте и~услышал вопль его от приставников его; Я~знаю скорби его>> (Исх 3:7)
        \end{enumerate}

        \item спасенные во Христе:
        \begin{quote}
        <<некогда не народ, а~ныне народ Божий; некогда непомилованные, а~ныне помилованы>>, 1~Пет. 2:10.
        \end{quote}
    \end{itemize}
\end{enumerate}

\subsection{3. Милость и~спасение}
Человек приобретает спасение исключительно по милости Божьей.
\begin{enumerate}
    \item грешники получают милость даром через искупление во Христе
    \begin{quote}
    <<потому что все согрешили и~лишены славы Божией, получая оправдание даром, по благодати Его, искуплением во Христе Иисусе>>, Рим. 3:23--24;
    \end{quote}
    \item милость Божья изменяет грешного человека, делая его свободным от тяжести вины за грех:
    \begin{quote}
    <<И такими были некоторые из вас; но омылись, но освятились, но оправдались именем Господа нашего Иисуса Христа и~Духом Бога нашего>> 1~Кор. 6:11;
    \end{quote}

    \item милость Божья изменяет статус грешного человека перед Самим Собой и~Своё отношение к~окончательной его участи:
    \begin{quote}
    <<некогда не народ, а~ныне народ Божий; некогда непомилованные, а~ныне помилованы>>, 1~Пет. 2:10.
    \end{quote}
\end{enumerate}

\noindent \emph{Вывод:} Спасительная Милость Божья приходит в~жизнь человека через покаяние и~веру в~жертву Христа. 

\vfill
\tiny{Актуальную (с последними правками) версию документа всегда можно найти на сайте \href{https://github.com/kdorichev/faith_basics/blob/main/God/lesson_3.7.pdf}{github.com/kdorichev/faith\_basics}}

\end{document}
