\documentclass[a4paper,12pt]{article} 
\usepackage[left=2.5cm,right=2.5cm,top=2cm,bottom=2cm]{geometry}
%%% Работа с~русским языком
\usepackage{cmap}					% поиск в~PDF
\usepackage[T2A]{fontenc}			% кодировка
\usepackage[utf8]{inputenc}			% кодировка исходного текста
\usepackage[russian]{babel}	% локализация и~переносы
\usepackage{hyperref}  % для гиперссылок
\hypersetup{
    colorlinks=true,
    linkcolor=blue,
    filecolor=magenta,      
    urlcolor=cyan,
}

\title{Основы веры}

\author{Библейская Церковь Санкт-Петербурга}
\date{Январь 2025}

\begin{document}

\maketitle

\thispagestyle{empty}

\begin{abstract}
    Данное пособие подготовлено в~помощь желающим познакомиться с~основами христианской веры и~построено на самостоятельном изучении Евангелия от Иоанна. Тексты Евангелия взяты по Синодальному переводу. Вопросы заимствованы из пособия Русской Библейской Церкви (г. Москва). Вёрстка выполнена в~Библейской Церкви г.\ Санкт-Петербурга.
\end{abstract}

\newpage

\tableofcontents

% \newpage

\newcommand{\myline}{\noindent\makebox[\linewidth]{\rule{\linewidth}{0.1pt}}}

\section{Евангелие от Иоанна 1:1-8. Превосходство Христа}

\textsubscript{1} В~начале было Слово, и~Слово было у~Бога, и~Слово было Бог. \textsubscript{2}~Оно было в~начале у~Бога. \textsubscript{3}~Все через Него начало быть, и~без Него ничто не начало быть, что начало быть. \textsubscript{4}~В~Нем была жизнь, и~жизнь была свет людей. \textsubscript{5}~И~свет во тьме светит, и~тьма не объяла его. \textsubscript{6}~Был человек, посланный от Бога; имя ему Иоанн. \textsubscript{7}~Он пришёл для свидетельства, чтобы свидетельствовать о~Свете, дабы все уверовали чрез него. \textsubscript{8}~Он не был свет, но был послан, чтобы свидетельствовать о~Свете.

\subsection*{Вопросы по наблюдению за текстом}
\begin{enumerate}
    \item Какие три важных обстоятельства в~этих первых стихах сообщает нам автор о~Христе как о~Слове Божьем (см. стих~1)? 
    
    \myline

    \myline

    \myline
    
    \item Кто, согласно тексту, был послан Богом, «чтобы свидетельствовать о~Свете», то есть о~Христе?
    
    \myline

\end{enumerate}

\subsection*{Вопросы по применению текста}

\begin{enumerate}
    \item Говоря о~Слове как об источнике созидания, почему Иоанн особо подчеркивает, что «без Него ничто не начало быть, что начало быть»?
    
    \myline
    
    \myline

    \item Поразмышляйте, что важно в~упоминании евангелиста о~«свете»? 
    
    \myline
    
    \myline
    
    \item Как вы думаете, каким образом Иоанн Креститель служил «образцом» христианского свидетеля? 
    
    \myline
    
    \myline
    
    \item Как понимание того, что Христос является «Словом», должно повлиять на ваше отношение к~Нему?
    
    \myline
    
    \myline

\end{enumerate}

% \newpage

% \newpage

\section{Евангелие от Иоанна 1:9-13. Реакция человечества на приход Христа}

\textsubscript{9}~Был Свет Истинный, Который просвещает всякого человека, приходящего в~мир. \textsubscript{10}~В~мире был, и~мир чрез Него начал быть, и~мир Его не познал. \textsubscript{11}~Пришёл к~Своим, и~Свои Его не приняли. \textsubscript{12}~А~тем, которые приняли Его, верующим во имя Его, дал власть быть чадами Божиими, \textsubscript{13}~которые ни от крови, ни от хотения плоти, ни от хотения мужа, но от Бога родились.

\subsection*{Вопросы по наблюдению за текстом}
\begin{enumerate}
    \item Чем, согласно стиху 10, «мир» отличается от «своих»? 
    
    \myline
    
    \myline

    \item Согласно стиху 9, что делает «Свет истинный»?

    \myline
    
    \myline   
\end{enumerate}

\subsection*{Вопросы по применению текста} 
\begin{enumerate}
    \item Что имеет в~виду Иоанн под словом «мир»? Что он сообщает нам о~нём (см. стих 10)? 
    
    \myline
    
    \myline

    \item На ваш взгляд, почему «свои» не приняли Христа (см. стих 11)? 
    
    \myline
    
    \myline

    \item Что происходит с~теми, кто принимает Его, согласно стихам 11-13? 
    
    \myline
    
    \myline

    \item Какие три обстоятельства выделяет автор в~описании спасения в~стихах 12, 13? Что это означает для вас?
    
    \myline
    
    \myline

\end{enumerate}


% \newpage

\section{Евангелие от Иоанна 1:14-18. Воплощение Христа}

\textsubscript{14}~И~Слово стало плотью, и~обитало с~нами, полное благодати и~истины; и~мы видели славу Его, славу, как Единородного от Отца. \textsubscript{15}~Иоанн свидетельствует о~Нем и, восклицая, говорит: «Сей был Тот, о~Котором я~сказал, что Идущий за мною стал впереди меня, потому что был прежде меня». \textsubscript{16}~И~от полноты Его все мы приняли и~благодать на благодать, \textsubscript{17}~ибо закон дан чрез Моисея; благодать же и~истина произошли чрез Иисуса Христа. \textsubscript{18}~Бога не видел никто никогда; Единородный Сын, сущий в~недре Отчем, Он явил.

\subsection*{Вопросы по наблюдению за текстом}
\begin{enumerate}
    \item Какими словами Иоанн Креститель свидетельствует о~Христе? Ответьте текстом Писания. 
    
    \myline
    
    \myline

    \item Какого ветхозаветного персонажа упоминает автор? И~что было дано через него? 
    
    \myline
    
    \myline
    \item Что 18~стих говорит об Отце и~о Сыне?
    
    \myline
    
    \myline
\end{enumerate}

\subsection*{Вопросы по применению текста} 
\begin{enumerate}
    \item Какие события отражает автор в~стихе 14? Почему это важно?
    
    \myline
    
    \myline
    \item Какова взаимосвязь между Иоанном и~Иисусом (см. стихи 6-9 и~15)? 
    
    \myline
    
    \myline
    \item Как вы думаете, почему евангелист включил данное повествование в~свое вступление? Дайте развернутый ответ. 
    
    \myline
    
    \myline
    \item Как через Иисуса Христа вы яснее увидели славу Божью?
    
    \myline
    
    \myline
\end{enumerate}


% \newpage

\section{Евангелие от Иоанна 1:19-34. Свидетельство Иоанна Крестителя о~Христе}

 \textsubscript{19}~И~вот свидетельство Иоанна, когда иудеи прислали из Иерусалима священников и~левитов спросить его: «Кто ты?» \textsubscript{20}~Он объявил, и~не отрекся, и~объявил: «Я не Христос». \textsubscript{21}~И~спросили его: «Что же? Ты Илия?» Он сказал: «Нет». «Пророк?» Он отвечал: «Нет». \textsubscript{22}~Сказали ему: «Кто же ты? Чтобы нам дать ответ пославшим нас: что ты скажешь о~себе самом?» \textsubscript{23}~Он сказал: «Я “глас вопиющего в~пустыне: исправьте путь Господу” , как сказал пророк Исаия». \textsubscript{24}~А~посланные были из фарисеев; \textsubscript{25}~И~они спросили его: «Что же ты крестишь, если ты ни Христос, ни Илия, ни пророк?» \textsubscript{26} Иоанн сказал им в~ответ: «Я крещу в~воде; но стоит среди вас Некто, Которого вы не знаете. \textsubscript{27}~Он-то Идущий за мною, но Который стал впереди меня. Я~недостоин развязать ремень у~обуви Его». \textsubscript{28}~Это происходило в~Вифаваре при Иордане, где крестил Иоанн. 
 
 \textsubscript{29}~На другой день видит Иоанн идущего к~нему Иисуса и~говорит: «Вот Агнец Божий, Который берет на Себя грех мира. \textsubscript{30}~Сей есть, о~Котором я~сказал: “За мною идет Муж, Который стал впереди меня, потому что Он был прежде меня”. \textsubscript{31}~Я~не знал Его; но для того пришёл крестить в~воде, чтобы Он явлен был Израилю». \textsubscript{32}~И~свидетельствовал Иоанн, говоря: «Я видел Духа, сходящего с~неба, как голубя, и~пребывающего на Нем. \textsubscript{33}~Я~не знал Его; но Пославший меня крестить в~воде сказал мне: “На Кого увидишь Духа сходящего и~пребывающего на Нем, Тот есть крестящий Духом Святым”. \textsubscript{34}~И~я~видел и~засвидетельствовал, что Сей есть Сын Божий».

\subsection*{Вопросы по наблюдению за текстом}
\begin{enumerate}
    \item Где крестил Иоанн? Кого иудеи подослали к~нему? Какие вопросы пришедшие задавали Иоанну Крестителю? 
    
    \myline
    
    \myline
    \item Какого пророка Ветхого Завета цитирует Иоанн Креститель? 
    
    \myline
    
    \myline
    \item Что данный текст говорит о~характере Иоанна Крестителя? Перечислите конкретные качества.
    
    \myline
    
    \myline
\end{enumerate}

\subsection*{Вопросы по применению текста} 
\begin{enumerate}
    \item В~этом разделе Иоанн использует четыре титула Иисуса: Мессия, Агнец Божий, крестящий Духом Святым, Сын Божий. Что означает титул «Мессия»? Что он сообщает нам об Иисусе? 
    
    \myline
    
    \myline
    \item В~стихе 29~Иоанн Креститель говорит о~Христе, как об «Агнце». Что это значит? Постарайтесь дать развернутый ответ. 
    
    \myline
    
    \myline
    \item Какое значение имеет то, что Иоанн Креститель видел голубя, пребывающего на Иисусе? 
    
    \myline
    
    \myline
\end{enumerate}


% \newpage

\section{Евангелие от Иоанна 1:35-51. Призвание первых учеников}

\textsubscript{35}~На другой день опять стоял Иоанн и~двое из учеников его. \textsubscript{36}~И, увидев идущего Иисуса, сказал: «Вот Агнец Божий». \textsubscript{37}~Услышав от него эти слова, оба ученика пошли за Иисусом. \textsubscript{38}~Иисус же, обратившись и~увидев их идущих, говорит им: «Что вам надобно? Они сказали Ему: «Равви, (что значит “учитель”), где живешь?» \textsubscript{39}~Говорит им: «Пойдите и~увидите». Они пошли и~увидели, где Он живет, и~пробыли у~Него день тот. Было около десятого часа. \textsubscript{40}~Один из двух, слышавших от Иоанна об Иисусе и~последовавших за Ним, был Андрей, брат Симона Петра. \textsubscript{41}~Он первый находит брата своего Симона и~говорит ему: «Мы нашли Мессию, что означает «Христос», \textsubscript{42}~и~привел его к~Иисусу. Иисус же, взглянув на него, сказал: «Ты~--- Симон, сын Ионин; ты наречешься Кифа», что означает «камень» (Петр). 

\textsubscript{43}~На другой день Иисус захотел идти в~Галилею, и~находит Филиппа и~говорит ему: «Иди за Мною». \textsubscript{44}~Филипп же был из Вифсаиды, из одного города с~Андреем и~Петром. \textsubscript{45}~Филипп находит Нафанаила и~говорит ему: «Мы нашли Того, о~Котором писали Моисей в~законе и~пророки: Иисуса, сына Иосифа, из Назарета». \textsubscript{46}~Но Нафанаил сказал ему: «Из Назарета может ли быть что доброе?» Филипп говорит ему: «Пойди и~посмотри». \textsubscript{47}~Иисус, увидев идущего к~Нему Нафанаила, говорит о~нём: «Вот подлинно израильтянин, в~котором нет лукавства». \textsubscript{48}~Нафанаил говорит Ему: «Откуда Ты знаешь меня?» Иисус сказал ему в~ответ: «Прежде нежели позвал тебя Филипп, когда ты был под смоковницей, Я~видел тебя». \textsubscript{49}~Нафанаил отвечал Ему: «Равви! Ты Сын Божий, Ты Царь Израилев». \textsubscript{50}~Иисус сказал ему в~ответ: «Ты веришь, потому что Я~тебе сказал: “Я видел тебя под смоковницей”,~--- увидишь больше этого». \textsubscript{51}~И~говорит ему: «Истинно, истинно говорю вам: отныне будете видеть небо отверстым и~ангелов Божиих восходящих и~нисходящих к~Сыну Человеческому». 

\subsection*{Вопросы по наблюдению за текстом}
\begin{enumerate}
    \item О~призвании каких учеников вы читаете в~этом тексте? Перечислите их имена. 
    
    \myline
    
    \myline
    \item Как первые ученики обращаются ко Христу (см. стих 38)? 
    
    \myline
    
    \myline
    \item Как Христос называет Себя в~51 стихе?
    
    \myline
    
    \myline
\end{enumerate}

\subsection*{Вопросы по применению текста} 
\begin{enumerate}
    \item Какие основные требования для того, чтобы стать учеником Христа, упоминает Иоанн? А~вы исполнили их? 
    
    \myline
    
    \myline
    \item Что делает первый ученик после того, как решает следовать за Христом (ст. 41)? Пример чего он показывает вам? 
    
    \myline
    
    \myline
    \item Что в~стихе 49~говорится о~Христе? Какими титулами он здесь называется? Что значат эти титулы? 
    
    \myline
    
    \myline
\end{enumerate}


% \newpage

\section{Евангелие от Иоанна 2:1-11. Начало чудес}

\textsubscript{1}~На третий день был брак в~Кане Галилейской, и~Матерь Иисуса была там. \textsubscript{2}~Был также зван Иисус и~ученики Его на брак. \textsubscript{3}~И~как недоставало вина, то Матерь Иисуса говорит Ему: «Вина нет у~них». \textsubscript{4}~Иисус говорит Ей: «Что Мне и~тебе, женщина? Ещё не пришёл час Мой». \textsubscript{5}~Мать Его сказала служителям: «Что скажет Он вам, то сделайте». \textsubscript{6}~Было же тут шесть каменных водоносов, стоявших по обычаю очищения иудейского, вмещавших по две или по три меры. \textsubscript{7}~Иисус говорит им: «Наполните сосуды водою». И~наполнили их доверху. \textsubscript{8}~И~говорит им: «Теперь почерпните и~несите к~распорядителю пира». И~понесли. \textsubscript{9}~Когда же распорядитель отведал воды, сделавшейся вином,~--- а~он не знал, откуда это вино, знали только служители, черпавшие воду,~--- тогда распорядитель зовет жениха \textsubscript{10}~и~говорит ему: «Всякий человек подает сперва хорошее вино, а~когда напьются, тогда худшее; а~ты хорошее вино сберег до сих пор». \textsubscript{11}~Так положил Иисус начало чудесам в~Кане Галилейской и~явил славу Свою; и~уверовали в~Него ученики Его.

\subsection*{Вопросы по наблюдению за текстом}
\begin{enumerate}
    \item Где находится Иисус со своими учениками? Какое событие описано автором? 
    
    \myline
    
    \myline
    \item С~какими словами мать Иисуса обратилась к~Нему? 
    
    \myline
    
    \myline
    \item Какое чудо совершил Христос? 
    
    \myline
    
    \myline
    \item Что удивило распорядителя пира? Ответьте, ссылаясь на текст.

    \myline
    
    \myline
\end{enumerate}

\subsection*{Вопросы по применению текста} 
\begin{enumerate}
    \item Почему, по-вашему, мать Иисуса попросила Его сделать что-нибудь для решения возникшей проблемы? 
    
    \myline
    
    \myline
    \item Прочитайте ещё раз ответ Иисуса матери. Что вас удивляет в~ответе Иисуса? И~почему? 
    
    \myline
    
    \myline
    \item Что стало немедленным результатом первого знамения Иисуса? Кто полностью понимал происходящее? Почему ученики уверовали в~Иисуса? 
    
    \myline
    
    \myline
    \item По вашему мнению, что совершённое чудо говорит о~природе Христа? 

    \myline
    
    \myline
\end{enumerate}


% \newpage

\section{Евангелие от Иоанна 2:12-25. Очищение храма}

\textsubscript{12}~После этого пришёл Он в~Капернаум, Сам и~мать Его, и~братья его, и~ученики Его; и~там пробыли немного дней. 

\textsubscript{13}~Приближалась Пасха иудейская, и~Иисус пришёл в~Иерусалим \textsubscript{14}~и~нашёл, что в~храме продавали волов, овец и~голубей, и~сидели меновщики денег. \textsubscript{15}~И, сделав бич из веревок, выгнал из храма всех, также и~овец и~волов; и~деньги у~меновщиков рассыпал, а~столы их опрокинул. \textsubscript{16}~И~сказал продающим голубей: «Возьмите это отсюда и~дома Отца Моего не делайте домом торговли». \textsubscript{17}~При этом ученики Его вспомнили, что написано: «Ревность по дому Твоем съедает Меня». \textsubscript{18}~На это иудеи сказали: «Каким знамением докажешь Ты нам, что имеешь власть так поступать?» \textsubscript{19}~Иисус сказал им в~ответ: «Разрушьте храм этот, и~Я в~три дня воздвигну его». \textsubscript{20}~На это сказали иудеи: «Этот храм строился сорок шесть лет, и~Ты в~три дня воздвигнешь его?» \textsubscript{21}~А~Он говорил о~храме тела Своего. \textsubscript{22}~Когда же воскрес Он из мертвых, то ученики Его вспомнили, что Он говорил это, и~поверили Писанию и~слову, которое сказал Иисус. \textsubscript{23}~И~когда Он был в~Иерусалиме на празднике Пасхи, то многие, видя чудеса, которые Он творил, уверовали во имя Его. \textsubscript{24}~Но Сам Иисус не вверял Себя им, потому что знал всех \textsubscript{25}~и~не имел нужды, чтобы кто засвидетельствовал о~человеке, ибо Сам знал, что в~человеке.

\subsection*{Вопросы по наблюдению за текстом}
\begin{enumerate}
    \item Куда пришёл Христос после совершения чуда в~Кане Галилейской? Кто сопровождал Его? 
    
    \myline
    
    \myline
    \item Какой праздник приближался? В~какой город пошёл Иисус? 
    
    \myline
   
    \myline
    \item Что Христос увидел в~храме? Как Он на это отреагировал?
    
    \myline
    
    \myline
\end{enumerate}

\subsection*{Вопросы по применению текста} 
\begin{enumerate}
    \item Почему Христос так радикально отреагировал на происходящее в~храме? На ваш взгляд, в~чём была проблема людей, которые находились на тот момент в~храме? 
    
    \myline
    
    \myline
    \item Как отреагировали ученики на очищение храма Христом?
    
    \myline
    
    \myline
    \item Как отреагировали религиозные лидеры на эти же события? 
    
    \myline
    
    \myline
    \item Что имел в~виду Иисус, говоря в~стихе 19: «Разрушьте храм этот, и~Я в~три дня воздвигну его»? 
    
    \myline
    
    \myline
    \item На ваш взгляд, что значат слова евангелиста, что многие уверовали во Христа (см. стих 23), но Он не вверял Себя им (см. стих 24)? Постарайтесь дать развернутый ответ. 
    
    \myline
    
    \myline
    \item В~свете всего данного отрывка поразмышляйте и~напишите, в~чём отличие поверхностной веры от подлинной.
    
    \myline
    
    \myline
\end{enumerate}


% \newpage

\section{Евангелие от Иоанна 3:1-21. Новое рождение}

\textsubscript{1}~Между фарисеями был некто по имени Никодим, один из начальников иудейских. \textsubscript{2}~Он пришёл к~Иисусу ночью и~сказал Ему: «Равви! Мы знаем, что Ты Учитель, пришедший от Бога, ибо таких чудес, какие Ты творишь, никто не может творить, если не будет с~ним Бог». \textsubscript{3}~Иисус сказал ему в~ответ: «Истинно, истинно говорю тебе: если кто не родится свыше, не может увидеть Царства Божьего». \textsubscript{4}~Никодим говорит Ему: «Как может человек родиться, будучи стар? Неужели может он в~другой раз войти в~утробу матери своей и~родиться?» \textsubscript{5}~Иисус отвечал: «Истинно, истинно говорю тебе: если кто не родится от воды и~Духа, не может войти в~Царство Божие. \textsubscript{6}~Рожденное от плоти есть плоть, а~рожденное от Духа есть дух. \textsubscript{7}~Не удивляйся тому, что Я~сказал тебе: должно вам родиться свыше. \textsubscript{8}~Дух дышит где хочет, и~голос его слышишь, а~не знаешь, откуда приходит и~куда уходит. Так бывает со всяким, рожденным от Духа». \textsubscript{9}~Никодим сказал Ему в~ответ: «Как это может быть?» \textsubscript{10}~Иисус отвечал и~сказал ему: «Ты учитель Израилев и~этого ли не знаешь? \textsubscript{11}~Истинно, истинно говорю тебе: Мы говорим о~том, что знаем, и~свидетельствуем о~том, что видели, а~вы свидетельства Нашего не принимаете. \textsubscript{12}~Если Я~сказал вам о~земном, и~вы не верите, как поверите, если буду говорить вам о~небесном? \textsubscript{13}~Никто не восходил на небо, как только сошедший с~небес Сын Человеческий, сущий на небесах. \textsubscript{14}~И~как Моисей вознес змею в~пустыне, так должен быть вознесен Сын Человеческий, \textsubscript{15}~дабы всякий, верующий в~Него, не погиб, но имел жизнь вечную. \textsubscript{16}~Ибо так возлюбил Бог мир, что отдал Сына Своего единородного, дабы всякий, верующий в~Него, не погиб, но имел жизнь вечную. \textsubscript{17}~Ибо не послал Бог Сына Своего в~мир, чтобы судить мир, но чтобы мир спасен был через Него. \textsubscript{18}~Верующий в~Него не судится, а~неверующий уже осужден, потому что не уверовал во имя единородного Сына Божьего. \textsubscript{19}~Суд же состоит в~том, что свет пришёл в~мир; но люди более возлюбили тьму, нежели свет, потому что дела их были злы; \textsubscript{20}~ибо всякий, делающий злое, ненавидит свет и~не идет к~свету, чтобы не обличились дела его, потому что они злы, \textsubscript{21}~а~поступающий по правде идет к~свету, дабы явны были дела его, потому что они в~Боге сделаны».

\subsection*{Вопросы по наблюдению за текстом}
\begin{enumerate}
    \item Согласно тексту, кто пришёл к~Иисусу? Кем был пришедший, и~в~какое время он пришёл? 
    
    \myline
    
    \myline
    \item Что не может увидеть тот, кто не родится свыше (см. стих~3)? 
    
    \myline
    
    \myline
    \item Согласно стихам с~5 по 8, кто инициатор возрождения (рождения свыше)? 
    
    \myline
    
    \myline
    \item Посмотрите внимательно на стих 16~и~скажите, что отличает верующего в~Сына Божьего от неверующего?
    
    \myline
    
    \myline
\end{enumerate}

\subsection*{Вопросы по применению текста} 
\begin{enumerate}
    \item Как вы думаете, в~чём была основная проблема Никодима? 
    
    \myline
    
    \myline
    \item Имеете ли вы рождение свыше (рождение от Духа)? Если да, то на каком основании вы делаете такой вывод? 
    
    \myline
    
    \myline
    \item Есть ли уверенность в~вашем сердце в~том, что вы увидите Царство Божье и~войдете в~него? 
    
    \myline
    
    \myline
    \item Кто из дорогих вам людей не имеет спасения? И~что нужно, чтобы они обрели его?
    
    \myline
    
    \myline
\end{enumerate}

% \newpage

\section{Евангелие от Иоанна 3:22-4:3. Христос и~Иоанн Креститель}

\textsubscript{22}~После этого пришёл Иисус с~учениками Своими в~землю иудейскую и~там жил с~ними и~крестил. \textsubscript{23}~А~Иоанн также крестил в~Еноне, близ Салима, потому что там было много воды; и~приходили туда, и~крестились, \textsubscript{24}~ибо Иоанн ещё не был заключен в~темницу. \textsubscript{25}~Тогда у~Иоанновых учеников произошёл спор с~иудеями об очищении. \textsubscript{26}~И~пришли к~Иоанну, и~сказали ему: «Равви! Тот, Который был с~тобой при Иордане и~о Котором ты свидетельствовал, вот, Он крестит, и~все идут к~Нему». \textsubscript{27}~Иоанн сказал в~ответ: «Не может человек ничего принимать на себя, если не будет дано ему с~неба. \textsubscript{28}~Вы сами мне свидетели в~том, что я~сказал: «Не я~Христос», но «Я послан пред Ним». \textsubscript{29}~Имеющий невесту есть жених, а~друг жениха, стоящий и~внимающий ему, радостью радуется, слыша голос жениха. Эта-то радость моя исполнилась. \textsubscript{30}~Ему должно расти, а~мне умаляться. \textsubscript{31}~Приходящий свыше и~есть выше всех; а~сущий от земли земной и~есть и~говорит, как сущий от земли. Приходящий с~небес выше всех, \textsubscript{32}~и~что Он видел и~слышал, о~том и~свидетельствует; и~никто не принимает свидетельства Его. \textsubscript{33}~Принявший Его свидетельство этим запечатлел, что Бог истинен. \textsubscript{34}~Ибо Тот, Которого послал Бог, говорит слова Божии, так как не мерой дает Бог Духа. \textsubscript{35}~Отец любит Сына и~всё дал в~руку Его. \textsubscript{36}~Верующий в~Сына имеет жизнь вечную, а~не верующий в~Сына не увидит жизни, но гнев Божий пребывает на нем».

\textsubscript{1} Когда же узнал Иисус о~дошедшем до фарисеев слухе, что Он более приобретает учеников и~крестит, нежели Иоанн,~--- \textsubscript{2}~хотя Сам Иисус не крестил, а~ученики Его,~--- \textsubscript{3}~то оставил Иудею и~пошёл опять в~Галилею. 

\subsection*{Вопросы по наблюдению за текстом}
\begin{enumerate}
    \item О~чём волновались ученики Иоанна Крестителя (см. стих 26)? 
    
    \myline
    
    \myline
    \item На основании стихов 27-30 кратко сформулируйте, в~чём заключался ответ Иоанна Крестителя своим ученикам?
    
    \myline
    
    \myline
\end{enumerate}

\subsection*{Вопросы по применению текста} 
\begin{enumerate}
    \item Поразмышляйте над тем, что Иоанн Креститель говорит о~Христе в~стихах 31-36. Выпишите то, как он характеризует Христа. 
    
    \myline
    
    \myline
    \item Как вы думаете, о~чём говорит стих 35? Что вы узнаете в~нём о~Христе? 
    
    \myline
    
    \myline
    \item На ваш взгляд, что значит словосочетание «гнев Божий»? И~как это относится к~вам?
    
    \myline
    
    \myline
\end{enumerate}


% \newpage

\section{Евангелие от Иоанна 4:4-42. Христос в~Самарии}

\textsubscript{4}~Надлежало же Ему проходить через Самарию. \textsubscript{5}~Итак, приходит Он в~город самарийский, называемый Сихарь, близ участка земли, данного Иаковом сыну своему Иосифу. \textsubscript{6}~Там был колодец Иакова. Иисус, утомившись от пути, сел у~колодца. Было около шестого часа. \textsubscript{7}~Приходит женщина из Самарии почерпнуть воды. Иисус говорит ей: «Дай Мне пить». \textsubscript{8}~Ибо ученики Его отлучились в~город купить пищи. \textsubscript{9}~Женщина самарийская говорит Ему: «Как Ты, будучи иудеем, просишь пить у~меня, самарянки?» Ибо иудеи с~самарянами не сообщаются. \textsubscript{10}~Иисус сказал ей в~ответ: «Если бы ты знала дар Божий и~Кто говорит тебе: „Дай Мне пить“, то ты сама просила бы у~Него, и~Он дал бы тебе воду живую». \textsubscript{11}~Женщина говорит Ему: «Господин! Тебе и~почерпнуть нечем, а~колодец глубок; откуда же у~Тебя вода живая? \textsubscript{12}~Неужели Ты больше отца нашего Иакова, который дал нам этот колодец, и~сам из него пил, и~дети его, и~скот его?» \textsubscript{13}~Иисус сказал ей в~ответ: «Всякий, пьющий воду эту, возжаждет опять, \textsubscript{14}~а~кто будет пить воду, которую Я~дам ему, тот не будет жаждать вовек; но вода, которую Я~дам ему, сделается в~нём источником воды, текущей в~жизнь вечную». \textsubscript{15}~Женщина говорит Ему: «Господин! Дай мне этой воды, чтобы мне не иметь жажды и~не приходить сюда черпать». \textsubscript{16}~Иисус говорит ей: «Пойди позови мужа твоего и~приди сюда». \textsubscript{17}~Женщина сказала в~ответ: «У меня нет мужа». Иисус говорит ей: «Правду ты сказала, что у~тебя нет мужа, \textsubscript{18}~ибо у~тебя было пять мужей, и~тот, которого ныне имеешь, не муж тебе; это справедливо ты сказала». \textsubscript{19}~Женщина говорит Ему: «Господи! Вижу, что Ты пророк. \textsubscript{20}~Отцы наши поклонялись на этой горе, а~вы говорите, что место, где должно поклоняться, находится в~Иерусалиме». \textsubscript{21}~Иисус говорит ей: «Поверь Мне, что наступает время, когда и~не на горе этой, и~не в~Иерусалиме будете поклоняться Отцу. \textsubscript{22}~Вы не знаете, чему поклоняетесь, а~мы знаем, чему поклоняемся, ибо спасение от иудеев. \textsubscript{23}~Но настанет время и~настало уже, когда истинные поклонники будут поклоняться Отцу в~духе и~истине, ибо таких поклонников Отец ищет Себе. \textsubscript{24}~Бог есть Дух, и~поклоняющиеся Ему должны поклоняться в~духе и~истине». \textsubscript{25}~Женщина говорит Ему: «Знаю, что придет Мессия, то есть Христос; когда Он придет, то возвестит нам всё». \textsubscript{26}~Иисус говорит ей: «Это Я, говорящий с~тобою». \textsubscript{27}~В~это время пришли ученики Его и~удивились, что Он разговаривал с~женщиной; однако ж ни один не сказал: «Чего Ты требуешь?» или «О чём говоришь с~ней?» \textsubscript{28}~Тогда женщина оставила водонос свой, и~пошла в~город, и~говорит людям: \textsubscript{29}~«Пойдите, посмотрите Человека, Который сказал мне всё, что я~сделала: не Он ли Христос?» \textsubscript{30}~Они вышли из города и~пошли к~Нему. \textsubscript{31}~Между тем ученики просили Его, говоря: «Равви! Ешь». \textsubscript{32}~Но Он сказал им: «У Меня есть пища, которой вы не знаете». \textsubscript{33}~Поэтому ученики говорили между собой: «Разве кто принес Ему есть?» \textsubscript{34}~Иисус говорит им: «Моя пища~--- творить волю Пославшего Меня и~совершить дело Его. \textsubscript{35}~Не говорите ли вы, что ещё четыре месяца~--- и~наступит жатва? А~Я говорю вам: возведите очи ваши и~посмотрите на нивы, как они побелели и~поспели к~жатве. \textsubscript{36}~Жнущий получает награду и~собирает плод в~жизнь вечную, так что и~сеющий, и~жнущий вместе радоваться будут, \textsubscript{37}~ибо в~этом случае справедливо изречение: „Один сеет, а~другой жнет“. \textsubscript{38}~Я~послал вас жать то, над чем вы не трудились,~--- другие трудились, а~вы вошли в~труд их». \textsubscript{39}~И~многие самаряне из города того уверовали в~Него по слову женщины, свидетельствовавшей, что Он сказал ей всё, что она сделала. \textsubscript{40}~И~потому, когда пришли к~Нему самаряне, то просили Его побыть у~них; и~Он пробыл там два дня. \textsubscript{41}~И~ещё большее число уверовало по Его слову. \textsubscript{42}~А~женщине той говорили: «Уже не по твоим речам веруем, ибо сами слышали и~узнали, что Он истинно Спаситель мира, Христос». 


\subsection*{Вопросы по наблюдению за текстом}
\begin{enumerate}
    \item Через какую местность проходил Христос? 
    
    \myline
    
    \item В~каком месте остановился Иисус, и~с кем Он там встретился? 
    
    \myline
    
    \myline
    \item Что необычного было в~этой встрече (см. стих~9)? 
    
    \myline
    
    \myline
\end{enumerate}

\subsection*{Вопросы по применению текста} 
\begin{enumerate}
    \item Как вы думаете, в~чём была основная нужда женщины, с~которой беседовал Христос (см. стихи 11-15)? Поясните свой ответ. 
    
    \myline
    
    \myline
    \item Иисус прямо сказал женщине, что Он~--- Мессия, Которого ждут. Почему Он так сделал (см. стихи 25-26)? 
    
    \myline
    
    \myline
    \item Каких поклонников Бог ищет себе (см. стихи 23-24)? Поясните свой ответ. 
    
    \myline
    
    \myline
    \item Почему уверовали жители г. Сихарь (см. стихи 39-42)? 
    
    \myline
    
    \myline
\end{enumerate}


% \newpage

\section{Евангелие от Иоанна 4:43-54. Второе чудо в~Галилее}

 \textsubscript{43}~По прошествии же двух дней Он вышел оттуда и~пошёл в~Галилею, \textsubscript{44}~ибо Сам Иисус свидетельствовал, что пророк не имеет чести в~своем отечестве. \textsubscript{45}~Когда пришёл Он в~Галилею, то галилеяне приняли Его, видев всё, что Он сделал в~Иерусалиме в~праздник,~--- ибо и~они ходили на праздник. \textsubscript{46}~Итак, Иисус опять пришёл в~Кану галилейскую, где претворил воду в~вино. В~Капернауме был некоторый царедворец, у~которого сын был болен. \textsubscript{47}~Он, услышав, что Иисус пришёл из Иудеи в~Галилею, пришёл к~Нему и~просил Его прийти и~исцелить сына его, который был при смерти. \textsubscript{48}~Иисус сказал ему: «Вы не уверуете, если не увидите знамений и~чудес». \textsubscript{49}~Царедворец говорит Ему: «Господи! Приди, пока не умер сын мой». \textsubscript{50}~Иисус говорит ему: «Пойди, сын твой здоров». Он поверил слову, которое сказал ему Иисус, и~пошёл. \textsubscript{51}~На дороге встретили его слуги его и~сказали: «Сын твой здоров». \textsubscript{52}~Он спросил у~них: «В котором часу стало ему легче?» Ему сказали: «Вчера в~седьмом часу горячка оставила его». \textsubscript{53}~Из этого отец узнал, что это был тот час, в~который Иисус сказал ему: «Сын твой здоров»,~--- и~уверовал сам и~весь дом его. \textsubscript{54}~Это второе чудо сотворил Иисус, возвратившись из Иудеи в~Галилею. 
\subsection*{Вопросы по наблюдению за текстом}
\begin{enumerate}
    \item Какие локации посетил Христос? 
    
    \myline
    
    \myline
    \item Кто обратился ко Христу за помощью? 
    
    \myline
    
    \myline
    \item Какое чудо совершил Христос и~как Он это сделал? 
    
    \myline
    
    \myline
\end{enumerate}

\subsection*{Вопросы по применению текста} 
\begin{enumerate}
    \item Поясните 44~стих: что, на ваш взгляд, хотел сказать Иисус? 
    
    \myline
    
    \myline
    \item Что доказывает чудо, которое совершил Христос? 
    
    \myline
    
    \myline
    \item Почему уверовал царедворец? 
    
    \myline
    
    \myline
    \item Как данное повествование было полезно для вас? Напишите несколько конкретных применений.
    
    \myline
    
    \myline
\end{enumerate}


% \newpage

\section{Евангелие от Иоанна 5:1-18. Исцеление расслабленного}

\textsubscript{1}~После этого был праздник иудейский, и~пришёл Иисус в~Иерусалим. \textsubscript{2}~Есть же в~Иерусалиме, у~Овечьих ворот, купальня, называемая по-еврейски Вифезда, при которой было пять крытых ходов. \textsubscript{3}~В~них лежало великое множество больных, слепых, хромых, иссохших, ожидающих движения воды, \textsubscript{4}~ибо ангел Господень по временам сходил в~купальню и~возмущал воду, и~кто первый входил в~нее по возмущении воды, тот выздоравливал, какой бы ни был одержим болезнью. \textsubscript{5}~Тут был человек, находившийся в~болезни тридцать восемь лет. \textsubscript{6}~Иисус, увидев его лежащего и~узнав, что он лежит уже долгое время, говорит ему: «Хочешь ли быть здоров?» \textsubscript{7}~Больной отвечал Ему: «Так, Господи, но не имею человека, который опустил бы меня в~купальню, когда возмутится вода; когда же я~прихожу, другой уже сходит прежде меня». \textsubscript{8}~Иисус говорит ему: «Встань, возьми постель твою и~ходи». \textsubscript{9}~И~он тотчас выздоровел, и~взял постель свою, и~пошёл. Было же это в~день субботний. \textsubscript{10}~Поэтому иудеи говорили исцеленному: «Сегодня суббота~--- не должно тебе брать постели». \textsubscript{11}~Он отвечал им: «Кто меня исцелил, Тот мне сказал: „Возьми постель твою и~ходи“». \textsubscript{12}~Его спросили: «Кто Тот Человек, Который сказал тебе: „Возьми постель твою и~ходи“?» \textsubscript{13}~Исцеленный же не знал, кто Он, ибо Иисус скрылся в~народе, бывшем на том месте. \textsubscript{14}~Потом Иисус встретил его в~храме и~сказал ему: «Вот, ты выздоровел; не греши больше, чтобы не случилось с~тобой чего худшего». \textsubscript{15}~Человек этот пошёл и~объявил иудеям, что исцеливший его есть Иисус. \textsubscript{16}~И~стали иудеи гнать Иисуса, и~искали убить Его за то, что Он делал такие дела в~субботу. \textsubscript{17}~Иисус же говорил им: «Отец Мой доныне делает, и~Я делаю». \textsubscript{18}~И~ещё более искали убить Его иудеи за то, что Он не только нарушал субботу, но и~Отцом Своим называл Бога, делая Себя равным Богу. 

\subsection*{Вопросы по наблюдению за текстом}
\begin{enumerate}
    \item В~какой город пришёл Иисус? 
    
    \myline
    
    \item Согласно данному отрывку, в~какой день недели Христос совершил чудо? 
    
    \myline
    \item Что хромой хотел от Иисуса буквально (см. стих~7)? 
    
    \myline
    
    \myline
    \item Как иудейские лидеры отреагировали на чудесное исцеление в~стихах 10~и~16? 
    
    \myline
    
    \myline
\end{enumerate}

\subsection*{Вопросы по применению текста} 
\begin{enumerate}
    \item Как вы думаете, на что хромой возлагал свое упование об исцелении? Почему он не мог исцелиться?
    
    \myline
    
    \myline
    \item Как вы понимаете предупреждение Иисуса этому человеку в~ст. 14? Поделитесь своими размышлениями.
    
    \myline
    
    \myline
    \item Данный отрывок показывает различную реакцию на дело Иисуса. Какую реакцию на дело Иисуса вы видите сегодня? Каким образом вы лично реагируете на Иисуса?
    
    \myline
    
    \myline
\end{enumerate}


% \newpage

\section{Евангелие от Иоанна 5:19-47. Отец и~Сын}

\textsubscript{19}~На это Иисус сказал: «Истинно, истинно говорю вам: Сын ничего не может творить Сам от Себя, если не увидит Отца творящего, ибо, что творит Он, то и~Сын творит также. \textsubscript{20}~Ибо Отец любит Сына и~показывает Ему всё, что творит Сам; и~покажет Ему дела больше этих, так что вы удивитесь. \textsubscript{21}~Ибо, как Отец воскрешает мертвых и~оживляет, так и~Сын оживляет кого хочет. \textsubscript{22}~Ибо Отец и~не судит никого, но весь суд отдал Сыну, \textsubscript{23}~дабы все чтили Сына, как чтут Отца. Кто не чтит Сына, тот не чтит и~Отца, пославшего Его. \textsubscript{24}~Истинно, истинно говорю вам: слушающий слово Моё и~верующий в~Пославшего Меня имеет жизнь вечную и~на суд не приходит, но перешёл от смерти в~жизнь. \textsubscript{25}~Истинно, истинно говорю вам: наступает время, и~настало уже, когда мертвые услышат глас Сына Божьего и, услышав, оживут. \textsubscript{26}~Ибо, как Отец имеет жизнь в~Самом Себе, так и~Сыну дал иметь жизнь в~Самом Себе. \textsubscript{27}~И~дал Ему власть производить и~суд, потому что Он есть Сын Человеческий. \textsubscript{28}~Не дивитесь этому, ибо наступает время, в~которое все, находящиеся в~могилах, услышат глас Сына Божьего; \textsubscript{29}~и~выйдут творившие добро~--- в~воскресение жизни, а~делавшие зло~--- в~воскресение осуждения. \textsubscript{30}~Я~ничего не могу творить Сам от Себя. Как слышу, так и~сужу, и~суд Мой праведен; ибо не ищу Моей воли, но воли пославшего Меня Отца. \textsubscript{31}~Если Я~свидетельствую Сам о~Себе, то свидетельство Моё не истинно. \textsubscript{32}~Есть Другой, свидетельствующий обо Мне; и~Я знаю, что истинно то свидетельство, которым Он свидетельствует обо Мне. \textsubscript{33}~Вы посылали к~Иоанну, и~он засвидетельствовал об истине. \textsubscript{34}~Впрочем, Я~не от человека принимаю свидетельство, но говорю это для того, чтобы вы спаслись. \textsubscript{35}~Он был светильник, горящий и~светящий; а~вы хотели малое время порадоваться при свете его. \textsubscript{36}~Я~же имею свидетельство больше Иоаннова: ибо дела, которые Отец дал Мне совершить, самые дела эти, Мною творимые, свидетельствуют обо Мне, что Отец послал Меня. \textsubscript{37}~И~пославший Меня Отец Сам засвидетельствовал обо Мне. А~вы ни голоса Его никогда не слышали, ни лица Его не видели; \textsubscript{38}~и~не имеете слова Его, пребывающего в~вас, потому что вы не веруете Тому, Которого Он послал. \textsubscript{39}~Исследуйте Писания, ибо вы думаете через них иметь жизнь вечную; а~они свидетельствуют обо Мне. \textsubscript{40}~Но вы не хотите прийти ко Мне, чтобы иметь жизнь. \textsubscript{41}~Не принимаю славы от людей, \textsubscript{42}~но знаю вас: вы не имеете в~себе любви к~Богу. \textsubscript{43}~Я~пришёл во имя Отца Моего~--- и~не принимаете Меня; а~если иной придет во имя свое, его примете. \textsubscript{44}~Как вы можете веровать, когда друг от друга принимаете славу, а~славы, которая от единого Бога, не ищете? \textsubscript{45}~Не думайте, что Я~буду обвинять вас пред Отцом,~--- есть на вас обвинитель Моисей, на которого вы уповаете. \textsubscript{46}~Ибо если бы вы верили Моисею, то поверили бы и~Мне, потому что он писал обо Мне. \textsubscript{47}~Если же его писаниям не верите, как поверите Моим словам?» 

\subsection*{Вопросы по наблюдению за текстом}
Прочитайте несколько раз отрывок внимательно и~отметьте, стараясь отвечать максимально по тексту Писания: 

\begin{enumerate}
    
    \item Как проявляется отношение Отца к~Сыну? 
    
    \myline
    
    \myline
    \item Как проявляется отношение Сына к~Отцу? 
    
    \myline
    
    \myline
    \item В~чём Их равенство? 
    
    \myline
    
    \myline
\end{enumerate}

\subsection*{Вопросы по применению текста} 
\begin{enumerate}
    \item Какие, на ваш взгляд, два ясных заявления делает Христос, определяющих Его как Божьего Сына? Как вы думаете, почему они важны?
    
    \myline
    
    \myline
    \item В~стихе 37~говорится: «И пославший Меня Отец Сам засвидетельствовал обо Мне». О~каком свидетельстве идет речь? 
    
    \myline
    
    \myline
    \item Прочитайте ещё раз 39~стих. Как вы думаете, что Христос хочет сказать своим оппонентам? 
    
    \myline
    
    \myline
    \item Насколько важно вам постоянно и~глубоко исследовать Писание? Укажите несколько причин.
    
    \myline
    
    \myline
\end{enumerate}

% \newpage

\section{Евангелие от Иоанна 6:1-15. Насыщение множества народа}

\textsubscript{1}~После этого пошёл Иисус на ту сторону моря Галилейского, в~окрестности Тивериады. \textsubscript{2}~За Ним последовало множество народа, потому что видели чудеса, которые Он творил над больными. \textsubscript{3}~Иисус взошёл на гору и~там сидел с~учениками Своими. \textsubscript{4}~Приближалась же Пасха, праздник иудейский. \textsubscript{5}~Иисус, возведя очи и~увидев, что множество народа идет к~Нему, говорит Филиппу: «Где нам купить хлебов, чтобы их накормить?» \textsubscript{6}~Говорил же это, испытывая его, ибо Сам знал, что хотел сделать. \textsubscript{7}~Филипп отвечал Ему: «Им на двести динариев не довольно будет хлеба, чтобы каждому из них досталось хоть немного». \textsubscript{8}~Один из учеников Его, Андрей, брат Симона Петра, говорит Ему: \textsubscript{9}~«Здесь есть у~одного мальчика пять хлебов ячменных и~две рыбки; но что это для такого множества?» \textsubscript{10}~Иисус сказал: «Велите им возлечь». Было же на том месте много травы. Итак, возлегли люди числом около пяти тысяч. \textsubscript{11}~Иисус, взяв хлебы и~воздав благодарение, раздал ученикам, а~ученики~--- возлежавшим, также и~рыбы, сколько кто хотел. \textsubscript{12}~И~когда насытились, то сказал ученикам Своим: «Соберите оставшиеся куски, чтобы ничего не пропало». \textsubscript{13}~И~собрали, и~наполнили двенадцать коробов кусками от пяти ячменных хлебов, оставшимися у~тех, которые ели. \textsubscript{14}~Тогда люди, видевшие чудо, сотворенное Иисусом, сказали: «Это истинно Тот Пророк, Которому должно прийти в~мир». \textsubscript{15}~Иисус же, узнав, что хотят прийти, взять Его и~сделать царем, опять удалился на гору один. 

\subsection*{Вопросы по наблюдению за текстом}
\begin{enumerate}
    \item В~какие окрестности пришёл Иисус? 
    
    \myline
    
    \item Почему за Ним следовало множество народа? 
    
    \myline
    
    \item Какая еда и~в~каком количестве была у~мальчика? 
    
    \myline
    
    \item Сколько коробов еды собрали ученики уже после того, как народ насытился? 
    
    \myline
\end{enumerate}

\subsection*{Вопросы по применению текста} 
\begin{enumerate}
    \item Как вы думаете, для чего евангелист Иоанн рассказывает про данное чудо Христа? Что он хочет достичь, рассказав о~нём? 
    
    \myline
    
    \myline
    \item Прокомментируйте короткий диалог Иисуса и~Филиппа в~стихах 5-7: была ли какая-нибудь проблема у~Филиппа, на ваш взгляд? В~чём мы можем быть подобны Филиппу?
    
    \myline
    
    \myline
    \item Что захотел сделать народ с~Иисусом после того, как Он совершил чудо? Почему народ так отреагировал? Поясните свой ответ. 
    
    \myline
    
    \myline
    \item Как чудеса показывали мощь Иисуса? Зачем Он их творил? 
    
    \myline
    
    \myline
\end{enumerate}


% \newpage

\section{Евангелие от Иоанна 6:16-24. Хождение по водам}

\textsubscript{16}~Когда же настал вечер, то ученики Его сошли к~морю \textsubscript{17}~и, войдя в~лодку, отправились на ту сторону моря, в~Капернаум. Становилось темно, а~Иисус не приходил к~ним. \textsubscript{18}~Дул сильный ветер, и~море волновалось. \textsubscript{19}~Проплыв около двадцати пяти или тридцати стадий, они увидели Иисуса, идущего по морю и~приближающегося к~лодке, и~испугались. \textsubscript{20}~Но Он сказал им: «Это Я; не бойтесь». \textsubscript{21}~Они хотели принять Его в~лодку; и~тотчас лодка пристала к~берегу, к~которому плыли. \textsubscript{22}~На другой день народ, стоявший по ту сторону моря, видел, что там, кроме одной лодки, в~которую вошли ученики Его, иной не было и~что Иисус не входил в~лодку с~учениками Своими, а~отплыли одни ученики Его. \textsubscript{23}~Между тем пришли из Тивериады другие лодки близко к~тому месту, где ели хлеб по благословении Господнем. \textsubscript{24}~Итак, когда народ увидел, что тут нет Иисуса, ни учеников Его, то вошли в~лодки и~приплыли в~Капернаум, ища Иисуса.

\subsection*{Вопросы по наблюдению за текстом}
\begin{enumerate}
    \item Куда отправились ученики Христа? 
    
    \myline
    \item Какое расстояние преодолели ученики до того момента, когда увидели Христа? 
    
    \myline
    
    \item Как отреагировали ученики на приближающегося к~ним Христа? 
    
    \myline
\end{enumerate}

\subsection*{Вопросы по применению текста} 
\begin{enumerate}
    \item Как вы думаете, почему Христос не сел в~лодку и~не отплыл с~учениками вместе? Почему Он отправил их одних? 
    
    \myline
    
    \myline
    \item Поразмышляйте, почему ученики испугались, когда увидели Христа? 
    
    \myline
    
    \myline
\end{enumerate}

% \newpage


\section{Евангелие от Иоанна 6:25-59. Дискуссия о~хлебе жизни}

\textsubscript{25}~И, найдя Его на той стороне моря, сказали Ему: «Равви! Когда Ты сюда пришёл?» \textsubscript{26}~Иисус сказал им в~ответ: «Истинно, истинно говорю вам: вы ищете Меня не потому, что видели чудеса, но потому, что ели хлеб и~насытились. \textsubscript{27}~Старайтесь не о~пище тленной, но о~пище, пребывающей в~жизнь вечную, которую даст вам Сын Человеческий, ибо на Нем положил печать Свою Отец, Бог». \textsubscript{28}~Итак, сказали Ему: «Что нам делать, чтобы творить дела Божии?» \textsubscript{29}~Иисус сказал им в~ответ: «Вот дело Божие: чтобы вы веровали в~Того, Кого Он послал». \textsubscript{30}~На это сказали Ему: «Какое же Ты дашь знамение, чтобы мы увидели и~поверили Тебе? Что Ты сделаешь? \textsubscript{31}~Отцы наши ели манну в~пустыне, как написано: „Хлеб с~неба дал им есть“ ». \textsubscript{32}~Иисус же сказал им: «Истинно, истинно говорю вам: не Моисей дал вам хлеб с~неба, а~Отец Мой дает вам истинный хлеб с~небес. \textsubscript{33}~Ибо хлеб Божий есть Тот, Который сходит с~небес и~дает жизнь миру». \textsubscript{34}~На это сказали Ему: «Господи! Подавай нам всегда такой хлеб». \textsubscript{35}~Иисус же сказал им: «Я~--- хлеб жизни; приходящий ко Мне не будет алкать, и~верующий в~Меня не будет жаждать никогда. \textsubscript{36}~Но Я~сказал вам, что вы и~видели Меня, и~не веруете. \textsubscript{37}~Все, что дает Мне Отец, ко Мне придет; и~приходящего ко Мне не изгоню вон, \textsubscript{38}~ибо Я~сошёл с~небес не для того, чтобы творить волю Мою, но волю пославшего Меня Отца. \textsubscript{39}~Воля же пославшего Меня Отца есть та, чтобы из того, что Он Мне дал, ничего не погубить, но всё то воскресить в~последний день. \textsubscript{40}~Воля Пославшего Меня есть та, чтобы всякий, видящий Сына и~верующий в~Него, имел жизнь вечную; и~Я воскрешу его в~последний день». \textsubscript{41}~Возроптали на Него иудеи за то, что Он сказал: «Я~--- хлеб, сошедший с~небес». \textsubscript{42}~И~говорили: «Не Иисус ли это, сын Иосифа, Которого отца и~мать мы знаем? Как же говорит Он: „Я сошёл с~небес“?» \textsubscript{43}~Иисус сказал им в~ответ: «Не ропщите между собой. \textsubscript{44}~Никто не может прийти ко Мне, если не привлечет его Отец, пославший Меня; и~Я воскрешу его в~последний день. \textsubscript{45}~У~пророков написано: „Иx будут все научены Богом“ . Всякий, слышавший от Отца и~научившийся, приходит ко Мне. \textsubscript{46}~Это не то, чтобы кто видел Отца, кроме Того, Кто есть от Бога,~--- Он видел Отца. \textsubscript{47}~Истинно, истинно говорю вам: верующий в~Меня имеет жизнь вечную. \textsubscript{48}~Я~— хлеб жизни. \textsubscript{49}~Отцы ваши ели манну в~пустыне и~умерли. \textsubscript{50}~Хлеб же, сходящий с~небес, таков, что едящий его не умрет. \textsubscript{51}~Я~— хлеб живой, сошедший с~небес; едящий хлеб этот будет жить вовек; хлеб же, который Я~дам, есть плоть Моя, которую Я~отдам за жизнь мира». \textsubscript{52}~Тогда иудеи стали спорить между собой, говоря: «Как Он может дать нам есть плоть Свою?» \textsubscript{53}~Иисус же сказал им: «Истинно, истинно говорю вам: если не будете есть плоть Сына Человеческого и~пить кровь Его, то не будете иметь в~себе жизни. \textsubscript{54}~Едящий Мою плоть и~пьющий Мою кровь имеет жизнь вечную, и~Я воскрешу его в~последний день. \textsubscript{55}~Ибо плоть Моя истинно есть пища и~кровь Моя истинно есть питье. \textsubscript{56}~Едящий Мою плоть и~пьющий Мою кровь пребывает во Мне, и~Я в~нём. \textsubscript{57}~Как послал Меня живой Отец и~Я живу Отцом, так и~едящий Меня жить будет Мной. \textsubscript{58}~Этот-то есть хлеб, сшедший с~небес. Не так, как отцы ваши ели манну и~умерли; едящий хлеб этот жить будет вовек». \textsubscript{59}~Это говорил Он в~синагоге, уча в~Капернауме. 

\subsection*{Вопросы по наблюдению за текстом}
\begin{enumerate}
    \item Где разворачивается данная беседа Христа? Ответьте текстом из Писания. 
    
    \myline
    
    \myline
    \item По какой причине народ искал Христа? 
    
    \myline
    
    \myline
    \item Сколько вопросов задали иудеи Христу? 
    
    \myline
    
    \myline
\end{enumerate}

\subsection*{Вопросы по применению текста} 
\begin{enumerate}
    \item В~чём разница между хлебом физическим и~хлебом духовным? 
    
    \myline
    
    \myline
    \item Подумайте, с~какой целью Христос приводит пример с~Моисеем? 
    
    \myline
    
    \myline
    \item На что указывает Христос, говоря о~Себе: «Я есмь хлеб жизни»? 
    
    \myline
    
    \myline
    \item Почему возроптали иудеи на Христа? 
    
    \myline
    
    \myline
    \item Является ли для вас Христос «хлебом жизни»?
    
    \myline
    
    \myline
\end{enumerate}


% \newpage

\section{Евангелие от Иоанна 6:60-71. Отношение учеников к~учению и~делам Христа}

 \textsubscript{60}~Многие из учеников Его, слыша то, говорили: «Какие странные слова! Кто может это слушать?» \textsubscript{61}~Но Иисус, зная в~Себе, что ученики Его ропщут на то, сказал им: «Это ли соблазняет вас? \textsubscript{62}~Что, если увидите Сына Человеческого, восходящего туда, где был прежде? \textsubscript{63}~Дух животворит; плоть не приносит никакой пользы. Слова, которые говорю Я~вам,~--- дух и~жизнь. \textsubscript{64}~Но есть из вас некоторые неверующие». Ибо Иисус от начала знал, кто неверующие и~кто предаст Его. \textsubscript{65}~И~сказал: «Для того-то и~говорил Я~вам, что никто не может прийти ко Мне, если то не дано будет ему от Отца Моего». \textsubscript{66}~С~этого времени многие из учеников Его отошли от Него и~уже не ходили с~Ним. \textsubscript{67}~Тогда Иисус сказал двенадцати: «Не хотите ли и~вы отойти?» \textsubscript{68}~Симон Петр отвечал Ему: «Господи! К~кому нам идти? Ты имеешь слова вечной жизни; \textsubscript{69}~и~мы уверовали и~познали, что Ты Христос, Сын Бога живого». \textsubscript{70}~Иисус отвечал им: «Не двенадцать ли вас избрал Я? Но один из вас дьявол». \textsubscript{71}~Это говорил Он об Иуде Симонове Искариоте, ибо этот хотел предать Его, будучи одним из двенадцати. 

\subsection*{Вопросы по наблюдению за текстом}
\begin{enumerate}
    \item В~каких стихах ясно видно подтверждение всезнания Христа? 
    
    \myline
    
    \item Кто из двенадцати отвечал Христу? 
    
    \myline
    
    \item Что мы узнаем об Иуде из данного отрывка? 
    
    \myline
    
    \myline
\end{enumerate}

\subsection*{Вопросы по применению текста} 
\begin{enumerate}
    \item Почему многие из учеников отошли и~больше не ходили со Христом? Постарайтесь максимально подробно ответить на вопрос. 
    
    \myline
    
    \myline
    \item Что Петр ответил Христу на вопрос, не хотят ли и~оставшиеся двенадцать отойти? Разъясните своими словами ответ Петра. 
    
    \myline
    
    \myline
\end{enumerate}


% \newpage

\section{Евангелие от Иоанна 7:1-9. Христос приходит из Галилеи в~Иерусалим}

 \textsubscript{1}~После этого Иисус ходил по Галилее, ибо по Иудее не хотел ходить, потому что иудеи искали убить Его. \textsubscript{2}~Приближался праздник иудейский~--- поставление кущей. \textsubscript{3}~Тогда братья Его сказали Ему: «Выйди отсюда и~пойди в~Иудею, чтобы и~ученики Твои видели дела, которые Ты делаешь. \textsubscript{4}~Ибо никто не делает чего-либо втайне, ища сам быть известным. Если Ты творишь такие дела, то яви Себя миру». \textsubscript{5}~Ибо и~братья Его не веровали в~Него. \textsubscript{6}~На это Иисус сказал им: «Мое время ещё не настало, а~для вас всегда время. \textsubscript{7}~Вас мир не может ненавидеть, а~Меня ненавидит, потому что Я~свидетельствую о~нём, что дела его злы. \textsubscript{8}~Вы пойдите на праздник этот, а~Я не пойду на этот праздник, потому что Моё время ещё не исполнилось». \textsubscript{9}~Сказав им это, остался в~Галилее.

\subsection*{Вопросы по наблюдению за текстом}
\begin{enumerate}
    \item В~какой местности ходил и~учил Христос? 
    
    \myline
    
    \item Согласно этому отрывку, какой иудейский праздник приближался? 
    
    \myline
    
    \item Кто вступил в~диалог со Христом? 
    
    \myline
    
\end{enumerate}

\subsection*{Вопросы по применению текста} 
\begin{enumerate}
    \item Почему Христос не хотел ходить по Иудее? 
    
    \myline
    
    \myline
    \item На ваш взгляд, какая основная проблема была у~братьев Христа? Поясните свой ответ максимально подробно. 
    
    \myline
    
    \myline
    \item Как вы думаете, почему мир ненавидит Христа? Прокомментируйте 7~стих. 
    
    \myline
    
    \myline
\end{enumerate}


% \newpage

\section{Евангелие от Иоанна 7:10-53. Учение Христа на празднике}

 \textsubscript{10}~Но когда пошли братья Его, тогда и~Он пошёл на праздник~--- не явно, а~как бы тайно. \textsubscript{11}~Иудеи же искали Его на празднике и~говорили: «Где Он?» \textsubscript{12}~И~много толков было о~Нем в~народе: одни говорили, что Он добр; а~другие говорили~--- нет, но обольщает народ. \textsubscript{13}~Впрочем, никто не говорил о~Нем явно, боясь иудеев. \textsubscript{14}~Но в~половине уже праздника вошёл Иисус в~храм и~учил. \textsubscript{15}~И~дивились иудеи, говоря: «Как Он знает Писания, не учившись?» \textsubscript{16}~Иисус, отвечая им, сказал: «Мое учение~--- не Моё, но Пославшего Меня; \textsubscript{17}~кто хочет творить волю Его, тот узнает об этом учении, от Бога ли оно, или Я~Сам от Себя говорю. \textsubscript{18}~Говорящий сам от себя ищет славы себе; а~Кто ищет славы Пославшему Его, Тот истинен и~нет неправды в~Нем. \textsubscript{19}~Не дал ли вам Моисей закон? И~никто из вас не поступает по закону. За что ищете убить Меня?» \textsubscript{20}~Народ сказал в~ответ: «Не бес ли в~Тебе? Кто ищет убить Тебя?» \textsubscript{21}~Иисус, продолжая речь, сказал им: «Одно дело сделал Я, и~все вы дивитесь. \textsubscript{22}~Моисей дал вам обрезание (хотя оно не от Моисея, но от отцов), и~в~субботу вы обрезываете человека. \textsubscript{23}~Если в~субботу принимает человек обрезание, чтобы не был нарушен закон Моисеев, на Меня ли негодуете за то, что Я~всего человека исцелил в~субботу? \textsubscript{24}~Не судите по наружности, но судите судом праведным». \textsubscript{25}~Тут некоторые из иерусалимлян говорили: «Не Тот ли это, Которого ищут убить? \textsubscript{26}~Вот, Он говорит явно, и~ничего не говорят Ему. Не удостоверились ли начальники, что Он подлинно Христос? \textsubscript{27}~Но мы знаем Его, откуда Он; когда же придет Христос, никто не будет знать, откуда Он». \textsubscript{28}~Тогда Иисус возгласил в~храме, уча и~говоря: «И знаете Меня, и~знаете, откуда Я; и~Я пришёл не Сам от Себя, но истинен Пославший Меня, Которого вы не знаете. \textsubscript{29}~Я~знаю Его, потому что Я~от Него, и~Он послал Меня». \textsubscript{30}~И~искали схватить Его, но никто не наложил на Него руку, потому что ещё не пришёл час Его. \textsubscript{31}~Многие же из народа уверовали в~Него и~говорили: «Когда придет Христос, неужели сотворит больше знамений, нежели сколько Сей сотворил?» \textsubscript{32}~Услышали фарисеи такие толки о~Нем в~народе, и~послали фарисеи и~первосвященники служителей схватить Его. \textsubscript{33}~Иисус же сказал им: «Ещё недолго быть Мне с~вами и~пойду к~Пославшему Меня; \textsubscript{34}~будете искать Меня и~не найдете; и~где буду Я, туда вы не можете прийти». \textsubscript{35}~При этом иудеи говорили между собой: «Куда Он хочет идти, так что мы не найдем Его? Не хочет ли Он идти в~эллинское рассеяние и~учить эллинов? \textsubscript{36}~Что значат эти слова, которые Он сказал: „Будете искать Меня и~не найдете; и~где буду Я, туда вы не можете прийти“?» \textsubscript{37}~В~последний же великий день праздника стоял Иисус и~возгласил, говоря: «Кто жаждет, иди ко Мне и~пей. \textsubscript{38}~Кто верует в~Меня, у~того, как сказано в~Писании, из чрева потекут реки воды живой». \textsubscript{39}~Это сказал Он о~Духе, Которого имели принять верующие в~Него, ибо ещё не было на них Духа Святого, потому что Иисус ещё не был прославлен. \textsubscript{40}~Многие из народа, услышав эти слова, говорили: «Он точно пророк». \textsubscript{41}~Другие говорили: «Это Христос». А~иные говорили: «Разве из Галилеи Христос придет? \textsubscript{42}~Не сказано ли в~Писании, что Христос придет от семени Давидова и~из Вифлеема, из того места, откуда был Давид?» \textsubscript{43}~Итак, произошла о~Нем распря в~народе. \textsubscript{44}~Некоторые из них хотели схватить Его, но никто не наложил на Него рук. \textsubscript{45}~Итак, служители возвратились к~первосвященникам и~фарисеям, и~эти сказали им: «Почему вы не привели Его?» \textsubscript{46}~Служители отвечали: «Никогда человек не говорил так, как Этот Человек». \textsubscript{47}~Фарисеи сказали им: «Неужели и~вы прельстились? \textsubscript{48}~Уверовал ли в~Него кто из начальников или из фарисеев? \textsubscript{49}~Но этот народ невежда в~законе, проклят он». \textsubscript{50}~Никодим, приходивший к~Нему ночью, будучи один из них, говорит им: \textsubscript{51}~«Судит ли закон наш человека, если прежде не выслушают его и~не узнают, что он делает?» \textsubscript{52}~На это сказали ему: «И ты не из Галилеи ли? Рассмотри и~увидишь, что из Галилеи не приходит пророк». \textsubscript{53}~И~разошлись все по домам. 

\subsection*{Вопросы по наблюдению за текстом}
\begin{enumerate}
    \item Читая начало повествования, отметьте, где Христос учил народ? 
    
    \myline
    
    \item Чему дивились иудеи? Почему многие из народа не могли говорить о~Христе? Ответьте непосредственно текстами из Писания. 
    
    \myline
    
    \myline
    \item Прочитайте Исайя 55:1-3 в~свете данного отрывка из Евангелия от Иоанна. Что Иисус объявляет о~Себе, намекая на эти стихи? 
    
    \myline
    
    \myline
    \item Что говорили люди о~Христе (см. 40-44 стихи)? Из-за чего произошла распря в~народе? 
    
    \myline
    
    \myline
\end{enumerate}

\subsection*{Вопросы по применению текста} 
\begin{enumerate}
    \item Прокомментируйте стихи 16, 17, дайте развернутый комментарий.
    
    \myline
    
    \myline
    \item Что значит прийти к~Иисусу и~пить (см. 37, 38~стихи)? Как Иисус удовлетворил вашу жажду? 
    
    \myline
    
    \myline
    \item Что такое «живая» вода в~данном тексте? 
    
    \myline
    
    \myline
\end{enumerate}

% \newpage

\section{Евангелие от Иоанна 8:1-11. Женщина, взятая в~прелюбодеянии}

\textsubscript{1}~Иисус же пошёл на гору Елеонскую. \textsubscript{2}~А~утром опять пришёл в~храм, и~весь народ шёл к~Нему. Он сел и~учил их. \textsubscript{3}~Тут книжники и~фарисеи привели к~Нему женщину, взятую в~прелюбодеянии, и, поставив её посередине, \textsubscript{4}~сказали Ему: «Учитель! Эта женщина взята в~прелюбодеянии, \textsubscript{5}~а~Моисей в~законе заповедал нам побивать таких камнями. Ты что скажешь?» \textsubscript{6}~Говорили же это, искушая Его, чтобы найти что-нибудь к~обвинению Его. Но Иисус, наклонившись низко, писал перстом на земле, не обращая на них внимания. \textsubscript{7}~Когда же продолжали спрашивать Его, Он, восклонившись, сказал им: «Кто из вас без греха, первый брось в~нее камень». \textsubscript{8}~И~опять, наклонившись низко, писал на земле. \textsubscript{9}~Они же, услышав это и~будучи обличаемы совестью, стали уходить один за другим, начиная от старших до последних; и~остался один Иисус и~женщина, стоящая посередине. \textsubscript{10}~Иисус, восклонившись и~не видя никого, кроме женщины, сказал ей: «Женщина! Где твои обвинители? Никто не осудил тебя?» \textsubscript{11}~Она отвечала: «Никто, Господи». Иисус сказал ей: «И Я~не осуждаю тебя; иди и~впредь не греши». 

\subsection*{Вопросы по наблюдению за текстом}
\begin{enumerate}
    \item Куда утром пришёл Христос, и~что Он начал делать? 
    
    \myline
    
    \myline
    \item Кого привели ко Христу? И~в~чём обвиняли этого человека? 
    
    \myline
    
    \myline
    \item На что опирались книжники и~фарисеи, выдвигая свое обвинение? На каком приговоре они настаивали? 
    
    \myline
    
    \myline
\end{enumerate}

\subsection*{Вопросы по применению текста} 
\begin{enumerate}
    \item Как реагировал Христос на действия фарисеев и~книжников? Что его реакция говорит о~Его характере? 
    
    \myline
    
    \myline
    \item Почему книжники и~фарисеи (и остальной народ) были обличаемы своей совестью? Поясните свой ответ, ссылаясь на текст из этой истории. 
    
    \myline
    
    \myline
    \item Прокомментируйте 11~стих: к~чему призвал Христос и~почему? Постарайтесь максимально развернуто дать ответ. 
    
    \myline
    
    \myline
\end{enumerate}


% \newpage

\section{Евангелие от Иоанна 8:12-59. Христос~--- свет миру}

\textsubscript{12}~Опять говорил Иисус к~народу и~сказал им: «Я~--- свет миру; кто последует за Мною, тот не будет ходить во тьме, но будет иметь свет жизни». \textsubscript{13}~Тогда фарисеи сказали Ему: «Ты Сам о~Себе свидетельствуешь, свидетельство Твоё не истинно». \textsubscript{14}~Иисус сказал им в~ответ: «Если Я~и~Сам о~Себе свидетельствую, свидетельство Моё истинно; потому что Я~знаю, откуда пришёл и~куда иду; а~вы не знаете, откуда Я~и~куда иду. \textsubscript{15}~Вы судите по плоти; Я~не сужу никого. \textsubscript{16}~А~если и~сужу Я, то суд Мой истинен, потому что Я~не один, но Я~и~Отец, пославший Меня. \textsubscript{17}~А~и~в~законе вашем написано, что свидетельство двух человек истинно. \textsubscript{18}~Я~Сам свидетельствую о~Себе, и~свидетельствует обо Мне Отец, пославший Меня». \textsubscript{19}~Тогда сказали Ему: «Где Твой Отец?» Иисус отвечал: «Вы не знаете ни Меня, ни Отца Моего; если бы вы знали Меня, то знали бы и~Отца Моего». \textsubscript{20}~Эти слова говорил Иисус у~сокровищницы, когда учил в~храме; и~никто не взял Его, потому что ещё не пришёл час Его. \textsubscript{21}~Опять сказал им Иисус: «Я отхожу, и~будете искать Меня, и~умрете во грехе вашем. Куда Я~иду, туда вы не можете прийти». \textsubscript{22}~Тут иудеи говорили: «Неужели Он убьет Сам Себя, что говорит: „Куда Я~иду, вы не можете прийти“?» \textsubscript{23}~Он сказал им: «Вы от нижних, Я~от вышних; вы от мира сего, Я~не от сего мира. \textsubscript{24}~Потому Я~и~сказал вам, что вы умрете во грехах ваших; ибо если не уверуете, что это Я, то умрете во грехах ваших». \textsubscript{25}~Тогда сказали Ему: «Кто же Ты?» Иисус сказал им: «От начала Сущий, как и~говорю вам. \textsubscript{26}~Много имею говорить и~судить о~вас; но Пославший Меня истинен, и~что Я~слышал от Него, то и~говорю миру». \textsubscript{27}~Не поняли, что Он говорил им об Отце. \textsubscript{28}~Итак, Иисус сказал им: «Когда вознесете Сына Человеческого, тогда узнаете, что это Я~и~что ничего не делаю от Себя, но как научил Меня Отец Мой, так и~говорю. \textsubscript{29}~Пославший Меня~--- со Мной; Отец не оставил Меня одного, ибо Я~всегда делаю то, что Ему угодно». \textsubscript{30}~Когда Он говорил это, многие уверовали в~Него. \textsubscript{31}~Тогда сказал Иисус к~уверовавшим в~Него иудеям: «Если пребудете в~слове Моем, то вы истинно Мои ученики, \textsubscript{32}~и~познаете истину, и~истина сделает вас свободными». \textsubscript{33}~Ему отвечали: «Мы семя Авраамово и~не были рабами никому никогда; как же Ты говоришь: „Сделаетесь свободными“?» \textsubscript{34}~Иисус отвечал им: «Истинно, истинно говорю вам: всякий, делающий грех, раб греха. \textsubscript{35}~Но раб не пребывает в~доме вечно; сын пребывает вечно. \textsubscript{36}~Итак, если Сын освободит вас, то истинно свободны будете. \textsubscript{37}~Знаю, что вы семя Авраамово; однако ищете убить Меня, потому что слово Моё не вмещается в~вас. \textsubscript{38}~Я~говорю то, что видел у~Отца Моего; а~вы делаете то, что видели у~отца вашего». \textsubscript{39}~Сказали Ему в~ответ: «Отец наш~--- Авраам». Иисус сказал им: «Если бы вы были дети Авраамовы, то дела Авраама делали бы. \textsubscript{40}~А~теперь ищете убить Меня, Человека, сказавшего вам истину, которую слышал от Бога. Авраам этого не делал. \textsubscript{41}~Вы делаете дела отца вашего». На это сказали Ему: «Мы не от любодеяния рождены; одного Отца имеем~--- Бога». \textsubscript{42}~Иисус сказал им: «Если бы Бог был Отцом вашим, то вы любили бы Меня, потому что Я~от Бога исшёл и~пришёл; ибо Я~не Сам от Себя пришёл, но Он послал Меня. \textsubscript{43}~Почему вы не понимаете речи Моей? Потому что не можете слышать слова Моего. \textsubscript{44}~Ваш отец~--- дьявол, и~вы хотите исполнять похоти отца вашего. Он был человекоубийца от начала и~не устоял в~истине, ибо нет в~нём истины. Когда говорит он ложь, говорит свое, ибо он лжец и~отец лжи. \textsubscript{45}~А~как Я~истину говорю, то не верите Мне. \textsubscript{46}~Кто из вас обличит Меня в~неправде? Если же Я~говорю истину, почему вы не верите Мне? \textsubscript{47}~Кто от Бога, тот слушает слова Божии. Вы потому не слушаете, что вы не от Бога». \textsubscript{48}~На это иудеи отвечали и~сказали Ему: «Не правду ли мы говорим, что Ты самарянин и~что бес в~Тебе?» \textsubscript{49}~Иисус отвечал: «Во Мне беса нет, но Я~чту Отца Моего, а~вы бесчестите Меня. \textsubscript{50}~Впрочем, Я~не ищу Моей славы. Есть Ищущий и~Судящий. \textsubscript{51}~Истинно, истинно говорю вам: кто соблюдет слово Моё, тот не увидит смерти вовек». \textsubscript{52}~Иудеи сказали Ему: «Теперь узнали мы, что бес в~Тебе. Авраам умер и~пророки, а~Ты говоришь: „Кто соблюдет слово Моё, тот не вкусит смерти вовек“. \textsubscript{53}~Неужели Ты больше отца нашего Авраама, который умер? И~пророки умерли. Кем Ты Себя делаешь?» \textsubscript{54}~Иисус отвечал: «Если Я~Сам Себя славлю, то слава Моя~--- ничто. Меня прославляет Отец Мой, о~Котором вы говорите, что Он Бог ваш. \textsubscript{55}~И~вы не познали Его, а~Я знаю Его; и~если скажу, что не знаю Его, то буду подобный вам лжец. Но Я~знаю Его и~соблюдаю слово Его. \textsubscript{56}~Авраам, отец ваш, рад был увидеть день Мой~--- и~увидел, и~возрадовался». \textsubscript{57}~На это сказали Ему иудеи: «Тебе нет ещё пятидесяти лет~--- и~Ты видел Авраама?» \textsubscript{58}~Иисус сказал им: «Истинно, истинно говорю вам: прежде, нежели был Авраам, Я~есмь». \textsubscript{59}~Тогда взяли камни, чтобы бросить в~Него; но Иисус скрылся и~вышел из храма, пройдя посреди них, и~пошёл далее. 

\subsection*{Вопросы по наблюдению за текстом}
\begin{enumerate}
    \item Как называет Себя Христос в~12 стихе? 
    
    \myline
    
    \item Кто вновь является оппонентом Христа в~данном разделе повествования? 
    
    \myline
    
    \myline
    \item Какой конфликт видел Иисус между родословием Своих иудейских оппонентов и~их действиями? Кто их настоящий отец? 
    
    \myline
    
    \myline
    \item Какая характеристика дается дьяволу в~стихе 44? 
    
    \myline
    
    \myline
    \item Что говорит Христос о~Себе в~58 стихе? 
    
    \myline
    
    \myline
\end{enumerate}

\subsection*{Вопросы по применению текста} 
\begin{enumerate}
    \item Поразмышляйте над 12~стихом и~напишите, как практически утверждение Христа относится к~вам?
    
    \myline
    
    \myline
    \item Кто, согласно 31~стиху, является учеником Христа? Можете ли вы назвать себя учеником Христа? Старайтесь развернуто отвечать на данные вопросы. 
    
    \myline
    
    \myline
    \item Что означает слово «истина» в~стихе 32? Что в~этом же стихе означает слово «свобода»? От какого рабства Иисус освободит Своих слушателей, если они придут познать истину? 
    
    \myline
    
    \myline
    \item Каким образом отношения с~Иисусом освобождают вас? 
    
    \myline
    
    \myline
    \item Что Иисус заявляет в~стихе 46? Как это относится к~тому, что Иисус сказал об истине и~чьем-то отце? Что данная концепция значит для вас?
    
    \myline
    
    \myline
\end{enumerate}


% \newpage

\section{Евангелие от Иоанна 9:1-41. Христос исцеляет слепорожденного}

\textsubscript{1}~И, проходя, увидел человека, слепого от рождения. \textsubscript{2}~Ученики Его спросили у~Него: «Равви! Кто согрешил: он или родители его, что родился слепым?» \textsubscript{3}~Иисус отвечал: «Не согрешил ни он, ни родители его, но это для того, чтобы на нём явились дела Божии. \textsubscript{4}~Мне должно делать дела Пославшего Меня, доколе день; приходит ночь, когда никто не может делать. \textsubscript{5}~Доколе Я~в~мире, Я~— свет миру». \textsubscript{6}~Сказав это, Он плюнул на землю, сделал брение из плюновения и~помазал брением глаза слепому \textsubscript{7}~и~сказал ему: «Пойди, умойся в~купальне Силоам (что означает „посланный“)». Он пошёл, и~умылся, и~пришёл зрячим. \textsubscript{8}~Тут соседи и~видевшие прежде, что он был слеп, говорили: «Не тот ли это, который сидел и~просил милостыню?» \textsubscript{9}~Иные говорили: «Это он», а~иные: «Похож на него». Он же говорил: «Это я». \textsubscript{10}~Тогда спрашивали у~него: «Как открылись у~тебя глаза?» \textsubscript{11}~Он сказал в~ответ: «Человек, называемый Иисус, сделал брение, помазал глаза мои и~сказал мне: „Пойди в~купальню Силоам и~умойся“. Я~пошёл, умылся и~прозрел». \textsubscript{12}~Тогда сказали ему: «Где Он?» Он отвечал: «Не знаю». \textsubscript{13}~Повели этого бывшего слепца к~фарисеям. \textsubscript{14}~А~была суббота, когда Иисус сделал брение и~отверз ему очи. \textsubscript{15}~Спросили его также и~фарисеи, как он прозрел. Он сказал им: «Брение положил Он на мои глаза, и~я умылся, и~вижу». \textsubscript{16}~Тогда некоторые из фарисеев говорили: «Не от Бога Этот Человек, потому что не хранит субботы». Другие говорили: «Как может человек грешный творить такие чудеса?» И~была между ними распря. \textsubscript{17}~Опять говорят слепому: «Ты что скажешь о~Нем, потому что Он отверз тебе очи?» Он сказал: «Это пророк». \textsubscript{18}~Тогда иудеи не поверили, что он был слеп и~прозрел, доколе не призвали родителей этого прозревшего \textsubscript{19}~и~спросили их: «Это ли сын ваш, о~котором вы говорите, что родился слепым? Как же он теперь видит?» \textsubscript{20}~Родители его сказали им в~ответ: «Мы знаем, что это сын наш и~что он родился слепым, \textsubscript{21}~а~как теперь видит, не знаем, или кто отверз ему очи, мы не знаем. Сам в~совершенных летах, самого спросите; пусть сам о~себе скажет». \textsubscript{22}~Так отвечали родители его, потому что боялись иудеев; ибо иудеи сговорились уже, чтобы, кто признает Его за Христа, того отлучать от синагоги. \textsubscript{23}~Поэтому-то родители его и~сказали: «Он в~совершенных летах, самого спросите». \textsubscript{24}~Итак, вторично призвали человека, который был слеп, и~сказали ему: «Воздай славу Богу; мы знаем, что Человек Тот грешник». \textsubscript{25}~Он сказал им в~ответ: «Грешник ли Он, не знаю; одно знаю, что я~был слеп, а~теперь вижу». \textsubscript{26}~Снова спросили его: «Что сделал Он с~тобой? Как отверз твои очи?» \textsubscript{27}~Отвечал им: «Я уже сказал вам, и~вы не слушали. Что ещё хотите слышать? Или и~вы хотите сделаться Его учениками?» \textsubscript{28}~Они же укорили его и~сказали: «Ты ученик Его, а~мы Моисеевы ученики. \textsubscript{29}~Мы знаем, что с~Моисеем говорил Бог; Сего же не знаем, откуда Он». \textsubscript{30}~Человек прозревший сказал им в~ответ: «Это и~удивительно, что вы не знаете, откуда Он, а~Он отверз мне очи. \textsubscript{31}~Но мы знаем, что грешников Бог не слушает; но кто чтит Бога и~творит волю Его, того слушает. \textsubscript{32}~От века не слыхано, чтобы кто отверз очи слепорожденному. \textsubscript{33}~Если бы Он не был от Бога, не мог бы творить ничего». \textsubscript{34}~Сказали ему в~ответ: «Во грехах ты весь родился, и~ты ли нас учишь?» И~выгнали его вон. \textsubscript{35}~Иисус, услышав, что выгнали его вон, и~найдя его, сказал ему: «Ты веруешь ли в~Сына Божьего?» \textsubscript{36}~Он отвечал и~сказал: «А кто Он, Господи, чтобы мне веровать в~Него?» \textsubscript{37}~Иисус сказал ему: «И видел ты Его, и~Он говорит с~тобой». \textsubscript{38}~Он же сказал: «Верую, Господи!» И~поклонился Ему. \textsubscript{39}~И~сказал Иисус: «На суд пришёл Я~в~мир этот, чтобы невидящие видели, а~видящие стали слепы». \textsubscript{40}~Услышав это, некоторые из фарисеев, бывших с~Ним, сказали Ему: «Неужели и~мы слепы?» \textsubscript{41}~Иисус сказал им: «Если бы вы были слепы, то не имели бы на себе греха; но так как вы говорите, что видите, то грех остается на вас. 

\subsection*{Вопросы по наблюдению за текстом}
\begin{enumerate}
    \item Как Христос называет Себя в~5 стихе? 
    
    \myline
    
    \item Что может быть причиной рождения слепого младенца? Ответьте, ссылаясь на текст из этого отрывка. 
    
    \myline
    
    \myline
    \item Каким способом Христос исцелил слепорожденного? 
    
    \myline
    
    \myline
    \item К~кому повели слепорожденного после исцеления? 
    
    \myline
    
    \myline
    \item Чего боялись родители слепорожденного, когда отвечали на вопросы фарисеев? 
    
    \myline
    
    \myline
\end{enumerate}

\subsection*{Вопросы по применению текста} 
\begin{enumerate}
    \item Что, на ваш взгляд, Христос хочет сказать словами «Я свет миру»? Постарайтесь дать развернутый ответ.
    
    \myline
    
    \myline
    \item Согласно данному тексту, как бы вы охарактеризовали реакцию фарисеев на исцеление, которое совершил Христос? 
    
    \myline
    
    \myline
    \item На ваш взгляд, какая проблема была у~фарисеев? Свой ответ аргументируйте, ссылаясь на тексты из данного отрывка. 
    
    \myline
    
    \myline
    \item Поразмышляйте и~напишите своими словами, в~чём смысл 39-41 стихов?
    
    \myline
    
    \myline
\end{enumerate}

% \newpage

\section{Евангелие от Иоанна 10:1-21. Христос~--- пастырь}

\textsubscript{1}~Истинно, истинно говорю вам: кто не дверью входит во двор овечий, но перелезает иным путем, тот вор и~разбойник; \textsubscript{2}~а~входящий дверью~--- пастырь овцам. \textsubscript{3}~Ему придверник отворяет, и~овцы слушаются голоса его, и~он зовет своих овец по имени и~выводит их. \textsubscript{4}~И~когда выведет своих овец, идет перед ними; а~овцы за ним идут, потому что знают голос его. \textsubscript{5}~За чужим же не идут, но бегут от него, потому что не знают чужого голоса». \textsubscript{6}~Эту притчу сказал им Иисус, но они не поняли, что такое Он говорил им. \textsubscript{7}~Итак, Иисус сказал им снова: «Истинно, истинно говорю вам, что Я~— дверь овцам. \textsubscript{8}~Все, сколько их ни приходило предо Мной,~--- воры и~разбойники; но овцы не послушали их. \textsubscript{9}~Я~— дверь: кто войдет Мною, тот спасется; и~войдет, и~выйдет, и~пажить найдет. \textsubscript{10}~Вор приходит только для того, чтобы украсть, убить и~погубить. Я~пришёл для того, чтобы имели жизнь, и~имели с~избытком. \textsubscript{11}~Я~— Пастырь добрый. Пастырь добрый полагает жизнь свою за овец. \textsubscript{12}~А~наемник, не пастырь, которому овцы не свои, видит приходящего волка, и~оставляет овец, и~бежит; и~волк расхищает овец и~разгоняет их. \textsubscript{13}~А~наемник бежит, потому что наемник и~не заботится об овцах. \textsubscript{14}~Я~— Пастырь добрый и~знаю Моих, и~Мои знают Меня. \textsubscript{15}~Как Отец знает Меня, так и~Я знаю Отца; и~жизнь Мою полагаю за овец. \textsubscript{16}~Есть у~Меня и~другие овцы, которые не этого двора, и~тех надлежит Мне привести: и~они услышат голос Мой, и~будет одно стадо и~один Пастырь. \textsubscript{17}~Потому любит Меня Отец, что Я~отдаю жизнь Мою, чтобы опять принять её. \textsubscript{18}~Никто не отнимает её у~Меня, но Я~Сам отдаю её. Имею власть отдать её и~власть имею опять принять её. Эту заповедь получил Я~от Отца Моего». \textsubscript{19}~От этих слов опять произошла между иудеями распря. \textsubscript{20}~Многие из них говорили: «Он одержим бесом и~безумствует. Что слушаете Его?» \textsubscript{21}~Другие говорили: «Это слова не бесноватого. Может ли бес отверзать очи слепым?» 

\subsection*{Вопросы по наблюдению за текстом}
\begin{enumerate}
    \item Как узнать: кто пастырь, а~кто наёмник? Ответьте, ссылаясь на текст из этого отрывка. 
    
    \myline
    
    \myline
    \item Как отреагировали иудеи на слова Христа? Что они говорили о~Христе? 
    
    \myline
    
    \myline
    \item Что вы узнаете о~взаимоотношениях Христа и~Отца из 17~и~18 стихов? 
    
    \myline
    
    \myline
\end{enumerate}

\subsection*{Вопросы по применению текста} 
\begin{enumerate}
    \item Почему именно овец Иисус брал как пример для пояснения? Кто Он такой? 
    
    \myline
    
    \myline
    \item Что хотел сказать Иисус, называя Себя дверью и~добрым Пастырем? 
    
    \myline
    
    \myline
    \item Как вы думаете, о~каких овцах говорит Христос в~16 стихе? 
    
    \myline
    
    \myline
    \item В~каких взаимоотношениях я~лично с~пастырем-Христом? 
    
    \myline
    
    \myline
\end{enumerate}

% \newpage

\section{Евангелие от Иоанна 10:22-42. Разговор на Празднике обновления}

\textsubscript{22}~Настал же тогда в~Иерусалиме праздник обновления, и~была зима. \textsubscript{23}~И~ходил Иисус в~храме, в~притворе Соломоновом. \textsubscript{24}~Тут иудеи обступили Его и~говорили Ему: «Долго ли Тебе держать нас в~недоумении? Если Ты Христос, скажи нам прямо». \textsubscript{25}~Иисус отвечал им: «Я сказал вам, и~не верите; дела, которые творю Я~во имя Отца Моего, они свидетельствуют обо Мне. \textsubscript{26}~Но вы не верите, ибо вы не из овец Моих, как Я~сказал вам. \textsubscript{27}~Овцы Мои слушаются голоса Моего, и~Я знаю их, и~они идут за Мной. \textsubscript{28}~И~Я даю им жизнь вечную~--- и~не погибнут вовек, и~никто не похитит их из руки Моей. \textsubscript{29}~Отец Мой, Который дал Мне их, больше всех; и~никто не может похитить их из руки Отца Моего. \textsubscript{30}~Я~и~Отец~--- одно». \textsubscript{31}~Тут опять иудеи схватили камни, чтобы побить Его. \textsubscript{32}~Иисус отвечал им: «Много добрых дел показал Я~вам от Отца Моего; за которое из них хотите побить Меня камнями?» \textsubscript{33}~Иудеи сказали Ему в~ответ: «Не за доброе дело хотим побить Тебя камнями, но за богохульство и~за то, что Ты, будучи человеком, делаешь Себя Богом». \textsubscript{34}~Иисус отвечал им: «Не написано ли в~законе вашем: „Я сказал: Вы боги“? \textsubscript{35}~Если Он назвал богами тех, к~которым было слово Божие,~--- и~не может нарушиться Писание,~--- \textsubscript{36}~Тому ли, Которого Отец освятил и~послал в~мир, вы говорите: „Богохульствуешь“, потому что Я~сказал: „Я Сын Божий“? \textsubscript{37}~Если Я~не творю дел Отца Моего, не верьте Мне; \textsubscript{38}~а~если творю, то, если не верите Мне, верьте делам Моим, чтобы узнать и~поверить, что Отец во Мне и~Я в~Нем». \textsubscript{39}~Тогда опять искали схватить Его, но Он уклонился от рук их. \textsubscript{40}~И~пошёл опять за Иордан, на то место, где прежде крестил Иоанн, и~остался там. \textsubscript{41}~Многие пришли к~Нему и~говорили, что Иоанн не сотворил никакого чуда, но всё, что сказал Иоанн о~Нём, было истинно. \textsubscript{42}~И~многие там уверовали в~Него. 

\subsection*{Вопросы по наблюдению за текстом}
\begin{enumerate}
    \item В~каком городе был Христос? Где развернулась беседа с~иудеями, в~каком месте? 
    
    \myline
    
    \myline
    \item Какой праздник упоминает евангелист? 
    
    \myline
    
    \myline
    \item Какой вопрос задали иудеи Христу? 
    
    \myline
    
    \myline
    \item Что опять иудеи хотели совершить со Христом? 
    
    \myline
    
    \myline
    \item Куда Христос ушёл после диалога с~иудеями? 
    
    \myline
    
    \myline
\end{enumerate}

\subsection*{Вопросы по применению текста} 
\begin{enumerate}
    \item Прочитайте ещё раз внимательно стихи 27--29. О~чём, на ваш взгляд, говорят эти стихи? Какую истину Христос хочет донести через эти слова? И~как эта истина относится к~вам?
    
    \myline
    
    \myline

    \myline    
    \item Что 30~стих говорит вам о~Христе? 
    
    \myline
    
    \myline
    \item Как вы думаете, почему иудеи хотели убить Христа?
    
    \myline
    
    \myline
\end{enumerate}


% \newpage

\section{Евангелие от Иоанна 11:1-44. Христос~--- победитель смерти}

 \textsubscript{1}~Был болен некто Лазарь из Вифании, из селения, где жили Мария и~Марфа, сестра её. \textsubscript{2}~Мария же, брат которой Лазарь был болен, была та, которая помазала Господа миром и~отерла ноги Его волосами своими. \textsubscript{3}~Сестры послали сказать Ему: «Господи! Вот, кого Ты любишь, болен». \textsubscript{4}~Иисус, услышав то, сказал: «Эта болезнь не к~смерти, но к~славе Божией, да прославится через нее Сын Божий». \textsubscript{5}~Иисус же любил Марфу, и~сестру её, и~Лазаря. \textsubscript{6}~Когда же услышал, что он болен, то пробыл два дня на том месте, где находился. \textsubscript{7}~После этого сказал ученикам: «Пойдем опять в~Иудею». \textsubscript{8}~Ученики сказали Ему: «Равви! Давно ли иудеи искали побить Тебя камнями, и~Ты опять идешь туда?» \textsubscript{9}~Иисус отвечал: «Не двенадцать ли часов в~дне? Кто ходит днем, тот не спотыкается, потому что видит свет мира сего; \textsubscript{10}~а~кто ходит ночью, спотыкается, потому что нет света с~ним». \textsubscript{11}~Сказав это, говорит им потом: «Лазарь, друг наш, уснул; но Я~иду разбудить его». \textsubscript{12}~Ученики Его сказали: «Господи! Если уснул, то выздоровеет». \textsubscript{13}~Иисус говорил о~смерти его, а~они думали, что Он говорит о~сне обыкновенном. \textsubscript{14}~Тогда Иисус сказал им прямо: «Лазарь умер; \textsubscript{15}~и~радуюсь за вас, что Меня не было там, дабы вы уверовали; но пойдем к~нему». \textsubscript{16}~Тогда Фома, иначе называемый Близнец, сказал ученикам: «Пойдем и~мы умрем с~Ним». \textsubscript{17}~Иисус, придя, нашёл, что он уже четыре дня в~гробнице. \textsubscript{18}~Вифания же была близ Иерусалима, стадиях в~пятнадцати; \textsubscript{19}~и~многие из иудеев пришли к~Марфе и~Марии утешать их в~печали о~брате их. \textsubscript{20}~Марфа, услышав, что идет Иисус, пошла навстречу Ему; Мария же сидела дома. \textsubscript{21}~Тогда Марфа сказала Иисусу: «Господи! Если бы Ты был здесь, не умер бы брат мой. \textsubscript{22}~Но и~теперь знаю: всё, чего Ты попросишь у~Бога, даст Тебе Бог». \textsubscript{23}~Иисус говорит ей: «Воскреснет брат твой». \textsubscript{24}~Марфа сказала Ему: «Знаю, что воскреснет в~воскресение, в~последний день». \textsubscript{25}~Иисус сказал ей: «Я~--- воскресение и~жизнь; верующий в~Меня, если и~умрет, оживет. \textsubscript{26}~И~всякий, живущий и~верующий в~Меня, не умрет вовек. Веришь ли этому?» \textsubscript{27}~Она говорит Ему: «Так, Господи! Я~верую, что Ты Христос, Сын Божий, грядущий в~мир». \textsubscript{28}~Сказав это, пошла и~позвала тайно Марию, сестру свою, говоря: «Учитель здесь и~зовет тебя». \textsubscript{29}~Она, как только услышала, поспешно встала и~пошла к~Нему. \textsubscript{30}~Иисус ещё не входил в~селение, но был на том месте, где встретила Его Марфа. \textsubscript{31}~Иудеи, которые были с~ней в~доме и~утешали её, видя, что Мария поспешно встала и~вышла, пошли за ней, полагая, что она пошла к~гробнице~--- плакать там. \textsubscript{32}~Мария же, придя туда, где был Иисус, и~увидев Его, пала к~ногам Его и~сказала Ему: «Господи! Если бы Ты был здесь, не умер бы брат мой». \textsubscript{33}~Иисус, когда увидел её плачущую и~пришедших с~ней иудеев плачущих, Сам восскорбел духом, и~взволновался, \textsubscript{34}~и~сказал: «Где вы положили его?» Говорят Ему: «Господи! Пойди и~посмотри». \textsubscript{35}~Иисус прослезился. \textsubscript{36}~Тогда иудеи говорили: «Смотри, как Он любил его». \textsubscript{37}~А~некоторые из них сказали: «Не мог ли Сей, отверзший очи слепому, сделать, чтобы и~этот не умер?» \textsubscript{38}~Иисус же, опять скорбя внутренне, приходит к~гробнице. То была пещера, и~камень лежал перед ней. \textsubscript{39}~Иисус говорит: «Уберите камень». Сестра умершего, Марфа, говорит Ему: «Господи! Уже смердит, ибо четыре дня, как он умер». \textsubscript{40}~Иисус говорит ей: «Не сказал ли Я~тебе, что, если будешь веровать, увидишь славу Божию?» \textsubscript{41}~Итак, убрали камень от пещеры, где лежал умерший. Иисус же возвел очи к~небу и~сказал: «Отче! Благодарю Тебя, что Ты услышал Меня. \textsubscript{42}~Я~и~знал, что Ты всегда услышишь Меня, но сказал это для народа, здесь стоящего, чтобы поверили, что Ты послал Меня». \textsubscript{43}~Сказав это, Он воззвал громким голосом: «Лазарь! Выходи!» \textsubscript{44}~И~вышел умерший, обвитый по рукам и~ногам погребальными пеленами, и~лицо его обвязано было платком. Иисус говорит им: «Развяжите его, пусть идет». 

\subsection*{Вопросы по наблюдению за текстом}
\begin{enumerate}
    \item Перечислите всех действующих лиц данного исторического повествования. 
    
    \myline
    
    \myline
    \item Кто такой был Лазарь? Откуда он был? И~что произошло с~ним? 
    
    \myline
    
    \myline
    \item Что вы узнаете о~Христе из 5~стиха? 
    
    \myline
    
    \myline
    \item Как Иисус воскресил Лазаря? В~каком состоянии был Лазарь? 
    
    \myline
    
    \myline
\end{enumerate}

\subsection*{Вопросы по применению текста} 
\begin{enumerate}
    \item На ваш взгляд, что является главным событием 11~главы Евангелия от Иоанна? 
    
    \myline
    
    \myline
    \item Опишите реакцию сестер Лазаря на его смерть. В~чём разница между реакциями сестер? 
    
    \myline
    
    \myline
    \item Что вы узнали о~Христе из данного отрывка? Как чудо, о~котором мы читаем в~отрывке, раскрывает Христа? Составьте небольшой список.
    
    \myline
    
    \myline
\end{enumerate}


% \newpage

\section{Евангелие от Иоанна 11:45-57. Последствия чуда}

\textsubscript{45}~Тогда многие из иудеев, пришедших к~Марии и~видевших, что сотворил Иисус, уверовали в~Него. \textsubscript{46}~А~некоторые из них пошли к~фарисеям и~сказали им, что сделал Иисус. \textsubscript{47}~Тогда первосвященники и~фарисеи собрали совет и~говорили: «Что нам делать? Этот Человек много чудес творит. \textsubscript{48}~Если оставим Его так, то все уверуют в~Него, и~придут римляне, и~овладеют и~местом нашим, и~народом». \textsubscript{49}~Один же из них, некто Каиафа, будучи на тот год первосвященником, сказал им: «Вы ничего не знаете, \textsubscript{50}~и~не подумаете, что лучше нам, чтобы один человек умер за людей, нежели чтобы весь народ погиб». \textsubscript{51}~Это же он сказал не от себя, но, будучи на тот год первосвященником, предсказал, что Иисус умрет за народ, \textsubscript{52}~и~не только за народ, но чтобы и~рассеянных детей Божиих собрать воедино. \textsubscript{53}~С~этого дня решили убить Его. \textsubscript{54}~Поэтому Иисус уже не ходил явно между иудеями, а~пошёл оттуда в~местность близ пустыни, в~город, называемый Ефраим, и~там оставался с~учениками Своими. \textsubscript{55}~Приближалась Пасха иудейская, и~многие из всей страны пришли в~Иерусалим перед Пасхой, чтобы очиститься. \textsubscript{56}~Тогда искали Иисуса и, стоя в~храме, говорили друг другу: «Как вы думаете: не придет ли Он на праздник?» \textsubscript{57}~Первосвященники же и~фарисеи дали приказ, что если кто узнает, где Он, то пусть сообщит, чтобы взять Его. 

\subsection*{Вопросы по наблюдению за текстом}
\begin{enumerate}
    \item Как люди отреагировали на произошедшее чудо? А~как отреагировали первосвященники и~фарисеи? 
    
    \myline
    
    \myline
    \item Кто был первосвященником в~тот год? Как он отреагировал на служение Христа? 
    
    \myline
    
    \myline
    \item Куда ушёл Христос со своими учениками после всех событий? 
    
    \myline
    
    \myline
\end{enumerate}

\subsection*{Вопросы по применению текста} 
\begin{enumerate}
    \item На ваш взгляд, исходя из данного отрывка, чего боялись фарисеи? 
    
    \myline
    
    \myline
    \item Подумайте и~напишите, какая перемена наступила в~служении Иисуса после совершения данного чуда? 
    
    \myline
    
    \myline
    \item Подумайте, как практически история о~Лазаре отражается на вашей жизни. Напишите два конкретных практических применения для своей жизни. 
    
    \myline
    
    \myline
\end{enumerate}


% \newpage

\section{Евангелие от Иоанна 12:1-8. Преданность Марии}

\textsubscript{1}~За шесть дней до Пасхи пришёл Иисус в~Вифанию, где был Лазарь умерший, которого Он воскресил из мертвых. \textsubscript{2}~Там приготовили Ему вечерю, и~Марфа служила, и~Лазарь был одним из возлежавших с~Ним. \textsubscript{3}~Мария же, взяв фунт чистого драгоценного нардового мира, помазала ноги Иисуса и~отерла волосами своими ноги Его; и~дом наполнился благоуханием от мира. \textsubscript{4}~Тогда один из учеников Его, Иуда Симонов Искариот, который хотел предать Его, сказал: \textsubscript{5}~«Почему бы не продать это миро за триста динариев и~не раздать нищим?» \textsubscript{6}~Сказал же он это не потому, что заботился о~нищих, но потому, что был вором. Он имел при себе денежный ящик и~брал из того, что туда опускали. \textsubscript{7}~Иисус же сказал: «Оставьте её; она сберегла это на день погребения Моего. \textsubscript{8}~Ибо нищих всегда имеете с~собой, а~Меня не всегда».

\subsection*{Вопросы по наблюдению за текстом}
\begin{enumerate}
    \item Отметьте время и~место действия. Где находится Иисус, что Он делает, кто окружает Его? 
    
    \myline
    
    \myline
    \item Как отреагировал Иуда на происходящее? Почему он так отреагировал? 
    
    \myline
    
    \myline
\end{enumerate}

\subsection*{Вопросы по применению текста} 
\begin{enumerate}
    \item Прочтите ещё раз стихи 7~и~8. Дайте развернутый комментарий на слова Христа. 
    
    \myline
    
    \myline
    \item Как вы относитесь к~тем, кто вас предал? Какой пример может вам преподать в~этом случае Иисус Своим отношением к~предателю-Иуде? 
    
    \myline
    
    \myline
\end{enumerate}


% \newpage

\section{Евангелие от Иоанна 12:9-11. Отношение к~пребыванию Христа в~Вифании}

\textsubscript{9}~Многие из иудеев узнали, что Он там, и~пришли не только ради Иисуса, но чтобы видеть и~Лазаря, которого Он воскресил из мертвых. \textsubscript{10}~Первосвященники же положили убить и~Лазаря, \textsubscript{11}~потому что из-за него многие из иудеев приходили и~веровали в~Иисуса.

\subsection*{Вопросы по наблюдению за текстом}
\begin{enumerate}
    \item С~какой целью приходили многие иудеи в~Вифанию? Где был на тот момент Иисус? 
    
    \myline
    
    \myline
    \item Что замыслили сделать первосвященники с~Лазарем? 
    
    \myline
    
    \myline
\end{enumerate}

\subsection*{Вопросы по применению текста} 
\begin{enumerate}
    \item Из-за кого многие из иудеев стали веровать в~Иисуса? Почему это происходило? 
    
    \myline
    
    \myline
    \item Свидетельством чего является ваша жизнь? Является ли ваша жизнь свидетельством вашего духовного воскресения? 
    
    \myline
    
    \myline
\end{enumerate}


% \newpage

\section{Евангелие от Иоанна 12:12-19. Въезд в~Иерусалим}

\textsubscript{12}~На другой день множество народа, пришедшего на праздник, услышав, что Иисус идет в~Иерусалим, \textsubscript{13}~взяли пальмовые ветви, вышли навстречу Ему и~восклицали: «Осанна! Благословен Грядущий во имя Господне~--- Царь Израилев!» \textsubscript{14}~Иисус же, найдя молодого осла, сел на него, как написано: \textsubscript{15}~«Не бойся, дочь Сиона! Вот, Царь твой грядет, сидя на молодом осле». \textsubscript{16}~Ученики Его сперва не поняли этого; но когда прославился Иисус, тогда вспомнили, что так было о~Нем написано и~это сделали Ему. \textsubscript{17}~Народ, бывший с~Ним прежде, свидетельствовал, что Он вызвал из гробницы Лазаря и~воскресил его из мертвых. \textsubscript{18}~Потому и~встретил Его народ, ибо слышал, что Он сотворил это чудо. \textsubscript{19}~Фарисеи же говорили между собой: «Видите ли, что не успеваете ничего? Весь мир идет за Ним».

\subsection*{Вопросы по наблюдению за текстом}
\begin{enumerate}
    \item Судя по ближайшему контексту, на какой праздник пришёл народ в~Иерусалим? 
    
    \myline
    
    \myline
    \item Что сделал народ, когда услышал, что Иисус идет в~Иерусалим? 
    
    \myline
    
    \myline
    \item На ком Христос въезжал в~Иерусалим? 
    
    \myline
    \item Каким восклицанием народ встречал Христа? 
    
    \myline
    
    \item Как отреагировали на происходящее ученики Христа, народ и~фарисеи? Опишите их реакцию своими словами. 
    
    \myline
    
    \myline
\end{enumerate}

\subsection*{Вопросы по применению текста} 
\begin{enumerate}
    \item Поразмышляйте над теми словами, которыми народ встречал Христа в~13 стихе. Почему, на ваш взгляд, народ встречал Христа именно этими словами и, согласно их реакции, кого они видели во Христе? 
    
    \myline
    
    \myline
    \item Что стихи 17, 18~говорят о~мотивах народа, который встретил Христа? Почему их влекло ко Христу и~почему вас влечет ко Христу? 
    
    \myline
    
    \myline
\end{enumerate}


% \newpage

\section{Евангелие от Иоанна 12:20-2.6 Греки в~поисках Христа}

\textsubscript{20}~Из пришедших на поклонение в~праздник были некоторые эллины. \textsubscript{21}~Они подошли к~Филиппу, который был из Вифсаиды галилейской, и~просили его, говоря: «Господин! Нам хочется видеть Иисуса». \textsubscript{22}~Филипп идет и~говорит о~том Андрею; и~потом Андрей и~Филипп говорят о~том Иисусу. \textsubscript{23}~Иисус же сказал им в~ответ: «Пришёл час прославиться Сыну Человеческому. \textsubscript{24}~Истинно, истинно говорю вам: если пшеничное зерно, упав в~землю, не умрет, то останется одно; а~если умрет, то принесет много плода. \textsubscript{25}~Любящий душу свою погубит её, а~ненавидящий душу свою в~мире этом сохранит её в~жизнь вечную. \textsubscript{26}~Кто Мне служит, Мне да последует; и~где Я, там и~слуга Мой будет. И~кто Мне служит, того почтит Отец Мой.

\subsection*{Вопросы по наблюдению за текстом}
\begin{enumerate}
    \item О~какой категории людей упоминает евангелист в~20 стихе? С~какой целью они пришли в~Иерусалим? 
    
    \myline
    
    \myline
    \item Кто из учеников Христа упоминается в~данном повествовании? 
    
    \myline
    
    \myline
    \item О~каком часе Христос сказал, что он пришёл (стих 23)? 
    
    \myline
    
    \myline
\end{enumerate}

\subsection*{Вопросы по применению текста} 
\begin{enumerate}
    \item Подумайте, почему, по-вашему, Иоанн упомянул, что на иудейском празднике были эллины? Можно ли было приглашать иностранцев, в~частности эллинов, на иудейские праздники?
    
    \myline
    
    \myline
    \item Что имеет в~виду Иисус под словом «прославиться» в~стихе 23? Поясните свой ответ. 
    
    \myline
    
    \myline
    \item Иисус рассказывает короткую притчу в~стихе 24. В~чём цель семени? Как оно достигает своей цели? Какие изменения происходят в~семени, пока оно исполняет свою цель? 
    
    \myline
    
    \myline
    \item Как то, о~чём говорит Христос в~стихе 23, связано с~тем, что Он говорит ниже в~последующих стихах? Постарайтесь ответить максимально подробно. 
    
    \myline
    
    \myline
\end{enumerate}


% \newpage

\section{Евангелие от Иоанна 12:27-36. Свидетельство и~уход}

\textsubscript{27}~Душа Моя теперь сильно взволнована; и~что Мне сказать? Отче! Избавь Меня от часа сего! Но на сей час Я~и~пришёл. \textsubscript{28}~Отче! Прославь имя Твое». Тогда пришёл с~неба глас: «И прославил, и~ещё прославлю». \textsubscript{29}~Народ, стоявший и~слышавший то, говорил: «Это гром»; а~другие говорили: «Ангел говорил Ему». \textsubscript{30}~Иисус на это сказал: «Не для Меня был глас этот, но для народа. \textsubscript{31}~Ныне суд миру сему; ныне князь мира сего изгнан будет вон. \textsubscript{32}~И~когда Я~вознесен буду от земли, всех привлеку к~Себе». \textsubscript{33}~Это говорил Он, давая разуметь, какой смертью Он умрет. \textsubscript{34}~Народ отвечал Ему: «Мы слышали из закона, что Христос пребывает вовек. Как же Ты говоришь, что должен быть вознесен Сын Человеческий? Кто Этот Сын Человеческий?» \textsubscript{35}~Тогда Иисус сказал им: «Ещё на малое время свет с~вами; ходите, пока есть свет, чтобы не объяла вас тьма; а~ходящий во тьме не знает, куда идет. \textsubscript{36}~Доколе свет с~вами, веруйте в~свет, чтобы стать вам сынами света». Сказав это, Иисус отошёл и~скрылся от них. 

\subsection*{Вопросы по наблюдению за текстом}
\begin{enumerate}
    \item Что вы узнаете о~Христе в~стихах 27~и~28? 
    
    \myline
    
    \myline
    \item Как отреагировал Отец на молитву Христа? 
    
    \myline
    
    \myline
    \item Для чего говорил Христос с~народом согласно 33~стиху? 
    
    \myline
    
    \myline
    \item Как назван дьявол в~31 стихе? 
    
    \myline
    
    \myline
\end{enumerate}

\subsection*{Вопросы по применению текста} 
\begin{enumerate}
    \item Как народ отреагировал на голос Божий в~ст. 28-29? Почему такая реакция народа была ожидаема? 
    
    \myline
    
    \myline
    \item Каким образом смерть Иисуса является знамением суда (стих 31)? 
    
    \myline
    
    \myline
    \item Как смерть Иисуса являет суд над дьяволом (стих 31)? 
    
    \myline
    
    \myline
    \item Какое предостережение в~отношении суда находится в~стихах 35-36? 
    
    \myline
    
    \myline
    \item Как вы думаете, почему Иисус скрылся от иудеев (см. стих 36)? 
    
    \myline
    
    \myline
\end{enumerate}


% \newpage

\section{Евангелие от Иоанна 12:37-50. Продолжающееся неверие}

\textsubscript{37}~Столько чудес сотворил Он перед ними, и~они не веровали в~Него, \textsubscript{38}~да сбудется слово Исаии, пророка: «Господи! Кто поверил слышанному от нас? И~кому открылась мышца Господня?» \textsubscript{39}~Потому не могли они веровать, что, как ещё сказал Исаия: \textsubscript{40}~«Народ этот ослепил глаза свои и~окаменил сердце свое, да не видят глазами, и~не уразумеют сердцем, и~не обратятся, чтобы Я~исцелил их». \textsubscript{41}~Это сказал Исаия, когда видел славу Его и~говорил о~Нем. \textsubscript{42}~Впрочем, и~из начальников многие уверовали в~Него; но из-за фарисеев не исповедовали, чтобы не быть отлученными от синагоги, \textsubscript{43}~ибо возлюбили больше славу человеческую, нежели славу Божию. \textsubscript{44}~Иисус же возгласил и~сказал: «Верующий в~Меня не в~Меня верует, но в~Пославшего Меня. \textsubscript{45}~И~видящий Меня видит Пославшего Меня. \textsubscript{46}~Я~— свет, пришёл в~мир, чтобы всякий, верующий в~Меня, не оставался во тьме. \textsubscript{47}~И~если кто услышит Мои слова и~не поверит, Я~не сужу его, ибо Я~пришёл не судить мир, но спасти мир. \textsubscript{48}~Отвергающий Меня и~не принимающий слов Моих имеет судью себе: слово, которое Я~говорил, оно будет судить его в~последний день. \textsubscript{49}~Ибо Я~говорил не от Себя, но пославший Меня Отец Сам дал Мне заповедь, что сказать и~что говорить. \textsubscript{50}~И~Я знаю, что заповедь Его есть жизнь вечная. Итак, что Я~говорю, говорю, как сказал Мне Отец». 

\subsection*{Вопросы по наблюдению за текстом}
\begin{enumerate}
    \item На какую реакцию народа указывает евангелист в~стихе 37? 
    
    \myline
    
    \myline
    \item Какого пророка Ветхого Завета цитирует Христос? 
    
    \myline
    
    \myline
    \item Как Христос говорит о~Себе в~46 стихе? 
    
    \myline
    
    \myline
    \item Как вы думаете, говорит ли 45~стих о~Божественности Христа? Поясните свой ответ максимально развернуто. 
    
    \myline
    
    \myline
\end{enumerate}

\subsection*{Вопросы по применению текста} 
\begin{enumerate}
    \item Поразмышляйте и~прокомментируйте 40~стих: как вы его понимаете в~контексте данного отрывка? 
    
    \myline
    
    \myline
    \item Почему многие начальники и~вожди народа израильского открыто не исповедовали веру во Христа и~не следовали за Ним? В~чём у~них была проблема? Актуальна ли она для сегодняшнего дня? Свой ответ поясните. 
    
    \myline
    
    \myline
\end{enumerate}


% \newpage

\section{Евангелие от Иоанна 13:1-38. Омовение ног и~последствия этой символической акции}

\textsubscript{1}~Перед праздником Пасхи Иисус, зная, что пришёл час Его перейти от мира сего к~Отцу, явил делом, что, возлюбив Своих, находящихся в~мире, до конца возлюбил их. \textsubscript{2}~И~во время вечери, когда дьявол уже вложил в~сердце Иуде Симонову Искариоту предать Его, \textsubscript{3}~Иисус, зная, что Отец всё отдал в~руки Его и~что Он от Бога исшёл и~к Богу отходит, \textsubscript{4}~встал с~вечери, снял с~Себя верхнюю одежду и, взяв полотенце, препоясался. \textsubscript{5}~Потом влил воды в~умывальницу и~начал умывать ноги ученикам и~отирать полотенцем, которым был препоясан. \textsubscript{6}~Подходит к~Симону Петру, и~тот говорит Ему: «Господи! Тебе ли умывать мои ноги?» \textsubscript{7}~Иисус сказал ему в~ответ: «Что Я~делаю, теперь ты не знаешь, а~уразумеешь после». \textsubscript{8}~Петр говорит Ему: «Не умоешь ног моих вовек». Иисус отвечал ему: «Если не умою тебя, не имеешь части со Мною». \textsubscript{9}~Симон Петр говорит Ему: «Господи! Не только ноги мои, но и~руки, и~голову». \textsubscript{10}~Иисус говорит ему: «Омытому нужно только ноги умыть, потому что чист весь; и~вы чисты, но не все». \textsubscript{11}~Ибо знал Он предателя Своего, потому и~сказал: «Не все вы чисты». \textsubscript{12}~Когда же умыл им ноги и~надел одежду Свою, то, возлегши опять, сказал им: «Знаете ли, что Я~сделал вам? \textsubscript{13}~Вы называете Меня Учителем и~Господом и~правильно говорите, ибо Я~то и~есть. \textsubscript{14}~Итак, если Я, Господь и~Учитель, умыл ноги вам, то и~вы должны умывать ноги друг другу. \textsubscript{15}~Ибо Я~дал вам пример, чтобы и~вы делали то же, что Я~сделал вам. \textsubscript{16}~Истинно, истинно говорю вам: раб не больше господина своего, и~посланник не больше пославшего его. \textsubscript{17}~Если это знаете, блаженны вы, когда исполняете. \textsubscript{18}~Не о~всех вас говорю; Я~знаю, которых избрал. Но да сбудется Писание: „Едящий со Мною хлеб поднял на Меня пяту свою“. \textsubscript{19}~Теперь говорю вам, прежде нежели это сбылось, дабы, когда сбудется, вы поверили, что это Я. \textsubscript{20}~Истинно, истинно говорю вам: принимающий того, кого Я~пошлю, Меня принимает; а~принимающий Меня принимает Пославшего Меня». \textsubscript{21}~Сказав это, Иисус возмутился духом, и~засвидетельствовал, и~сказал: «Истинно, истинно говорю вам, что один из вас предаст Меня». \textsubscript{22}~Тогда ученики озирались друг на друга, недоумевая, о~ком Он говорит. \textsubscript{23}~Один же из учеников Его, которого любил Иисус, возлежал у~груди Иисуса. \textsubscript{24}~Ему Симон Петр сделал знак, чтобы спросил, кто это, о~котором говорит. \textsubscript{25}~Он, припав к~груди Иисуса, сказал Ему: «Господи! Кто это?» \textsubscript{26}~Иисус отвечал: «Тот, кому Я, обмакнув кусок хлеба, подам». И, обмакнув кусок, подал Иуде Симонову Искариоту. \textsubscript{27}~И~после этого куска вошёл в~него сатана. Тогда Иисус сказал ему: «Что делаешь, делай скорее». \textsubscript{28}~Но никто из возлежавших не понял, к~чему Он это сказал ему. \textsubscript{29}~А~как у~Иуды был ящик, то некоторые думали, что Иисус говорит ему: «Купи, что нам нужно к~празднику»~--- или чтобы дал что-нибудь нищим. \textsubscript{30}~Он, приняв кусок, тотчас вышел. Была же ночь. \textsubscript{31}~Когда он вышел, Иисус сказал: «Ныне прославился Сын Человеческий, и~Бог прославился в~Нем. \textsubscript{32}~Если Бог прославился в~Нем, то и~Бог прославит Его в~Себе, и~вскоре прославит Его. \textsubscript{33}~Дети! Недолго уже быть Мне с~вами. Будете искать Меня, и~как сказал Я~иудеям: „Куда Я~иду, вы не можете прийти“,~--- так и~вам говорю теперь. \textsubscript{34}~Заповедь новую даю вам: да любите друг друга; как Я~возлюбил вас, так и~вы да любите друг друга. \textsubscript{35}~По тому узнают все, что вы Мои ученики, если будете иметь любовь между собою». \textsubscript{36}~Симон Петр сказал Ему: «Господи! Куда Ты идешь?» Иисус отвечал ему: «Куда Я~иду, ты не можешь теперь за Мною идти, но после пойдешь за Мною». \textsubscript{37}~Петр сказал Ему: «Господи! Почему я~не могу идти за Тобой теперь? Я~душу мою положу за Тебя». \textsubscript{38}~Иисус отвечал ему: «Душу свою за Меня положишь? Истинно, истинно говорю тебе: не пропоет петух, как отречешься от Меня трижды. 

\subsection*{Вопросы по наблюдению за текстом}
\begin{enumerate}
    \item Перед каким праздником разворачиваются события в~данной главе? 
    
    \myline
    
    \myline
    \item О~каком часе говорится в~первом стихе? 
    
    \myline
    
    \myline
    \item Кто главные действующие персонажи главы? 
    
    \myline
    
    \myline
    \item Какую заповедь дает Христос Своим ученикам? 
    
    \myline
    
    \myline
    \item Какое пророчество произносит Христос относительно Петра? 
\end{enumerate}

\subsection*{Вопросы по применению текста} 
\begin{enumerate}
    \item В~чём значение того, что Иисус был одет и~поступал, как слуга (ст.~4)? Как это подытоживает цель Его пришествия на землю?
    
    \myline
    
    \myline
    \item Согласно ст. 1, что значит «до конца возлюбил их»? Какова взаимосвязь между мытьем ног учеников и~крестом? 
    
    \myline
    
    \myline
    \item Как Иисус объяснил Свои действия в~ст. 13-14? 
    
    \myline
    
    \myline
    \item Мыл ли Иисус ноги Иуды? В~чём значение этого действия? Насколько и~почему тяжело вам служить сложным людям? 
    
    \myline
    
    \myline
    \item Почему, по-вашему, Петр пытался помешать Иисусу омыть его ноги? В~какие моменты ваша гордость мешает вам получать заботу от других людей?
    
    \myline
    
    \myline
    \item Должны ли мы сегодня мыть ноги друг другу? Поясните свой ответ. 
    
    \myline
    
    \myline
    \item Как люди, которые окружают нас, могут узнать, что мы ученики Христа? Являетесь ли вы учеником Христа? Если да, то что должно быть в~вашей жизни, согласно 34-35 стихам? 
    
    \myline
    
    \myline
\end{enumerate}


% \newpage

\section{Евангелие от Иоанна 14:1-31. Обещания и~заповеди ученикам}

\textsubscript{1}~Да не смущается сердце ваше; веруйте в~Бога и~в~Меня веруйте. \textsubscript{2}~В~доме Отца Моего обителей много. А~если бы не так, Я~сказал бы вам: „Я иду приготовить место вам“. \textsubscript{3}~И~когда пойду и~приготовлю вам место, приду опять и~возьму вас к~Себе, чтобы и~вы были, где Я. \textsubscript{4}~А~куда Я~иду, вы знаете и~путь знаете». \textsubscript{5}~Фома сказал Ему: «Господи! Не знаем, куда идешь. И~как можем знать путь?» \textsubscript{6}~Иисус сказал ему: «Я~--- путь, и~истина, и~жизнь; никто не приходит к~Отцу, как только через Меня. \textsubscript{7}~Если бы вы знали Меня, то знали бы и~Отца Моего. И~отныне знаете Его и~видели Его». \textsubscript{8}~Филипп сказал Ему: «Господи! Покажи нам Отца~--- и~довольно для нас». \textsubscript{9}~Иисус сказал ему: «Столько времени Я~с~вами, и~ты не знаешь Меня, Филипп? Видевший Меня видел Отца. Как же ты говоришь: „Покажи нам Отца“? \textsubscript{10}~Разве ты не веришь, что Я~в~Отце и~Отец во Мне? Слова, которые говорю Я~вам, говорю не от Себя; Отец, пребывающий во Мне, Он творит дела. \textsubscript{11}~Верьте Мне, что Я~в~Отце и~Отец во Мне; а~если не так, то верьте Мне по самим делам. \textsubscript{12}~Истинно, истинно говорю вам: верующий в~Меня, дела, которые творю Я, и~он сотворит, и~больше этих сотворит, потому что Я~к~Отцу Моему иду. \textsubscript{13}~И~если чего попросите у~Отца во имя Моё, то сделаю, да прославится Отец в~Сыне. \textsubscript{14}~Если чего попросите во имя Моё, Я~то сделаю. \textsubscript{15}~Если любите Меня, соблюдите Мои заповеди. \textsubscript{16}~И~Я умолю Отца, и~даст вам другого Утешителя, да пребудет с~вами вовек, \textsubscript{17}~Духа истины, Которого мир не может принять, потому что не видит Его и~не знает Его; а~вы знаете Его, ибо Он с~вами пребывает и~в~вас будет. \textsubscript{18}~Не оставлю вас сиротами, приду к~вам. \textsubscript{19}~Ещё немного~--- и~мир уже не увидит Меня; а~вы увидите Меня, ибо Я~живу, и~вы будете жить. \textsubscript{20}~В~тот день узнаете вы, что Я~в~Отце Моем, и~вы во Мне, и~Я в~вас. \textsubscript{21}~Кто имеет заповеди Мои и~соблюдает их, тот любит Меня; а~кто любит Меня, тот возлюблен будет Отцом Моим; и~Я возлюблю его и~явлюсь ему Сам». \textsubscript{22}~Иуда~--- не Искариот~--- говорит Ему: «Господи! Что это, что Ты хочешь явить Себя нам, а~не миру?» \textsubscript{23}~Иисус сказал ему в~ответ: «Кто любит Меня, тот соблюдет слово Моё; и~Отец Мой возлюбит его, и~Мы придем к~нему и~обитель у~него сотворим. \textsubscript{24}~Не любящий Меня не соблюдает слов Моих; слово же, которое вы слышите, не Моё, но пославшего Меня Отца. \textsubscript{25}~Это сказал Я~вам, находясь с~вами. \textsubscript{26}~Утешитель же, Дух Святой, Которого пошлет Отец во имя Моё, научит вас всему и~напомнит вам всё, что Я~говорил вам. \textsubscript{27}~Мир оставляю вам, мир Мой даю вам; не так, как мир дает, Я~даю вам. Да не смущается сердце ваше и~да не устрашается. \textsubscript{28}~Вы слышали, что Я~сказал вам: „Иду от вас и~приду к~вам“. Если бы вы любили Меня, то возрадовались бы, что Я~сказал: иду к~Отцу,~--- ибо Отец Мой больше Меня. \textsubscript{29}~И~вот, Я~сказал вам о~том, прежде нежели сбылось, дабы вы поверили, когда сбудется. \textsubscript{30}~Уже немного Мне говорить с~вами, ибо идет князь мира этого, и~во Мне не имеет ничего. \textsubscript{31}~Но это сбудется, чтобы мир знал, что Я~люблю Отца и, как заповедал Мне Отец, так и~творю. Встаньте, пойдем отсюда. 

\subsection*{Вопросы по наблюдению за текстом}
\begin{enumerate}
    \item Почему Иисус после Своего воскресения должен был уйти к~Отцу (см. стихи 2,~3)? 
    
    \myline
    
    \myline
    \item Что говорит Христос о~Себе в~стихе 6? 
    
    \myline
    
    \myline
    \item Какова задача Духа Святого здесь на земле (см. стихи 26, 27)? 
    
    \myline
    
    \myline
    \item Кто из учеников попросил Христа показать Отца? Как Христос ответил ему? Что ответ Христа говорит о~Его взаимоотношениях с~Отцом? 
    
    \myline
    
    \myline
\end{enumerate}

\subsection*{Вопросы по применению текста} 
\begin{enumerate}
    \item Согласно стиху 6, какой есть единственный путь к~Отцу? 
    
    \myline
    
    \myline
    \item Прокомментируете стих 6~подробнее: что для вас практически значит, что Христос есть путь, истина, жизнь? 
    
    \myline
    
    \myline
    \item В~Ин. 14:16 Дух назван «другим Утешителем». Кто был первым Утешителем? Как вы понимаете взаимоотношения между Иисусом и~Святым Духом? 
    
    \myline
    
    \myline
    \item Слово, переведенное как «утешитель» (греч. «параклетос»), буквально означает «пришедший быть рядом, чтобы помочь». Как Святой Дух утешает, приходя быть рядом с~вами? 
    
    \myline
    
    \myline
    \item В~древнем мире слово «параклетос» также использовалось для описания законного адвоката, который «пришёл быть рядом» с~человеком во время суда. Каким образом Дух является вашим защитником? 
    
    \myline
    
    \myline
    \item Согласно 15~и~23 стихам, в~чём практически должна выражаться любовь ко Христу? Есть ли это в~вашей жизни?
    
    \myline
    
    \myline
\end{enumerate}


% \newpage

\section{Евангелие от Иоанна 15:1-17. Притча о~виноградной лозе}

\textsubscript{1}~Я~— истинная виноградная лоза, а~Отец Мой~— виноградарь. \textsubscript{2}~Всякую у~Меня ветвь, не приносящую плода, Он отсекает; и~всякую, приносящую плод, очищает, чтобы более принесла плода. \textsubscript{3}~Вы уже очищены через слово, которое Я~проповедал вам. \textsubscript{4}~Пребудьте во Мне, и~Я в~вас. Как ветвь не может приносить плода сама собой, если не будет на лозе, так и~вы, если не будете во Мне. \textsubscript{5}~Я~— лоза, а~вы ветви. Кто пребывает во Мне, и~Я в~нём, тот приносит много плода; ибо без Меня не можете делать ничего. \textsubscript{6}~Кто не пребудет во Мне, извергнется вон, как ветвь, и~засохнет; а~такие ветви собирают и~бросают в~огонь, и~они сгорают. \textsubscript{7}~Если пребудете во Мне и~слова Мои в~вас пребудут, то, чего ни пожелаете, просите~--- и~будет вам. \textsubscript{8}~Тем прославится Отец Мой, если вы принесете много плода и~будете Моими учениками. \textsubscript{9}~Как возлюбил Меня Отец, и~Я возлюбил вас; пребудьте в~любви Моей. \textsubscript{10}~Если заповеди Мои соблюдете, пребудете в~любви Моей, как и~Я соблюл заповеди Отца Моего и~пребываю в~Его любви. \textsubscript{11}~Это сказал Я~вам, чтобы радость Моя в~вас была, и~радость ваша будет совершенна. \textsubscript{12}~Это заповедь Моя: да любите друг друга, как Я~возлюбил вас. \textsubscript{13}~Нет больше той любви, как если кто положит душу свою за друзей своих. \textsubscript{14}~Вы~--- друзья Мои, если исполняете то, что Я~заповедую вам. \textsubscript{15}~Я~уже не называю вас рабами, ибо раб не знает, что делает господин его; но Я~назвал вас друзьями, потому что сказал вам всё, что слышал от Отца Моего. \textsubscript{16}~Не вы Меня избрали, а~Я вас избрал и~поставил вас, чтобы вы шли и~приносили плод и~чтобы плод ваш пребывал, дабы, чего ни попросите у~Отца во имя Моё, Он дал вам. \textsubscript{17}~Это заповедую вам: да любите друг друга. 

\subsection*{Вопросы по наблюдению за текстом}
\begin{enumerate}
    \item Какой образ использует Христос, чтобы донести до Своих учеников истину? 
    
    \myline
    
    \myline
    \item Как Христос называет Себя и~Своего Отца? 
    
    \myline
    
    \myline
    \item Какую заповедь Христос вновь напоминает Своим ученикам (см. 12~стих)? 
    
    \myline
    
    \myline
\end{enumerate}

\subsection*{Вопросы по применению текста} 
\begin{enumerate}
    \item Поразмышляйте над 5~стихом в~его контексте: что, на ваш взгляд, хочет сказать Христос? Что значит пребывать в~Нем?
    
    \myline
    
    \myline
    \item Согласно этому отрывку, кто может пребывать в~Иисусе и~каков результат этого? 
    
    \myline
    
    \myline
    \item Что для вас значит практически то, что вы пребываете в~Боге, а~Бог во вас (см. 4, 5~стихи)? 
    
    \myline
    
    \myline
    \item Поразмышляйте и~прокомментируйте 16~стих: что утверждает данный текст? Дайте развернутый ответ.
    
    \myline
    
    \myline
\end{enumerate}


% \newpage

\section{Евангелие от Иоанна 15:18-16:33. Наставление учеников}

\textsubscript{18}~Если мир вас ненавидит, знайте, что Меня прежде вас возненавидел. \textsubscript{19}~Если бы вы были от мира, то мир любил бы свое; а~как вы не от мира, но Я~избрал вас от мира, потому ненавидит вас мир. \textsubscript{20}~Помните слово, которое Я~сказал вам: „Раб не больше господина своего“. Если Меня гнали, будут гнать и~вас; если Моё слово соблюдали, будут соблюдать и~ваше. \textsubscript{21}~Но всё то сделают вам за имя Моё, потому что не знают Пославшего Меня. \textsubscript{22}~Если бы Я~не пришёл и~не говорил им, то не имели бы греха; а~теперь не имеют извинения в~грехе своем. \textsubscript{23}~Ненавидящий Меня ненавидит и~Отца Моего. \textsubscript{24}~Если бы Я~не сотворил между ними дел, каких никто другой не делал, то не имели бы греха; а~теперь и~видели, и~возненавидели и~Меня, и~Отца Моего. \textsubscript{25}~Но да сбудется слово, написанное в~законе их: „Возненавидели Меня напрасно“. \textsubscript{26}~Когда же придет Утешитель, Которого Я~пошлю вам от Отца, Дух истины, Который от Отца исходит, Он будет свидетельствовать обо Мне; \textsubscript{27}~а~также и~вы будете свидетельствовать, потому что вы от начала со Мной. \textsubscript{1}~Это сказал Я~вам, чтобы вы не соблазнились. \textsubscript{2}~Изгонят вас из синагог; даже наступает время, когда всякий, убивающий вас, будет думать, что он тем служит Богу. \textsubscript{3}~Так будут поступать, потому что не познали ни Отца, ни Меня. \textsubscript{4}~Но Я~сказал вам это для того, чтобы вы, когда придет то время, вспомнили, что Я~сказал вам о~том. Не говорил же этого вам сначала, потому что был с~вами. \textsubscript{5}~А~теперь иду к~Пославшему Меня, и~никто из вас не спрашивает Меня: „Куда идешь?“ \textsubscript{6}~Но оттого, что Я~сказал вам это, печалью исполнилось сердце ваше. \textsubscript{7}~Но Я~истину говорю вам: лучше для вас, чтобы Я~пошёл; ибо, если Я~не пойду, Утешитель не придет к~вам; а~если пойду, то пошлю Его к~вам. \textsubscript{8}~И~Он, придя, обличит мир о~грехе, и~о правде, и~о суде: \textsubscript{9}~о~грехе, что не веруют в~Меня; \textsubscript{10}~о~правде, что Я~иду к~Отцу Моему и~уже не увидите Меня; \textsubscript{11}~о~суде же, что князь мира сего осужден. \textsubscript{12}~Ещё многое имею сказать вам, но вы теперь не можете вместить. \textsubscript{13}~Когда же придет Он, Дух истины, то наставит вас на всякую истину, ибо не от Себя говорить будет, но будет говорить, что услышит, и~будущее возвестит вам. \textsubscript{14}~Он прославит Меня, потому что от Моего возьмет и~возвестит вам. \textsubscript{15}~Все, что имеет Отец,~--- Моё, потому Я~сказал, что от Моего возьмет и~возвестит вам. \textsubscript{16}~Вскоре вы не увидите Меня и~опять вскоре увидите Меня, ибо Я~иду к~Отцу». \textsubscript{17}~Тут некоторые из учеников Его сказали друг другу: «Что это Он говорит нам: „Вскоре не увидите Меня и~опять вскоре увидите Меня“ и~„Я иду к~Отцу“?» \textsubscript{18}~Итак, они говорили: «Что это Он говорит: „вскоре“? Не знаем, что говорит». \textsubscript{19}~Иисус, уразумев, что хотят спросить Его, сказал им: «О том ли спрашиваете вы один другого, что Я~сказал: „Вскоре не увидите Меня и~опять вскоре увидите Меня“? \textsubscript{20}~Истинно, истинно говорю вам: вы восплачете и~возрыдаете, а~мир возрадуется; вы печальны будете, но печаль ваша в~радость будет. \textsubscript{21}~Женщина, когда рождает, терпит скорбь, потому что пришёл час её; но когда родит младенца, уже не помнит скорби от радости, потому что родился человек в~мир. \textsubscript{22}~Так и~вы теперь имеете печаль; но Я~увижу вас опять, и~возрадуется сердце ваше, и~радости вашей никто не отнимет у~вас. \textsubscript{23}~И~в~тот день вы не спросите Меня ни о~чём. Истинно, истинно говорю вам: о~чём ни попросите Отца во имя Моё, даст вам. \textsubscript{24}~Доныне вы ничего не просили во имя Моё; просите~--- и~получите, чтобы радость ваша была совершенна. \textsubscript{25}~До сих пор Я~говорил вам притчами, но наступает время, когда уже не буду говорить вам притчами, но прямо возвещу вам об Отце. \textsubscript{26}~В~тот день будете просить во имя Моё. И~не говорю вам, что Я~буду просить Отца о~вас, \textsubscript{27}~ибо Сам Отец любит вас, потому что вы возлюбили Меня и~уверовали, что Я~исшёл от Бога. \textsubscript{28}~Я~исшёл от Отца и~пришёл в~мир; и~опять оставляю мир и~иду к~Отцу». \textsubscript{29}~Ученики Его сказали Ему: «Вот, теперь Ты прямо говоришь и~притчи не говоришь никакой. \textsubscript{30}~Теперь видим, что Ты знаешь всё и~не имеешь нужды, чтобы кто спрашивал Тебя. Поэтому веруем, что Ты от Бога исшёл». \textsubscript{31}~Иисус отвечал им: «Теперь веруете? \textsubscript{32}~Вот наступает час, и~настал уже, что вы рассеетесь каждый в~свою сторону и~Меня оставите одного; но Я~не один, потому что Отец со Мной. \textsubscript{33}~Это сказал Я~вам, чтобы вы имели во Мне мир. В~мире будете иметь скорбь, но мужайтесь: Я~победил мир». 

\subsection*{Вопросы по наблюдению за текстом}
\begin{enumerate}
    \item Согласно тексту 15:26, Кого Христос пошлет вместо Себя? 
    
    \myline
    
    \myline
    \item О~чём говорит Христос в~стихе 16:16? На какое событие Он указывает? 
    
    \myline
    
    \myline
\end{enumerate}

\subsection*{Вопросы по применению текста} 
\begin{enumerate}
    \item В~Ин. 16:8 мы узнаем, что Дух обличит мир о~грехе. Как Он это сделает? Вы испытывали подобное обличение от Духа? 
    
    \myline
    
    \myline
    \item Какая функция Духа описана в~Ин. 16:13? Каким образом Дух наставляет вас в~истине? Какова взаимосвязь между Святым Духом и~Библией? Приведите пример, как Дух «наставлял вас в~истине». 
    
    \myline
    
    \myline
\end{enumerate}


% \newpage

\section{Евангелие от Иоанна 17:1-26. Молитва Христа}

\textsubscript{1}~После этих слов Иисус возвел очи Свои к~небу и~сказал: «Отче! Пришёл час, прославь Сына Твоего, да и~Сын Твой прославит Тебя, \textsubscript{2}~так как Ты дал Ему власть над всякою плотью, да всему, что Ты дал Ему, даст Он жизнь вечную. \textsubscript{3}~Сия же есть жизнь вечная, да знают Тебя, единого истинного Бога, и~посланного Тобою Иисуса Христа. \textsubscript{4}~Я~прославил Тебя на земле, совершил дело, которое Ты поручил Мне исполнить. \textsubscript{5}~И~ныне прославь Меня Ты, Отче, у~Тебя Самого славой, которую Я~имел у~Тебя прежде бытия мира. \textsubscript{6}~Я~открыл имя Твоё людям, которых Ты дал Мне от мира; они были Твои, и~Ты дал их Мне, и~они сохранили слово Твоё. \textsubscript{7}~Ныне уразумели они, что всё, что Ты дал Мне, от Тебя есть, \textsubscript{8}~ибо слова, которые Ты дал Мне, Я~передал им, и~они приняли, и~уразумели истинно, что Я~исшёл от Тебя, и~уверовали, что Ты послал Меня. \textsubscript{9}~Я~о~них молю~--- не обо всем мире молю, но о~тех, которых Ты дал Мне, потому что они Твои. \textsubscript{10}~И~всё Моё~--- Твоё, и~Твоё~--- Моё; и~Я прославился в~них. \textsubscript{11}~Я~уже не в~мире, но они в~мире, а~Я к~Тебе иду. Отче святой! Соблюди их во имя Твоё, тех, которых Ты Мне дал, чтобы они были едины, как и~Мы. \textsubscript{12}~Когда Я~был с~ними в~мире, Я~соблюдал их во имя Твоё: тех, которых Ты дал Мне, Я~сохранил, и~никто из них не погиб, кроме сына погибели; да сбудется Писание. \textsubscript{13}~Ныне же к~Тебе иду и~это говорю в~мире, чтобы они имели в~себе радость Мою совершенную. \textsubscript{14}~Я~передал им слово Твоё; и~мир возненавидел их, потому что они не от мира, как и~Я не от мира. \textsubscript{15}~Не молю, чтобы Ты взял их из мира, но чтобы сохранил их от зла. \textsubscript{16}~Они не от мира, как и~Я не от мира. \textsubscript{17}~Освяти их истиною Твоею! Слово Твоё есть истина. \textsubscript{18}~Как Ты послал Меня в~мир, так и~Я послал их в~мир. \textsubscript{19}~И~за них Я~посвящаю Себя, чтобы и~они были освящены истиною. \textsubscript{20}~Не о~них же только молю, но и~о верующих в~Меня по слову их. \textsubscript{21}~Да будут все едины, как Ты, Отче, во Мне и~Я в~Тебе, так и~они да будут в~Нас едины,~--- да уверует мир, что Ты послал Меня. \textsubscript{22}~И~славу, которую Ты дал Мне, Я~дал им. Да будут едины, как Мы едины. \textsubscript{23}~Я~в~них, и~Ты во Мне; да будут совершены воедино, и~да познает мир, что Ты послал Меня и~возлюбил их, как возлюбил Меня. \textsubscript{24}~Отче! Которых Ты дал Мне, хочу, чтобы там, где Я, и~они были со Мной,~--- да видят славу Мою, которую Ты дал Мне, потому что возлюбил Меня прежде основания мира. \textsubscript{25}~Отче праведный! И~мир Тебя не познал; а~Я познал Тебя, и~эти познали, что Ты послал Меня. \textsubscript{26}~И~Я открыл им имя Твоё и~открою~--- да любовь, которой Ты возлюбил Меня, в~них будет, и~Я в~них». 


\subsection*{Вопросы по наблюдению за текстом}
\begin{enumerate}
    \item За кого Христос молится в~стихах 1-5, 6-19, 20-26? 
    
    \myline
    
    \myline
    \item Что из 5~стиха мы узнаем о~Христе? 
    
    \myline
    
    \myline
    \item Согласно 23~стиху, в~каких отношениях верующие с~Отцом? 
    
    \myline
    
    \myline
\end{enumerate}

\subsection*{Вопросы по применению текста} 
\begin{enumerate}
    \item Почему важно то, что молитва Иисуса приводится именно в~этом месте? 
    
    \myline
    
    \myline
    \item Согласно молитве Христа в~стихах 6-19, откуда вы можете знать, что вы, как ученики, находитесь «в полной безопасности»? Подтвердите свой ответ текстом из отрывка.
    
    \myline
    
    \myline
    \item В~стихах 20-21 Христос молит Отца о~единстве Церкви. Как вы думаете, когда Иисус молит о~единстве, что Он имеет в~виду? 
    
    \myline
    
    \myline
    \item Иисус молился больше об учениках, чем о~Себе. Когда вы молитесь, что преобладает в~вашей молитве (о ком вы больше молитесь: о~себе или о~других людях)? 
    
    \myline
    
    \myline
\end{enumerate}


% \newpage

\section{Евангелие от Иоанна 18:1-11. Предательство и~арест}

\textsubscript{1}~Сказав это, Иисус вышел с~учениками Своими за поток Кедрон, где был сад, в~который вошёл Сам и~ученики Его. \textsubscript{2}~Знал же это место и~Иуда, предатель Его, потому что Иисус часто собирался там с~учениками Своими. \textsubscript{3}~Итак, Иуда, взяв отряд воинов и~служителей от первосвященников и~фарисеев, приходит туда с~факелами, и~светильниками, и~оружием. \textsubscript{4}~Иисус же, зная всё, что с~Ним будет, вышел и~сказал им: «Кого ищете?» \textsubscript{5}~Ему отвечали: «Иисуса Назорея». Иисус говорит им: «Это Я». Стоял же с~ними и~Иуда, предатель Его. \textsubscript{6}~И~когда сказал им: «Это Я», они отступили назад и~пали на землю. \textsubscript{7}~Опять спросил их: «Кого ищете?» Они сказали: «Иисуса Назорея». \textsubscript{8}~Иисус отвечал: «Я сказал вам, что это Я; итак, если Меня ищете, оставьте их, пусть идут». \textsubscript{9}~Да сбудется слово, реченное Им: «Из тех, которых Ты Мне дал, Я~не погубил никого». \textsubscript{10}~Симон же Петр, имея меч, извлек его, и~ударил раба первосвященника, и~отсек ему правое ухо. Имя раба было Малх. \textsubscript{11}~Но Иисус сказал Петру: «Вложи меч в~ножны; неужели Мне не пить чаши, которую дал Мне Отец?» 

\subsection*{Вопросы по наблюдению за текстом}
\begin{enumerate}
    \item Согласно 1~стиху, куда ушёл Христос с~учениками? В~каком месте разворачиваются события, описанные в~данном отрывке? 
    
    \myline
    
    \myline
    \item Кто пришёл вместе с~Иудой, чтобы схватить Христа? 
    
    \myline
    
    \myline
    \item Как отреагировал Петр, когда пришли схватить Христа? 
    
    \myline
    
    \myline
\end{enumerate}

\subsection*{Вопросы по применению текста} 
\begin{enumerate}
    \item Почему здесь важно то, что инициатива исходит от Иисуса (см. стих~4)? Что значит, что Иисус «знал всё, что с~Ним будет»? Дайте максимально подробный ответ. 
    
    \myline
    
    \myline
    \item Поразмышляйте над стихом 11. Напишите свои размышления по этому стиху. 
    
    \myline
    
    \myline
    \item Подумайте над реакцией Христа, когда Его пришли схватить: какие добродетели Он продемонстрировал? В~чём мы можем подражать Ему? 
    
    \myline
    
    \myline
    \item Продолжите предложение: «Ничто не выделяется здесь так ярко, как…». Как что? Поясните свою мысль.
    
    \myline
    
    \myline
\end{enumerate}


% \newpage

\section{Евангелие от Иоанна 18:12-27. Суд перед Анной и~отречение Петра}

\textsubscript{12}~Тогда воины, и~тысяченачальник, и~служители иудейские взяли Иисуса, и~связали Его, \textsubscript{13}~и~отвели Его сперва к~Анне, ибо он был тесть Каиафе, который был на тот год первосвященником. \textsubscript{14}~Это был Каиафа, который подал совет иудеям, что лучше одному человеку умереть за народ. \textsubscript{15}~За Иисусом следовали Симон Петр и~другой ученик; ученик же этот был знаком первосвященнику и~вошёл с~Иисусом во двор первосвященника. \textsubscript{16}~А~Петр стоял снаружи, за дверями. Потом другой ученик, который был знаком первосвященнику, вышел, и~сказал привратнице, и~ввел Петра. \textsubscript{17}~Тут рабыня-привратница говорит Петру: «И ты не из учеников ли Этого Человека?» Он сказал: «Нет». \textsubscript{18}~Между тем рабы и~служители, разведя огонь, потому что было холодно, стояли и~грелись. Петр также стоял с~ними и~грелся. \textsubscript{19}~Первосвященник же спросил Иисуса об учениках Его и~об учении Его. \textsubscript{20}~Иисус отвечал ему: «Я говорил явно миру; Я~всегда учил в~синагоге и~в~храме, где всегда иудеи сходятся, и~тайно не говорил ничего. \textsubscript{21}~Что спрашиваешь Меня? Спроси слышавших, что Я~говорил им; вот, они знают, что Я~говорил». \textsubscript{22}~Когда Он сказал это, один из служителей, стоявший близко, ударил Иисуса по щеке, сказав: «Так отвечаешь Ты первосвященнику?» \textsubscript{23}~Иисус отвечал ему: «Если Я~сказал худо, покажи, что худо; а~если хорошо, что ты бьешь Меня?» \textsubscript{24}~Анна послал Его, связанного, к~первосвященнику Каиафе. \textsubscript{25}~Симон же Петр стоял и~грелся. Тут сказали ему: «Не из учеников ли Его и~ты?» Он отрекся и~сказал: «Нет». \textsubscript{26}~Один из рабов первосвященника, родственник тому, которому Петр отсек ухо, говорит: «Не тебя ли я~видел с~Ним в~саду?» \textsubscript{27}~Петр опять отрекся; и~тотчас запел петух. 

\subsection*{Вопросы по наблюдению за текстом}
\begin{enumerate}
    \item На суд куда и~к кому отвели Христа? 
    
    \myline
    
    \myline
    \item Каким образом обвинители пытаются уличить Христа? 
    
    \myline
    
    \myline
    \item Кто из учеников следовал на протяжении этого отрывка за Христом? 
    
    \myline
    
    \myline
    \item О~чём первосвященник спросил Христа (см. стих 19)? 
    
    \myline
    
    \myline
\end{enumerate}

\subsection*{Вопросы по применению текста} 
\begin{enumerate}
    \item Что было отличительной чертой характера Петра и~в~то же время его самым уязвимым местом? Какой урок вы можете извлечь из этого примера?
    
    \myline
    
    \myline
    \item Слуга в~доме первосвященника хотел показать, что он~--- «шишка» по сравнению с~Иисусом, он поднял на Него руку. Как вы используете служебное положение, которое вам дал Бог? Не злоупотребляете ли вы своим положением, своими полномочиями? Как вы относитесь к~подчиненным?
    
    \myline
    
    \myline
\end{enumerate}


% \newpage

\section{Евангелие от Иоанна 18:28-19:16. Суд перед Пилатом}

\textsubscript{28}~От Каиафы повели Иисуса в~преторию. Было утро; и~они не вошли в~преторию, чтобы не оскверниться, но чтобы можно было есть пасху. \textsubscript{29}~Пилат вышел к~ним и~сказал: «В чём вы обвиняете Человека Сего?» \textsubscript{30}~Они сказали ему в~ответ: «Если бы Он не был злодеем, мы не предали бы Его тебе». \textsubscript{31}~Пилат сказал им: «Возьмите Его вы и~по закону вашему судите Его». Иудеи сказали ему: «Нам не позволено предавать смерти никого». \textsubscript{32}~Да сбудется слово Иисуса, которое сказал Он, давая разуметь, какой смертью Он умрет. \textsubscript{33}~Тогда Пилат опять вошёл в~преторию, и~призвал Иисуса, и~сказал Ему: «Ты~--- Царь иудейский?» \textsubscript{34}~Иисус отвечал ему: «От себя ли ты говоришь это или другие сказали тебе обо Мне?» \textsubscript{35}~Пилат отвечал: «Разве я~иудей? Твой народ и~первосвященники предали Тебя мне. Что Ты сделал?» \textsubscript{36}~Иисус отвечал: «Царство Моё не от мира сего; если бы от мира сего было Царство Моё, то служители Мои подвизались бы за Меня, чтобы Я~не был предан иудеям; но ныне Царство Моё не отсюда». \textsubscript{37}~Пилат сказал Ему: «Итак, Ты~--- Царь?» Иисус отвечал: «Ты говоришь, что Я~— Царь. Я~на то родился и~на то пришёл в~мир, чтобы свидетельствовать об истине; всякий, кто от истины, слушает голос Мой». \textsubscript{38}~Пилат сказал Ему: «Что есть истина?» И, сказав это, опять вышел к~иудеям, и~сказал им: «Я никакой вины не нахожу в~Нем. \textsubscript{39}~Есть же у~вас обычай, чтобы я~одного отпускал вам на Пасху; хотите ли, отпущу вам Царя иудейского?» \textsubscript{40}~Тогда опять закричали все, говоря: «Не Его, но Варавву». Варавва же был разбойник. \textsubscript{1}~Тогда Пилат взял Иисуса и~велел бичевать Его. \textsubscript{2}~И~воины, сплетши венец из терна, возложили Ему на голову, и~одели Его в~багряницу, \textsubscript{3}~и~говорили: «Радуйся, Царь иудейский!»~--- и~били Его по ланитам. \textsubscript{4}~Пилат опять вышел и~сказал им: «Вот я~вывожу Его к~вам, чтобы вы знали, что я~не нахожу в~Нем никакой вины». \textsubscript{5}~Тогда вышел Иисус в~терновом венце и~в~багрянице. И~сказал им Пилат: «Вот, Человек!» \textsubscript{6}~Когда же увидели Его первосвященники и~служители, то закричали: «Распни, распни Его!» Пилат говорит им: «Возьмите Его вы и~распните, ибо я~не нахожу в~Нем вины». \textsubscript{7}~Иудеи отвечали ему: «Мы имеем закон, и~по закону нашему Он должен умереть, потому что сделал Себя Сыном Божиим». \textsubscript{8}~Пилат, услышав это слово, ещё больше устрашился. \textsubscript{9}~И~опять вошёл в~преторию, и~сказал Иисусу: «Откуда Ты?» Но Иисус не дал ему ответа. \textsubscript{10}~Пилат говорит Ему: «Мне ли не отвечаешь? Не знаешь ли, что я~имею власть распять Тебя и~власть имею отпустить Тебя?» \textsubscript{11}~Иисус отвечал: «Ты не имел бы надо Мной никакой власти, если бы не было дано тебе свыше; поэтому более греха на том, кто предал Меня тебе». \textsubscript{12}~С~этого времени Пилат искал отпустить Его. Иудеи же кричали: «Если отпустишь Его, ты не друг кесарю; всякий, делающий себя царем, противник кесарю». \textsubscript{13}~Пилат, услышав это слово, вывел вон Иисуса и~сел на судилище, на месте, называемом Лифостротон, а~по-еврейски~--- Гаввафа. \textsubscript{14}~Тогда была пятница перед Пасхой, и~час шестой. И~сказал Пилат иудеям: «Вот, Царь ваш!» \textsubscript{15}~Но они закричали: «Возьми, возьми, распни Его!» Пилат говорит им: «Царя ли вашего распну?» Первосвященники отвечали: «Нет у~нас царя, кроме кесаря». \textsubscript{16}~Тогда наконец он предал Его им на распятие. И~взяли Иисуса, и~повели. 
\subsection*{Вопросы по наблюдению за текстом}
\begin{enumerate}
    \item Куда повели Христа после иудейского суда? И~почему иудеи не смогли войти туда? 
    
    \myline
    
    \myline
    \item Согласно данному отрывку, кто судил Христа? Почему иудеи отказались судить Христа по своему закону (см. стих 31)? 
    
    \myline
    
    \myline
    \item Нашёл ли Пилат, в~чём обвинить Христа? Ответьте текстами из отрывка. 
    
    \myline
    
    \myline
    \item Что предложил Пилат народу иудейскому в~стихах 39, 40, и~как на его предложение ответил народ? 
    
    \myline
    
    \myline
    \item Какой приказ получили воины? Каким образом они издевались над Христом (см. стихи 19:1-3)? 
    
    \myline
    
    \myline
\end{enumerate}

\subsection*{Вопросы по применению текста} 
\begin{enumerate}
    \item Почему власти хотели официально казнить Христа, а~не просто убить Его? Поразмышляете и~постарайтесь дать развернутый ответ. 
    
    \myline
    
    \myline
    \item Что имел в~виду Христос, говоря, что Его царство «не от мира сего» (ст. 36)? Означает ли это, что христиане должны избегать политики? Почему? 
    
    \myline
    
    \myline
    \item В~чём обвинили Христа иудеи (см. стих 19:7)?
    
    \myline
    
    \myline
    \item Что означало быть «другом кесаря» (см. стих 19:12)? 
    
    \myline
    
    \myline
    \item Что случилось с~иудаизмом в~результате заявления первосвященников: «Нет у~нас царя, кроме кесаря»? На что указывают эти их слова? 
    
    \myline
    
    \myline
\end{enumerate}


% \newpage

\section{Евангелие от Иоанна 19:17-37. Распятие}

\textsubscript{17}~И, неся крест Свой, Он вышел на место, называемое Лобное, по-еврейски~--- Голгофа. \textsubscript{18}~Там распяли Его и~с Ним двух других, по ту и~по другую сторону, а~посредине~--- Иисуса. \textsubscript{19}~Пилат же написал и~надпись и~поставил на кресте. Написано было: «Иисус Назорей, Царь иудейский». \textsubscript{20}~Эту надпись читали многие из иудеев, потому что место, где был распят Иисус, было недалеко от города, и~написано было по-еврейски, по-гречески, по-римски. \textsubscript{21}~Первосвященники же иудейские сказали Пилату: «Не пиши: „Царь иудейский“, но что Он говорил: „Я~--- Царь иудейский“». \textsubscript{22}~Пилат отвечал: «Что я~написал, то написал». \textsubscript{23}~Воины же, когда распяли Иисуса, взяли одежды Его и~разделили на четыре части, каждому воину по части, и~хитон. Хитон же был не сшитый, а~весь тканый сверху. \textsubscript{24}~Итак, сказали друг другу: «Не станем раздирать его, а~бросим о~нём жребий, чей будет». Да сбудется реченное в~Писании: «Разделили ризы Мои между собой и~об одежде Моей бросали жребий» . Так поступили воины. \textsubscript{25}~При кресте Иисуса стояли мать Его, и~сестра матери Его, Мария Клеопова, и~Мария Магдалина. \textsubscript{26}~Иисус, увидев мать и~ученика, тут стоящего, которого любил, говорит матери Своей: «Женщина! Вот сын твой». \textsubscript{27}~Потом говорит ученику: «Вот мать твоя!» И~с~этого времени ученик этот взял её к~себе. \textsubscript{28}~После того Иисус, зная, что уже всё совершилось, да сбудется Писание, говорит: «Жажду». \textsubscript{29}~Тут стоял сосуд, полный уксуса. Воины, напоив уксусом губку и~наложив на иссоп, поднесли к~устам Его. \textsubscript{30}~Когда же Иисус вкусил уксуса, сказал: «Совершилось!» И, склонив главу, предал дух. \textsubscript{31}~Но так как тогда была пятница, то иудеи, дабы не оставить тел на кресте в~субботу,~--- ибо та суббота была днем великим,~--- просили Пилата, чтобы перебить у~них голени и~снять их. \textsubscript{32}~Итак, пришли воины, и~у первого перебили голени, и~у другого, распятого с~Ним. \textsubscript{33}~Но, подойдя к~Иисусу и~увидев, что Он уже мертв, не перебили у~Него голеней, \textsubscript{34}~но один из воинов копьем пронзил Ему бок, и~тотчас истекла кровь и~вода. \textsubscript{35}~И~видевший засвидетельствовал, и~истинно свидетельство его, и~он знает, что говорит истину, дабы вы поверили. \textsubscript{36}~Ибо это произошло, да сбудется Писание: «Кость Его да не сокрушится». \textsubscript{37}~Также и~в~другом месте Писание говорит: «Воззрят на Того, Которого пронзили». 

\subsection*{Вопросы по наблюдению за текстом}
\begin{enumerate}
    \item Где распяли Христа? Кого ещё распяли рядом со Христом? 
    
    \myline
    
    \myline
    \item Табличку с~какой надписью прикрепили ко кресту? 
    
    \myline
    
    \myline
    \item Что делали воины с~одеждой Христа? 
    
    \myline
    
    \myline
    \item Согласно стихам 25-27, кто присутствовал «при кресте» на распятии Христа? 
    
    \myline
    
    \myline
\end{enumerate}

\subsection*{Вопросы по применению текста} 
\begin{enumerate}
    \item Что означает фраза «да сбудется реченное в~Писании», которую употребляет евангелист несколько раз на протяжении этого отрывка? Как вы думаете, что он хочет этим сказать? 
    
    \myline
    
    \myline
    \item Что имел в~виду Иисус, сказав: «Совершилось!» (см. стих 30)? Что совершилось?
    
    \myline
    
    \myline
    \item Почему важно упоминание «крови и~воды» в~ст. 34? Почему для Иоанна важно указать, что Иисус точно был мертв? 
    
    \myline
    
    \myline
\end{enumerate}


% \newpage

\section{Евангелие от Иоанна 19:38-42. Погребение}

\textsubscript{38}~После этого Иосиф из Аримафеи~--- ученик Иисуса, но тайный, из страха перед иудеями,~--- просил Пилата, чтобы снять тело Иисуса; и~Пилат позволил. Он пошёл и~снял тело Иисуса. \textsubscript{39}~Пришёл также и~Никодим, приходивший прежде к~Иисусу ночью, и~принес состав из смирны и~алоэ~--- литр около ста. \textsubscript{40}~Итак, они взяли тело Иисуса и~обвили его пеленами с~благовониями, как обыкновенно погребают иудеи. \textsubscript{41}~На том месте, где Он распят, был сад, и~в~саду гробница новая, в~которой ещё никто не был положен. \textsubscript{42}~Там положили Иисуса ради пятницы иудейской, потому что гробница была близко.

\subsection*{Вопросы по наблюдению за текстом}
\begin{enumerate}
    \item Кто попросил у~Пилата тело Христа? Откуда был родом этот человек? 
    
    \myline
    
    \myline
    \item Что принес Никодим для погребения? 
    
    \myline
    
    \myline
    \item Где погребли Христа? 
    
    \myline
    
    \myline
\end{enumerate}

\subsection*{Вопросы по применению текста} 
\begin{enumerate}
    \item Рисковали ли чем-нибудь те люди, которые пришли за телом Христа к~Пилату? Поясните свой ответ. 
    
    \myline
    
    \myline
\end{enumerate}


% \newpage

\section{Евангелие от Иоанна 20:1-9. Пустая гробница и~свидетельство Петра и~другого ученика}

\textsubscript{1}~В~первый же день недели Мария Магдалина приходит к~гробнице рано, когда было ещё темно, и~видит, что камень отвален от гробницы. \textsubscript{2}~Итак, бежит и~приходит к~Симону Петру и~к другому ученику, которого любил Иисус, и~говорит им: «Унесли Господа из гробницы, и~не знаем, где положили Его». \textsubscript{3}~Тотчас вышел Петр и~другой ученик и~пошли к~гробнице. \textsubscript{4}~Они побежали оба вместе; но другой ученик бежал быстрее Петра и~пришёл к~гробнице первым. \textsubscript{5}~И, наклонившись, увидел лежащие пелены; но не вошёл в~гроб. \textsubscript{6}~Вслед за ним приходит Симон Петр, и~входит в~гробницу, и~видит одни пелены лежащие \textsubscript{7}~и~плат, который был на голове Его, не с~пеленами лежащий, но особо свитый~--- на другом месте. \textsubscript{8}~Тогда вошёл и~другой ученик, прежде пришедший к~гробнице, и~увидел, и~уверовал. \textsubscript{9}~Ибо они ещё не знали из Писания, что Ему надлежало воскреснуть из мертвых. 

\subsection*{Вопросы по наблюдению за текстом}
\begin{enumerate}
    \item Кто и~когда приходит ко гробу Христа? Какая картина разворачивается перед глазами того, кто пришёл ко гробу Христа? 
    
    \myline
    
    \myline
    \item Кто из учеников первый узнает, что Христа нет в~гробнице? Что ученики видят в~гробнице на том месте, где должно было находиться тело Христа? 
    
    \myline
    
    \myline
\end{enumerate}

\subsection*{Вопросы по применению текста} 
\begin{enumerate}
    \item Почему Иоанн уделяет такое внимание погребальным одеждам? 
    
    \myline
    
    \myline
    \item Какая разница между воскресением Иисуса и~воскресением Лазаря? 
    
    \myline
    
    \myline
    \item Прокомментируете стих 9: на что указывает здесь Евангелие и~почему?
    
    \myline
    
    \myline
\end{enumerate}


% \newpage

\section{Евангелие от Иоанна 20:10-18. Явление Христа Марии}

\textsubscript{10}~Итак, ученики опять возвратились к~себе. \textsubscript{11}~А~Мария стояла у~гробницы и~плакала. И~когда плакала, наклонилась в~гробницу \textsubscript{12}~и~видит двух ангелов, в~белом одеянии сидящих: одного у~главы и~другого у~ног, где лежало тело Иисуса. \textsubscript{13}~И~они говорят ей: «Женщина! Что ты плачешь?» Говорит им: «Унесли Господа моего, и~не знаю, где положили Его». \textsubscript{14}~Сказав это, обернулась и~увидела Иисуса стоящего; но не узнала, что это Иисус. \textsubscript{15}~Иисус говорит ей: «Женщина! Что ты плачешь? Кого ищешь?» Она, думая, что это садовник, говорит Ему: «Господин! Если ты вынес Его, скажи мне, где ты положил Его, и~я возьму Его». \textsubscript{16}~Иисус говорит ей: «Мария!» Она, обернувшись, говорит Ему по-еврейски: «Раввуни!»~--- что значит «Учитель». \textsubscript{17}~Иисус говорит ей: «Не прикасайся ко Мне, ибо Я ещё не восшёл к~Отцу Моему, а~иди к~братьям Моим и~скажи им: „Восхожу к~Отцу Моему и~Отцу вашему и~к Богу Моему и~Богу вашему“». \textsubscript{18}~Мария Магдалина идет и~возвещает ученикам, что видела Господа и~что Он это сказал ей. 

\subsection*{Вопросы по наблюдению за текстом}
\begin{enumerate}
    \item Что делала Мария, когда стояла у~гроба? Кого она там увидела? 
    
    \myline
    
    \myline
    \item За кого принимает Мария воскресшего Христа? 
    
    \myline
    
    \myline
    \item Что Мария возвещает ученикам? 
    
    \myline
    
    \myline
\end{enumerate}

\subsection*{Вопросы по применению текста} 
\begin{enumerate}
    \item На ваш взгляд, почему Мария сразу не узнает Иисуса? 
    
    \myline
    
    \myline
    \item Как вы думаете, почему Иисус просит Марию не трогать Его? 
    
    \myline
    
    \myline
    \item Каким образом Иоанн доказывает, что воскресение~--- это исторический факт? Дайте ответ, опираясь на тексты Писания. 
    
    \myline
    
    \myline
    \item Насколько сильно воскресение затрагивает вас лично? 
    
    \myline
    
    \myline
\end{enumerate}


% \newpage

\section{Евангелие от Иоанна 20:19-23. Первое явление Христа ученикам (без Фомы)}

\textsubscript{19}~В~тот же первый день недели, вечером, когда двери дома, где собирались ученики Его, были заперты из опасения перед иудеями, пришёл Иисус, и~стал посредине, и~говорит им: «Мир вам!» \textsubscript{20}~Сказав это, Он показал им руки, и~ноги, и~бок Свой. Ученики обрадовались, увидев Господа. \textsubscript{21}~Иисус же сказал им вторично: «Мир вам! Как послал Меня Отец, так и~Я посылаю вас». \textsubscript{22}~Сказав это, дунул и~говорит им: «Примите Духа Святого. \textsubscript{23}~Кому простите грехи, тому простятся; на ком оставите, на том останутся». 

\subsection*{Вопросы по наблюдению за текстом}
\begin{enumerate}
    \item Согласно повествованию евангелиста, в~какой день недели происходят события? 
    
    \myline
    
    \myline
    \item Что было с~дверями того места, где находились ученики? В~каком состоянии находились ученики Христа? 
    
    \myline
    
    \myline
\end{enumerate}

\subsection*{Вопросы по применению текста} 
\begin{enumerate}
    \item Какие первые слова были сказаны Христом ученикам (см. стих 19)? Что означает это приветствие? 
    
    \myline
    
    \myline
    \item Несмотря на событие из стиха 22, почему предпочтительнее думать, что истинное излияние Святого Духа произошло в~Пятидесятницу? В~чём же тогда заключается смысл события, описанного в~22 стихе? 
    
    \myline
    
    \myline
    \item Поразмышляйте над 23~стихом: как вы думаете, что имеет в~виду Христос? Как это относится к~вам лично? Постарайтесь дать максимально развернутый ответ. 
    
    \myline
    
    \myline
\end{enumerate}


% \newpage

\section{Евангелие от Иоанна 20:24-29. Второе явление Христа ученикам (с Фомой)} 

 \textsubscript{24}~Фома же, один из двенадцати, называемый Близнец, не был тут с~ними, когда приходил Иисус. \textsubscript{25}~Другие ученики сказали ему: «Мы видели Господа». Но он сказал им: «Если не увижу на руках Его ран от гвоздей, и~не вложу перста моего в~раны от гвоздей, и~не вложу руки моей в~бок Его, не поверю». \textsubscript{26}~После восьми дней опять были в~доме ученики Его, и~Фома с~ними. Пришёл Иисус, когда двери были заперты, стал посреди них и~сказал: «Мир вам!» \textsubscript{27}~Потом говорит Фоме: «Подай перст твой сюда и~посмотри руки Мои; подай руку твою и~вложи в~бок Мой; и~не будь неверующим, но верующим». \textsubscript{28}~Фома сказал Ему в~ответ: «Господь мой и~Бог мой!» \textsubscript{29}~Иисус говорит ему: «Ты поверил, потому что увидел Меня; блаженны не видевшие и~уверовавшие».

\subsection*{Вопросы по наблюдению за текстом}
\begin{enumerate}
    \item Какого ученика не было с~остальными, когда приходил воскресший Христос? 
    
    \myline
    
    \myline
    \item Как он отреагировал на весть о~воскресении Учителя? 
    
    \myline
    
    \myline
    \item Через сколько дней Христос вновь является ученикам? 
    
    \myline
    
    \myline
\end{enumerate}

\subsection*{Вопросы по применению текста} 
\begin{enumerate}
    \item В~чём, на ваш взгляд, заключается провал Фомы? В~чём была его проблема?
    
    \myline
    
    \myline
    \item Какими словами Фома отреагировал на воскресшего Христа? Что говорят вам его слова о~Христе (см. стих 28)? 
    
    \myline
    
    \myline
    \item Что на самом деле означает «уверовать»? Прокомментируете 29~стих, дайте развернутый ответ.
    
    \myline
    
    \myline
\end{enumerate}


% \newpage

\section{Евангелие от Иоанна 20:30-31 Цель Евангелия относительно знамений и~веры}

\textsubscript{30}~Много сотворил Иисус пред учениками Своими и~других чудес, о~которых не написано в~книге этой. \textsubscript{31}~Это же написано, чтобы вы уверовали, что Иисус есть Христос, Сын Божий, и, веруя, имели жизнь во имя Его. 

\subsection*{Вопросы по наблюдению за текстом}
\begin{enumerate}
    \item Медленно прочтите данный короткий отрывок 5~раз и~поразмышляйте над этими словами. 
    
    \myline
    
    \myline
    \item Что говорит вам 30~стих о~служении Христа? Насколько много совершил Христос чудес и~дел?
    
    \myline
    
    \myline
\end{enumerate}

\subsection*{Вопросы по применению текста} 
\begin{enumerate}
    \item Согласно данному тексту, с~какой целью евангелист Иоанн пишет Евангелие? 
    
    \myline
    
    \myline
    \item Как эти стихи связаны с~повествованием, которое вы читали выше?
    
    \myline
    
    \myline
    \item Согласно стиху 30, какую взаимосвязь можно увидеть между написанным Словом и~верой? 
    
    \myline
    
    \myline

    \item Евангелист утверждает, что вечную жизнь невозможно получить без веры во Христа. Верите ли вы во Христа как Господа и~Спасителя? Имеете ли вы жизнь вечную? Почему? 
    
    \myline
    
    \myline
\end{enumerate}


% \newpage

\section{Евангелие от Иоанна 21:1-14 Третье явление Христа ученикам}

\textsubscript{1}~После того опять явился Иисус ученикам Своим при море Тивериадском. Явился же так: \textsubscript{2}~были вместе Симон Петр, и~Фома, называемый Близнец, и~Нафанаил из Каны галилейской, и~сыновья Зеведея, и~двое других из учеников Его. \textsubscript{3}~Симон Петр говорит им: «Иду ловить рыбу». Говорят ему: «Идем и~мы с~тобой». Пошли и~тотчас вошли в~лодку и~не поймали в~ту ночь ничего. \textsubscript{4}~А~когда уже настало утро, Иисус стоял на берегу, но ученики не узнали, что это Иисус. \textsubscript{5}~Иисус говорит им: «Дети! Есть ли у~вас какая пища?» Они отвечали Ему: «Нет». \textsubscript{6}~Он же сказал им: «Закиньте сеть по правую сторону лодки~--- и~поймаете». Они закинули и~уже не могли вытащить сети от множества рыбы. \textsubscript{7}~Тогда ученик, которого любил Иисус, говорит Петру: «Это Господь». Симон же Петр, услышав, что это Господь, опоясался одеждой, ибо он был наг, и~бросился в~море. \textsubscript{8}~А~другие ученики приплыли в~лодке, ибо недалеко были от земли~--- локтей около двухсот, таща сеть с~рыбой. \textsubscript{9}~Когда же сошли на землю, видят разложенный огонь и~на нём лежащую рыбу и~хлеб. \textsubscript{10}~Иисус говорит им: «Принесите из тех рыб, которые вы теперь поймали». \textsubscript{11}~Симон Петр пошёл и~вытащил на землю сеть, наполненную большими рыбами, которых было сто пятьдесят три; и~при таком множестве не прорвалась сеть. \textsubscript{12}~Иисус говорит им: «Придите, обедайте». Из учеников же никто не смел спросить Его: «Кто Ты?»~--- зная, что это Господь. \textsubscript{13}~Иисус приходит, берет хлеб и~дает им, также и~рыбу. \textsubscript{14}~Это уже в~третий раз явился Иисус ученикам Своим по воскресении Своем из мертвых. 

\subsection*{Вопросы по наблюдению за текстом}
\begin{enumerate}
    \item В~каком месте Христос вновь является ученикам? 
    
    \myline
    
    \myline
    \item Согласно данному отрывку, каким ученикам явился Христос? Перечислите их. 
    
    \myline
    
    \myline
    \item Чем занимались ученики, когда Христос явился им? 
    
    \myline
    
    \myline
    \item Что готовит для учеников Христос (см. стих~9)? 
    
    \myline
    
    \myline
    \item Какое количество рыбы поймали ученики? 
    
    \myline
    
    \myline
\end{enumerate}

\subsection*{Вопросы по применению текста} 
\begin{enumerate}
    \item Почему, на ваш взгляд, так важна глава 21? 
    
    \myline
    
    \myline
    \item В~чём была разница между результатами, когда ученики пошли по собственной инициативе рыбачить (см. стих~3) и~когда они рыбачили по указанию Иисуса (см. стих 11)? 
    
    \myline
    
    \myline
    \item Каким образом данное повествование демонстрирует секреты плодотворного служения? Поразмышляйте и~дайте развернутый ответ.
    
    \myline
    
    \myline
\end{enumerate}


% % \newpage

\section{Евангелие от Иоанна 21:15-25 Обращение Христа к~Петру}

 \textsubscript{15}~Когда же они обедали, Иисус говорит Симону Петру: «Симон Ионин! Любишь ли ты Меня больше, нежели они?» Петр говорит Ему: «Так, Господи! Ты знаешь, что я~люблю Тебя». Иисус говорит ему: «Паси агнцев Моих». \textsubscript{16}~Ещё говорит ему в~другой раз: «Симон Ионин! Любишь ли ты Меня?» Петр говорит Ему: «Так, Господи! Ты знаешь, что я~люблю Тебя». Иисус говорит ему: «Паси овец Моих». \textsubscript{17}~Говорит ему в~третий раз: «Симон Ионин! Любишь ли ты Меня?» Петр опечалился, что в~третий раз спросил его: «Любишь ли Меня?» И~сказал Ему: «Господи! Ты всё знаешь; Ты знаешь, что я~люблю Тебя». Иисус говорит ему: «Паси овец Моих. \textsubscript{18}~Истинно, истинно говорю тебе: когда ты был молод, то препоясывался сам и~ходил куда хотел; а~когда состаришься, то прострешь руки твои, и~другой препояшет тебя и~поведет куда не хочешь». \textsubscript{19}~Сказал же это, давая разуметь, какой смертью Петр прославит Бога. И, сказав это, говорит ему: «Иди за Мною». \textsubscript{20}~Петр же, обернувшись, видит идущего за ним ученика, которого любил Иисус и~который во время вечери, приклонившись к~груди Его, сказал: «Господи! Кто предаст Тебя?» \textsubscript{21}~Его увидев, Петр говорит Иисусу: «Господи! А~он что?» \textsubscript{22}~Иисус говорит ему: «Если Я~хочу, чтобы он пребыл, пока приду, что тебе до того? Ты иди за Мною». \textsubscript{23}~И~пронеслось это слово между братьями, что ученик тот не умрет. Но Иисус не сказал ему, что не умрет, но: «Если Я~хочу, чтобы он пребыл, пока приду, что тебе до того?» \textsubscript{24}~Этот ученик и~свидетельствует об этом, и~написал это; и~знаем, что истинно свидетельство его. \textsubscript{25}~Многое и~другое сотворил Иисус; но если бы писать о~том подробно, то, думаю, и~самому миру не вместить бы написанных книг. Аминь. 

\subsection*{Вопросы по наблюдению за текстом}
\begin{enumerate}
    \item С~кем беседует Христос в~данном повествовании? 
    
    \myline
    
    \myline
    \item Какой вопрос и~сколько раз задает Христос Своему ученику? К~какому служению Он призвал Своего ученика? 
    
    \myline
    
    \myline
    \item О~чём говорят стихи 18-19? Прокомментируйте их. 
    
    \myline
    
    \myline
\end{enumerate}

\subsection*{Вопросы по применению текста} 
\begin{enumerate}
    \item Как вы думаете, почему Петр опечалился (см. стих 17)?
    
    \myline
    
    \myline
    \item Раскрываемая в~этом отрывке истина: следовать за Иисусом и~любить Его~--- означает взять на себя обязанность заботиться о~Его людях. Каким образом вы практикуете эту истину? 
    
    \myline
    
    \myline
    \item Что 25~стих говорит о~природе Христа? 
    
    \myline
    
    \myline
    
    \item Какова ваша главная цель в~жизни? Откуда вы это знаете? 
    
    \myline
    
    \myline
\end{enumerate}


\end{document}